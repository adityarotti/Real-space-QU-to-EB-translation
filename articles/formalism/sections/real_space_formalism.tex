\section{Polarization primer} \label{sec:pol-primer}
\subsection{Heuristic, real-space construction of E/B fields on the sphere} \label{sec:qu2eb_heuristic}

CMB polarization is measured in terms of Stokes parameters, time averages of the linear polarization of the electric field along cartesian axes perpendicular to the line of sight.\footnote{Throughout we use the conventions of HEALPix \cite{healpix_primer}, measuring the polarization angle East of South.} Thus Stokes $Q$ and $U$ depend on the choice of the local coordinate system, and a rotation by an angle $\psi$ around the line of sight transforms them as:
%
\beq \label{eq:qu-rot}
\fqu' = \begin{bmatrix} \cos{2 \psi} &  \sin{2 \psi} \\ -\sin{2\psi} & \cos{2 \psi} \end{bmatrix} \fqu \,.
\eeq
Equivalently, the object $_{\pm 2}{X}(\hat{n}) = Q(\hat{n}) \pm i U (\hat{n})$ transforms as ${}_{\pm 2}f' = e^{\mp 2i\psi} {}_{s}f$ and hence forms a spin $\pm$2 field \cite{Zaldarriaga1997}.

The standard construction of $E$ and $B$ fields arise from the desire to have a coordinate independent description of the polarization. This follows from operations that raise (or lower) the spin of the field ${}_{\pm2}{X}$ to construct scalar fields.  By understanding the transformation properties of the Stokes parameters and those of the Euler angles on the sphere, we can already construct a heuristic argument for what these operations must look like in real space. We consider the contribution to a scalar field at $\hat n_e$ from the polarization field at $\hat n_q$. 

\fig{fig:euler_angles} shows that the transformation of the local coordinate system between the two positions can be described by a counter-clockwise rotation around the local $\hat n_q$ (unit radial vector at $(\theta_q,\phi_q)$ pointing outward) by angle $\alpha$, parallel transport by angle $\beta$ along the shorter geodesic and a counter-clockwise rotation around $\hat n_q$ by $-\gamma$.  This corresponds to a rotation by Euler angles $(\alpha,\beta,-\gamma)$ in the $z-y_1-z_2$ convention.\footnote{The Euler angles in the more standard $z-y-z$ convention are related to those in the $z-y_1-z_2$ convection by the following rule: $(\alpha,\beta,\gamma)_{z-y-z} =(\gamma,\beta,\alpha)_{z-y_1-z_2}$ \cite{varshalovich}.}
%
\begin{figure}%[!hbt]
\centering
\includegraphics[width=0.5\columnwidth]{euler.pdf}
\caption{This figure depicts the Euler angles in the $z-y_1-z_2$ convention. The cartesian coordinate axes shown in dark solid green are those that lie in the tangent plane at location $\hat{n}_q = (\theta_q, \phi_q)$ while those shown in dark solid red are the ones that lie in the tangent plane at location $\hat{n}_e = (\theta_e, \phi_e)$. The blue axes represent the parallel transport along the geodesic connection between the two locations $\hat{n}_q$ and $\hat{n}_e$ on the sphere.}
\label{fig:euler_angles}
\end{figure}
%

We consider the impact of local rotations  the Stokes parameters and on these Euler angles. Rotating the cartesian coordinates in the tangent plane at location $\hat{n}_q$ by an angle $\psi$ about the local $\hat{z}_q$ axis, the Stokes parameters in the new coordinate system relate to those in the original coordinate system as:
$\mathcal{R}_{\hat{z}_q}(\psi)[{}_{+2}X(\hat{n}_q)] =  {}_{+2}X(\hat{n}_q) e^{-i2\psi} $.
This same rotation by $\psi$ alters the Euler angle $\alpha_{qe}$, the angle that aligns the $x$-axis at $\hat{n}_q$ along the geodesic to the location $\hat{n}_e$, so that: $\mathcal{R}_{\hat{z}_q}(\psi)[\alpha_{qe}] = \alpha_{qe} - \psi$.  Therefore one can see that: $\mathcal{R}_{\hat{z}_q}(\psi)[e^{-i2\alpha_{qe}}] =  e^{-i2\alpha_{qe}} e^{i2\psi}$.

Given these transformation properties, the combination ${}_{+2}X(\hat{n}_q)e^{-i2\alpha_{qe}}$ is invariant under rotations and hence must be spin-0 or scalar quantity by definition:
\beq
\mathcal{R}_{\hat{z}_q}(\psi)[{}_{+2}X(\hat{n}_q)e^{-i2\alpha'_{qe}}] = {}_{+2}X(\hat{n}_q) e^{-i2\alpha_{qe}} \,. \label{eq:invariant}
\eeq
Thus we can build a scalar polarization quantity out of such combinations.
Further note that both $Q$ and $\cos2 \alpha$ have even parity since they do not change sign when $\hat{x} \rightarrow -\hat{x}$ (or $\hat{y} \rightarrow -\hat{y}$).  Meanwhile $U$ and $\sin 2 \alpha$ change sign under this transformation and hence have odd parity. The real part of the function  ${}_{+2}X(\hat{n}_q)e^{-i2\alpha_{qe}}$ must have even parity, because it is composed of terms containing $Q\cos{2\alpha}$ and  $U\sin{2 \alpha}$ which are product of functions with the same parity. Similarly, the imaginary part of the function must have odd parity, because it is composed of $Q\sin{2 \alpha}$ and $U\cos{2\alpha}$ which are product of functions with opposite parity.  Therefore we can make the association that contributions to  $(E+iB)(\hat n_e)$ must be proportional to $ {}_{+2}X(\hat{n}_q) e^{-i2\alpha_{qe}}$.


The same rotation $\mathcal{R}_{\hat{z}_q}(\psi)$ leaves the Euler angle $|\beta_{qe}|$ unaltered (it measures the angular distance between the points).  Thus we further conclude that the contribution to $(E+iB)(\hat n_e)$ from the position $n_q$ must have the generic form:
\beq
{}_{+2}X(\hat{n}_q) f(\beta_{qe})  e^{-i2\alpha_{qe}}
\eeq
for some real function $f$.  Note that when the two locations coincide ($\beta_{qe}=0$) then  $\alpha_{qe}=0,2\pi,4\pi,\dots$, implying $E + iB \propto Q+iU$.  This is a contradiction because $Q+iU$ does not transform as a spin-0 field under local rotations, hence we must have $f(\beta_{qe} = 0 ) = 0$. This immediately implies that the $E/B$ fields are necessarily defined non-locally.  A similar contradiction arises when the two locations are diametrically opposite, $\beta_{qe} = \pi$ and therefore we also require that $f(\beta_{qe} = \pi ) = 0$.  Any such function $f$ will let us construct $E/B$-like scalar fields.  Below we derive the particular one that gives rise to our familiar $E/B$ modes.

Note that this type of real-space construction can be generalized to transform a field of any spin to a field of any other spin, not just two and zero, and so we can use a similar construction (in the opposite direction) to transform $E/B$ maps back to the Stokes parameters (i.e. transforming spin-0 fields to spin-2).

%--------------------------------------------------------
\subsection{Standard $E$/$B$ fields}


The standard construction of $E/B$ fields is expressed in terms of the spin-raising and spin-lowering operators and this operation is usually carried out in harmonic space. The spin-raising operator ($\eth$) applied to a field of spin-s $_{s}g$, results in a field with spin-$(s+1)$: $(\eth _{s}g)' = e^{-i(s+1)\psi}(\eth _{s}g)$  \cite{goldberg67}.  The complementary spin-lowering operator $(\bar{\eth})$  similarly results in a field with spin-$(s-1)$: $(\bar{\eth} _{s}g)' = e^{-i(s-1)\psi}(\bar{\eth} _{s}g)$.  The complex spin-0 scalar now arise from these spin lowering/raising operations of the spin-2 fields ${_{\pm 2}X}$ as follows:
%
\begin{subequations}\label{eq:ebdef}
\beqry
\mathcal{E}(\hat{n}) + i \mathcal{B}(\hat{n}) &=& -\bar{\eth}^2 _{+ 2}\bar{X}(\hat{n}) \,,\label{eq:ebdef_lower}\\
\mathcal{E}(\hat{n}) - i \mathcal{B}(\hat{n}) &=& -{\eth}^2 _{-2}\bar{X}(\hat{n}) \,.
\eeqry
\end{subequations}
%
The $\cal E/B$ fields are defined locally at point $\hat n$ in terms of the operators $\eth$ and $\bar \eth$. It is possible to decompose the complex field $_{\pm 2}\bar{X}$ into spin spherical harmonic functions: ${}_{\pm 2}\bar{X}(\hat{n}) = \sum_{\ell m} {}_{\pm 2} \tilde X_{\ell m} {}_{\pm 2}Y_{\ell m}(\hat{n})$. Applying the spin raising and lowering operators on the spin spherical harmonic functions leads to the following identities \cite{goldberg67}:
%
\begin{subequations}\label{eq:spinopylm} 
\beqry
\eth _s Y_{lm}(\hat{n}) &=& \sqrt{(\ell-s)(\ell+s+1)} _{s+1} Y_{lm}(\hat{n}) \,, \\
\bar{\eth} _s Y_{lm}(\hat{n}) &=& -\sqrt{(\ell+s)(\ell-s+1)} _{s-1} Y_{lm}(\hat{n}) \,, 
\eeqry
\end{subequations}
%
where $_s Y_{lm}(\hat{n}) $ denote the spin-s spherical harmonics.

From the definition of $\mathcal{E/B}$, the spin spherical harmonic decomposition of ${}_{\pm2}\bar{X}$, and the identities given in \eq{eq:spinopylm}, it follows that the scalar fields $\mathcal{E}/\mathcal{B}$ are given by the equations:
%
\beq \label{eq:pseudo}
\mathcal{E}(\hat{n}) = \sum_{\ell m} a^{E}_{\ell m} \sqrt{\frac{(\ell+2)!}{(\ell-2)!}} Y_{\ell m} (\hat{n})\qquad;\qquad
\mathcal{B}(\hat{n})  =\sum_{\ell m} a^{B}_{\ell m} \sqrt{\frac{(\ell+2)!}{(\ell-2)!}} Y_{\ell m} (\hat{n}) \,,
\eeq
%
where the harmonic coefficients $a^{E}_{\ell m}$ and  $a^{B}_{\ell m}$ relate to the harmonic coefficients of the spin-2 polarization field via the following equations:
%
\beq\label{eq:x2eb}
a^{E}_{\ell m} = -\frac{1}{2} \Big[ {}_{+2}\tilde{X}_{\ell m} + {}_{-2}\tilde{X}_{\ell m} \Big]\qquad;\qquad a^{B}_{\ell m} = -\frac{1}{2i} \Big[ {}_{+2}\tilde{X}_{\ell m} - {}_{-2}\tilde{X}_{\ell m} \Big] \,.
\eeq
%
In the remainder of this article, we will work with the scalar $E$ and pseudo scalar $B$ fields, defined by: 
%
\beq \label{eq:realeb}
E(\hat{n}) = \sum_{\ell m} a^{E}_{\ell m} Y_{\ell m} (\hat{n})\qquad;\qquad B(\hat{n})  =\sum_{\ell m} a^{B}_{\ell m} Y_{\ell m} (\hat{n}) \,.
\eeq
%
These $E/B$ fields are merely versions of $\mathcal{E}/\mathcal{B}$ that downweight higher-$\ell$ modes  (i.e. their spherical harmonic coefficients of expansion are reduced by the factor $[{(\ell-2)!}/{(\ell+2)!}]^{1/2}$).

\subsection{Matrix notation} \label{sec:mat_pol_intro}
Our derivations of the real space operators are more transparent in a matrix-vector notation.\footnote{While we work with the matrix and vector sizes given in terms of some pixelization parameter $\rm N_{\rm pix}$, all the relations are equally valid in the continuum limit attained by allowing $\rm N_{\rm pix}\rightarrow \infty$}
We introduce a matrix that encodes spin spherical harmonic basis vectors:
%
\beq
{}_{|s|}\mathcal{Y}= \bmat _{+s}Y & 0 \\ 0 & _{-s}Y \emat _{2 \rm N_{\rm pix} \times 2 \rm N_{\rm alms}} ;\qquad  {}_{|s|}\mathcal{Y}^{\ddagger}= \Delta \Omega \bmat _{+s}Y^{\dagger} & 0 \\ 0 & _{-s}Y^{\dagger} \emat _{2 \rm N_{\rm alms} \times 2 \rm N_{\rm pix}} \,,
\eeq
%
where $s$ denotes the spin of the basis functions and our definition of ${}_{|s|}\mathcal{Y}^{\ddagger}$ differs from the conventional conjugate transpose operation by the factor $\Delta \Omega$.  We introduce this to ensure the orthonormality of these operations on the discretized sphere when the pixel size is sufficiently small, ${}_{|s|}\mathcal{Y}^{\ddagger} {}_{|s|}\mathcal{Y} = I_{2 \rm N_{\rm alms} \times2 \rm N_{\rm alms}}$, and also maintain the standard definition of spherical harmonics. 

We will be working with cases $s \in [0,2]$. Each column of ${}_{|s|}\mathcal{Y}$ maps to a specific harmonic basis function (i.e. indexed by $\ell m$) and each row maps to a pixel on the sphere. This matrix is not square in general: the number of rows is determined by the pixelization and the number of columns is set by the number of basis functions (e.g. determined by the band limit).

We now define the different polarization data vectors and their representation in real and harmonic space as follows\footnote{We adopt a convention in which real space quantities are denoted by bar-ed variable while those in harmonic space are denoted by tilde-ed variables.}
%
\beqrys
\bar{S} &=& \bmat E \\ B  \emat_{2 \rm N_{\rm pix} \times 1};\qquad \bar{X} = \bmat _{+2}X \\ _{-2}X \emat_{2 \rm N_{\rm pix} \times 1};\qquad \bar{P} =\fqu_{\tiny {2 \rm N_{\rm pix} \times 1}} \,, \\
\tilde{S} &=& \bmat a^{E} \\ a^{B} \emat _{2 \rm N_{\rm alms} \times 1};\qquad \tilde{X} = \bmat _{+2} \tilde{X} \\ _{-2} \tilde{X} \emat_{2 \rm N_{\rm alms} \times 1} \,.
\eeqrys
%
The symbols have the same meaning as in \sec{sec:pol-primer}, except that the subscript ${\ell m}$ for the spherical harmonic coefficients is suppressed for cleaner notation.

We define an operator that transforms between different representations of the polarization field (i.e. from $Q,U$ to $_{\pm2}\bar{X}$ and back):
%
\beqrys
\bar T &=& \qutox_{2 \rm N_{\rm pix} \times 2 \rm N_{\rm pix}} ;\qquad \bar T^{-1} = \frac{1}{2} \bar T^{\dagger} \,, \\
\tilde T &=& -\qutox_{2 \rm N_{\rm alms} \times 2 \rm N_{\rm alms}};\qquad \tilde T^{-1} = \frac{1}{2} \tilde T^{\dagger} \,,
\eeqrys
%
The sign conventions we have chosen matches that of HEALPix.
Using the data vectors and the matrix operators defined above we can now express, in compact notation, the forward and inverse relations between different representations of the polarization data vectors via the following equations:
%
\begin{subequations} \label{eq:pol_data_relns}
  \beqry
  \bar{X} &= \bar T  \bar{P} ; &\qquad \bar{P} = \frac{1}{2} \bar T^{\dagger}  \bar{X} \,, \\
  \tilde{X} &= \tilde T \tilde{S}; &\qquad \tilde{S} = \frac{1}{2}\tilde T^{\dagger} \tilde{X} \,.
  \eeqry
  Meanwhile the spherical harmonic transforms are written as:
  \beqry
  \bar X &=  {{}_2\mathcal{Y}}  \tilde X; &\qquad \tilde X ={{}_2\mathcal{Y}}^{\ddagger}  \bar X  ; \\
  \bar S &=  {{}_0\mathcal{Y}} \tilde S; &\qquad  \tilde S =  {{}_0\mathcal{Y}}^{\ddagger} \bar S \,.
  \eeqry
\end{subequations}
%
Finally we introduce the operators that project harmonic space data vector to the $E$ or $B$ subspace:
%
\begin{subequations} \label{eq:har_eb_op}
\beqry
\tilde O_E &=& \bmat \mathbb{1} & \mathbb{0} \\ \mathbb{0} & \mathbb{0} \emat _{2 \rm N_{\rm alms} \times 2 \rm N_{\rm alms} }; \qquad \tilde S_E = \tilde O_E  \tilde S ,\\
\tilde O_B &=& \bmat \mathbb{0} & \mathbb{0} \\ \mathbb{0} & \mathbb{1} \emat _{2 \rm N_{\rm alms} \times 2 \rm N_{\rm alms} }; \qquad \tilde S_B = \tilde O_B  \tilde S .
\eeqry
\end{subequations}
%
Note that these harmonic space matrices are idempotent ($\tilde O_E  \tilde O_E = \tilde O_E;  \tilde O_B  \tilde O_B= \tilde O_B$), orthogonal ($\tilde O_E  \tilde O_B = \mathbb{0}$) and sum to the identity matrix ($\tilde O_E + \tilde O_B = \mathbb{1}$).

The above relations for these harmonic space operators are exactly valid.  In the following sections we derive the real space analogues ($O_E,O_B$) of these harmonic space operators.

\section{Real space polarization operators} \label{sec:real_space_operators}
\subsection{Evaluating scalars $E/B$ from Stokes $Q/U$}\label{sec:qu2eb}
In \sec{sec:pol-primer} we described the conventional procedure of computing the scalar fields $E/B$ from the Stokes parameters $Q/U$. 
In this section we derive the real space operators which can be used to directly evaluate the scalar fields $E$/$B$ on the sphere.  We use the vector-matrix notation introduced in \sec{sec:mat_pol_intro} to write down an operator equation relating the real space vector of scalars \vs to the Stokes polarization vector \vp{}:
%
\beqrys
\bar{S} &=& {{}_0\mathcal{Y}} \, \tilde T^{-1}  \, {{}_2\mathcal{Y}^{\ddagger}} \, \bar T  \bar{P}
= \frac{1}{2} {{}_0\mathcal{Y}} \, \tilde T^{\dagger} {{}_2\mathcal{Y}^{\ddagger}} \, \bar T \bar{P} \,,   \\
&=&  \bar O \bar{P}. \label{eq:qu2eb_op}
\eeqrys
%
The explicit form of the real space operator $\bar O$ can be derived by contracting over all the matrix operators.\footnote{While this manuscript was in preparation, work appeared in \cite{2018JCAP...05..059L} that implements the similar idea of real-space transformation of Stokes $Q/U$ parameters to their $E/B$ counterparts in real space, but we treat it here in much more detail.}
This procedure is explicitly worked out in the following set of equations:
%
\beqrys
\bar{O} &=& \frac{1}{2} {{}_0\mathcal{Y}}\, \tilde T^{\dagger} {{}_2\mathcal{Y}^{\ddagger}} \, \bar T \,, \\
&=& -0.5 \Delta \Omega \yzmat{e} \qutoxd \ymatc{q} \qutox   \,, \\
&=& -0.5 \Delta \Omega \begin{bmatrix} \sum ({}_{0}Y_e ~{}_{2}Y^{\dag}_q  +  {}_{0}Y_e~ {}_{-2}Y^{\dag}_q) & {\rm i}  \sum ({}_{0}Y_e ~ {}_{2}Y^{\dag}_q - {}_{0}Y_e ~{}_{-2}Y^{\dag}_q)  \\  - {\rm i} \sum  ({}_{0}Y_e ~ {}_{2}Y^{\dag}_q - {}_{0}Y_e~ {}_{-2}Y^{\dag}_q) & \sum ({}_{0}Y_e~ {}_{2}Y^{\dag}_q + {}_{0}Y_e ~{}_{-2}Y^{\dag}_q)  \end{bmatrix} \,, \label{eq:qu2eb_ker_1}
\eeqrys
%
where the symbol ${}_{0}Y_e$ is used to denote the sub-matrix ${}_{0}Y_{\hat{n}_e \times \ell m} \equiv {}_{0}Y_{\ell m}(\hat{n}_e)$, the symbol ${}_{\pm 2}Y^{\dag}_q$ is used to denote the transposed conjugated matrix ${}_{\pm 2}Y^*_{\ell m \times \hat{n}_q} \equiv {}_{\pm 2}Y^*_{\ell m}(\hat{n}_q)$ and the summation is over the multipole indices $\ell,m$. As before, we use the notation that the index $e$ denotes the location where the scalar fields are being evaluated, and the index $q$ denotes the location from which  the Stokes parameters are being accessed. Using the conjugation properties of the spin spherical harmonic functions it can be shown that the following identity holds true:
%
\beq
 \left [\sum_{\ell m} {}_{0}Y_{\ell m}(\hat{n}_e){}_{+2}Y^*_{\ell m}(\hat{n}_q)\right]^* = \sum_{\ell m} {}_{0}Y_{\ell m}(\hat{n}_e){}_{-2}Y^*_{\ell m}(\hat{n}_q) \,,
 \eeq
 %
where the terms on either side of the equation are those that appear in \eq{eq:qu2eb_ker_1}. Note that the operator $\bar{O}$ is real as one expects, since each sub-matrix in \eq{eq:qu2eb_ker_1} is formed by summing a complex number and its conjugate. 

\eq{eq:qu2eb_ker_1} already presents a real space operator, but it is not in a form which can be practically implemented. To proceed, we use the fact that the $m$ sum over the product of two spin spherical harmonic functions can be expressed as a function of the Euler angles \cite{varshalovich}:
%
\beq \label{eq:sum_spin_shf}
 \sum_{m}{{}_{s_1}Y}^*_{\ell m}(\hat{n}_i)\,{{}_{s_2}Y}_{\ell m}(\hat{n}_j) = \sqrt{\frac{2\ell+1}{4 \pi}} {{}_{s_2}}Y_{\ell \,-s_1}(\beta_{ij},\alpha_{ij}) e^{- i s_2 \gamma_{ij}} \,,
\eeq
%
where $(\alpha_{ij}, ~\beta_{ij}, \gamma_{ij})$ denote the Euler angles that specifically transform $(i \rightarrow j)$ so that the coordinate system at $\hat{n}_i$ aligns with the coordinate system at $\hat{n}_j$\footnote{The sense of the rotation becomes more obvious when this equation is written in terms of the Wigner-$D$ functions.}. Using this identity, the different parts of the real space operator $\bar{O}$  (from eq.~\ref{eq:qu2eb_ker_1}) are completely specified by the following complex function:
%
\begin{subequations}\label{eq:qu2eb_gen_kernel}
\beqry
\mathcal{M}( \hat{n}_e, \hat{n}_q)  &=& \mathcal{M}_{r} + i \mathcal{M}_{i}  \,,\nonumber \\ 
 &=&\sum_{\ell m} {{_0}Y}_{\ell m}(\hat n_e) \, {{_{-2}}Y}^*_{\ell m}(\hat n_q) = \sum_{\ell} \sqrt{\frac{2\ell+1}{ 4 \pi}}{{_0Y}_{\ell 2}}(\beta_{qe},\alpha_{qe})\,,\\
&=&  \Big [ \cos(2 \alpha_{qe}) + i \sin(2 \alpha_{qe} ) \Big]   \sum_{\ell=\ell_{\rm min}}^{\ell_{\rm max}} {\frac{2\ell+1}{ 4 \pi}} \sqrt{\frac{(\ell-2)!}{(\ell+2)!}}P_{\ell}^2 (\cos\beta_{qe}) \,, \label{eq:rad_ker_queb} \\
\mathcal{M}(\beta_{qe}, \alpha_{qe})  &=&  \Big [ \cos(2 \alpha_{qe}) + i \sin(2 \alpha_{qe} ) \Big] \quad {{}_{\mm}f}(\beta_{qe},\ell_{\rm min},\ell_{\rm max}) \,, 
\eeqry
\end{subequations}
%
where we have used the identity in \eq{eq:sum_spin_shf} to simplify the product of the spherical harmonic functions. Note that the function depends only on two out of the three Euler angles.  The azimuthal part depends only on the Euler angle $\alpha_{qe}$ and its harmonic transform has no multipole $\ell$ dependence.  The azimuthal part is the crucial operation that translates between different spin representation of CMB polarization. The radial part $f(\beta_{qe})$ depends only on the angular separation between locations and completely incorporates all the multipole $\ell$ dependence. Recall that we had guessed the general form of the kernel using simple heuristic arguments in \sec{sec:qu2eb_heuristic}. Here we have rigorously derived the exact form of the function $f(\beta)$. Studying the $P_{\ell}^2$ functions in the limits $\beta \rightarrow 0,\pi$ it can be shown that $f(\beta)$ vanishes at $\beta=0,\pi$, which we had argued is a crucial property to yield a field of correct spin.

Employing \eq{eq:qu2eb_gen_kernel} to simplify the product of spherical harmonic functions in \eq{eq:qu2eb_ker_1}, the real space operator $\bar{O}$ can now be cast in this more useful form:
%
\beq\label{eq:op_qu2eb_rad}
\bar O =- \Delta \Omega \bmat  \mathcal{M}_{r} & \mathcal{M}_{i} \\  -\mathcal{M}_{i}  & \mathcal{M}_{r} \emat = - \Delta \Omega {{}_{\mm}f}(\beta_{qe},\ell_{\rm min},\ell_{\rm max})\bmat \cos(2 \alpha_{qe}) & \sin(2\alpha_{qe})\\  -\sin(2 \alpha_{qe})  & \cos(2 \alpha_{qe}) \emat \,,
\eeq
%
where $(\alpha_{qe}, ~\beta_{qe}, \gamma_{qe})$ denote the Euler angles which rotate the local cartesian system at $\hat{n}_q$ (location where Stokes parameters are accessed) to the cartesian system at  $\hat{n}_e$ (location where the scalar fields are evaluated).

It turns out that the real space operations can \emph{also} be expressed in terms of the inverse rotation.  These two expressions lead to two sets of kernels that are conceptually different, although they both ultimately yield the same mathematical result. The first set act like Green's functions, where, for example, a pixel of the Stokes parameters broadcasts or radiates an $E$/$B$ field.  The second set act like a convolving beams, gathering Stokes contributions to the $E/B$ fields at a point. On the equator, or in the flat sky approximation, these kernels are identical and this distinction is immaterial.  On the curved sky it matters, and the kernels are especially different near the poles.


\paragraph{Radiation kernel.} The expression above, based on the $\hat n_q \rightarrow \hat n_e$ rotation,  we call the \textit{radiation kernel}.  It allows us, like a Green's function, to evaluate the $E/B$ field contribution due to a single Stoke parameter ``charge'' at a fixed location. The total $E/B$ maps can then be thought of as the superposed radiation emerging from Stokes charges across the sphere. In this picture, we are effectively in the frame of the Stokes charge ${}_{\pm2}X$ and evaluating its contribution to the complex spin-0 scalar field $E+iB$ across the sphere. This one-to-many mapping from a point in the spin-2 Stokes field to the complex spin-0 (scalar) field across the sphere is graphically represented by the blue circle in \fig{fig:planar_euler_angles}.

The $E$/$B$ contribution from the Stokes parameters at some location $\hat{n}_q$ is given by the following expression (\eq{eq:op_qu2eb_rad} and \eq{eq:qu2eb_op}):
%
\beq  \label{eq:qu2eb_radiation_explicit}
\bar{S}_q(\hat n_e) = \bmat E_e \\ B_e  \emat_{q} =- {{}_{\mm}f}(\beta_{qe},\ell_{\rm min},\ell_{\rm max})\bmat \cos(2 \alpha_{q e}) & \sin(2\alpha_{q e})\\  -\sin(2 \alpha_{q e})  & \cos(2 \alpha_{q e}) \emat  \bmat Q_{q} \\ U_{q}  \emat \Delta \Omega \,.
\eeq
%
The total map can be simply evaluated by summing over the contribution from the Stokes parameters at each location $\hat{n}_q$: $\bar{S} = \sum_{q=1}^{N_{\rm pix}} \bar{S}_q$. This operation can be cast concisely as:
%
\begin{subequations} \label{eq:qu2eb_radiation_concise}
\beqry 
\left[E + iB\right](\hat{n}_e) &=& -\Delta \Omega  \sum_{q=1}^{N_{\rm pix}} \Big[ {}_{+2}X(\hat{n}_{q}) e^{-i2\alpha_{q e}} \Big]  {{}_{\mm}f}(\beta_{q e}) \,, \\
&=& \Delta \Omega \sum_{q=1}^{N_{\rm pix}} {}_{+2}X(\hat{n}_{q}) \,  \mathcal{M}_{G}(\hat n_q) \,,
\eeqry
\end{subequations}
%
where $\Delta \Omega$ denotes the pixel area and the last line is a simple scalar multiplication between complex numbers.  The radiation kernel is then: $ \mathcal{M}_{G} =- \mm(\beta_{qe},\alpha_{qe})^*$ which can be thought of as the Green's function of the operator, since $[E +iB]= \mathcal{M}_{G}$ is the spin-0 scalar field generated from the delta-function Stokes field $[Q+iU] = [\delta(\hat{n}-\hat{n}_q)/{\Delta \Omega} + i0]$. We display the kernel later in \fig{fig:vis_kernel}.
 

%
\begin{figure}[!t]
\centering
\includegraphics[width=0.5\columnwidth]{radiation_convolution.pdf}
\caption{The local cartesian coordinates $(\hat{x},\hat{y})$ are drawn on the red circle(sphere), representative of the coordinate dependence of the Stokes parameters. The two sets of dotted lines drawn at representative points denote great circles, one which passes through the central point labelled `$e$' and the other chosen such that the two have locally orthogonal tangent vectors $(\hat{r},\hat{\phi})$. The angle $\alpha_{qe}$ defines a rotation operator that aligns the local $\hat{x}$ with $\hat{r}$.   \textit{Radiation kernel:} We can compute the contribution from the Stokes parameter at `$q$' to all the points on the blue circle and this is a function of the Euler angle $\alpha_{qe}$. \textit{Convolution kernel:} The resultant scalar field at `$e$' can also be evaluated by summing over the contribution from all the Stokes parameters on the red circle. This convolution is performed with kernels which are defined in term of the Euler angle $\gamma_{eq}$.}
\label{fig:planar_euler_angles}
\end{figure}
%

\paragraph{Convolution kernel.} We can also formulate the real space operator as a convolution operation, where the scalar field at $\hat n_e$ gathers contributions from the Stokes fields.  This is based around the inverse rotation from the previous section (to align the coordinate system at $\hat{n}_e$ with that at $\hat{n}_q$).  The inverse rotation Euler angles relates to the forward rotation Euler angles by the following relations: $\alpha_{eq}=-\gamma_{qe}$, $\beta_{eq} = -\beta_{qe}$ and  $\gamma_{eq} =-\alpha_{qe}$. Since the kernel depends on the cosine of the Euler angle $\beta$, it is immune to changes in its sign. The operator equation can be expressed as a function of the Euler angle $\gamma_{eq}$ as follows:
%
\beq \label{eq:qu2eb_convolution_explicit}
\bmat E_e \\ B_e  \emat =- \Delta \Omega\sum_{q=1}^{N_{\rm pix}}{{}_{\mm}f}(\beta_{eq},\ell_{\rm min},\ell_{\rm max})\bmat \cos(2 \gamma_{eq}) & -\sin(2\gamma_{eq})\\  \sin(2 \gamma_{eq})  & \cos(2 \gamma_{eq}) \emat  \bmat Q_q \\ U_q  \emat \,,
\eeq
%
This formulation of the real space operator can be interpreted as integrating at some fixed location $\hat{n}_e$ the $E/B$ mode contribution arising from the Stokes parameters at all location $\hat{n}_q$ on the sphere. This operation can be expressed more concisely as follows:
%
\begin{subequations} \label{q:qu2eb_convolution_concise}
\beqry 
[E + iB](\hat{n}_e) &=& - \Delta \Omega \sum_{q=1}^{N_{\rm pix}}{{}_{\mm}f}(\beta_{eq},\ell_{\rm min},\ell_{\rm max}) {\Bigg( e^{i2 \gamma_{eq}}   {}_{+2}X (\hat{n}_q) \Bigg)}, \label{eq:qu2eb_physical}\\
&=& \Bigg\lbrace \mathcal{M}_{B} \star {}_{+2}X \Bigg\rbrace(\hat{n}_e) \,, \label{eq:qu2eb_convolution} 
\eeqry
\end{subequations}
%
where $\star$ denotes a convolution and $\mathcal{M}_{B} = -\mm(\beta_{eq},\gamma_{eq})$.  When $\mm$ is expressed as a function of the Euler angle $\gamma_{eq}$ it can be thought of as an effective instrument beam pointing to the direction $\hat{n}_e$. This many-to-one mapping from the spin-2 Stokes field on the sphere to the complex spin-0 (scalar) field at a point on the sphere is graphically represented in \fig{fig:planar_euler_angles}. We display the kernel in \fig{fig:vis_kernel}.
%--------------------------------------------------------




%--------------------------------------------------------
\subsection{Evaluating Stokes $Q$/$U$ from scalar $E$/$B$}\label{sec:eb2qu}
The real space operator which translates $E$/$B$ fields to Stokes parameters $Q$/$U$ can be derived using a similar procedure. Expressed in the matrix-vector notation, the inverse operator is given by the following equation:
%
\begin{subequations}
\beqry
\bar{P} &=& \bar{T}^{-1} {{}_2\mathcal{Y}}\, \tilde T {{}_0\mathcal{Y}^{\ddagger}}\bar{S} = \frac{1}{2} \bar{T}^{\dagger} {{}_2\mathcal{Y}} \,\tilde T {{}_0\mathcal{Y}^{\ddagger}}\bar{S}\,,  \\
&=&  \bar O^{-1} \bar{S}\,.
\eeqry
\end{subequations}
%

The inverse operator expressed in terms of the function $\mm$ given in \eq{eq:qu2eb_gen_kernel} is given by the following equation:
%
\beq
{\bar O}^{-1}=- \Delta \Omega\bmat \mathcal{M}_{r} & -\mathcal{M}_{i} \\  \mathcal{M}_{i}  & \mathcal{M}_{r} \emat=- \Delta \Omega{{}_{\mm}f}(\beta_{eq},\ell_{\rm min},\ell_{\rm max})\bmat \cos(2 \alpha_{qe}) & -\sin(2\alpha_{qe})\\  \sin(2 \alpha_{qe})  & \cos(2 \alpha_{qe}) \emat \,,
\eeq
%
where all the symbols have the same meaning as discussed in \sec{sec:qu2eb}. Note that the kernel in the above equation differs from the one in \eq{eq:op_qu2eb_rad} by a change in sign on the off-diagonals of the block matrix. When expressed in terms of the same set of Euler angles used to define the operator $\bar{O}$, it can be shown that the different forms of the real space operator are given by the following equations:
%
\beqry
    {}_{+2}X(\hat{n}_q) &&=  \Delta \Omega \sum_{e=1}^{N_{\rm pix}} [E+iB](\hat{n}_{e})\   \mathcal{M}^*_{B}(\hat{n}_e) \hspace{0.8cm}\textrm{\emph{Radiation kernel}},\label{eq:eb2qu_radiation} \\
    {}_{+2}X(\hat{n}_q) &&= \Bigg\lbrace \mathcal{M}^*_{G} \star [E+iB] \Bigg\rbrace(\hat{n}_q) \hspace{1.4cm}\textrm{\emph {Convolution kernel}}, \label{eq:eb2qu_convolution}
\eeqry
%
where all the symbols have the same meaning as defined in \sec{sec:qu2eb}. Note that the conjugated forms of the radiation kernel (Green's function) and the convolution kernel (effective beam) for the operator $\bar{O}$ have their roles reversed for the inverse operator $\bar{O}^{-1}$.
%--------------------------------------------------------

\subsection{Visualizing the real space kernels} \label{sec:visualize_operator}
%
\newlength{\kernelfigwidth}
\setlength{\kernelfigwidth}{0.2\columnwidth}
\newlength{\kernelfigspace}
\setlength{\kernelfigspace}{-1.8mm}


\begin{figure}[t] 
%%%% NEW MG/MB
\begin{center}
\begin{tabular}{m{8ex}m{\kernelfigwidth}m{\kernelfigwidth}|m{\kernelfigwidth}m{\kernelfigwidth}}
$b=90^\circ$ &
\hspace{\kernelfigspace}\includegraphics[width=\kernelfigwidth]{new_kernel/qu2eb_rker_rad_lat90_lon45.pdf} &
\hspace{\kernelfigspace}\includegraphics[width=\kernelfigwidth]{new_kernel/qu2eb_iker_rad_lat90_lon45.pdf} &
\hspace{\kernelfigspace}\includegraphics[width=\kernelfigwidth]{new_kernel/qu2eb_rker_con_lat90_lon45.pdf} &
\hspace{\kernelfigspace}\includegraphics[width=\kernelfigwidth]{new_kernel/qu2eb_iker_con_lat90_lon45.pdf} \\
$b=87^\circ$&
\hspace{\kernelfigspace}\includegraphics[width=\kernelfigwidth]{new_kernel/qu2eb_rker_rad_lat87_lon45.pdf} &
\hspace{\kernelfigspace}\includegraphics[width=\kernelfigwidth]{new_kernel/qu2eb_iker_rad_lat87_lon45.pdf} &
\hspace{\kernelfigspace}\includegraphics[width=\kernelfigwidth]{new_kernel/qu2eb_rker_con_lat87_lon45.pdf} &
\hspace{\kernelfigspace}\includegraphics[width=\kernelfigwidth]{new_kernel/qu2eb_iker_con_lat87_lon45.pdf} \\
$b=80^\circ$&
\hspace{\kernelfigspace}\includegraphics[width=\kernelfigwidth]{new_kernel/qu2eb_rker_rad_lat80_lon30.pdf} &
\hspace{\kernelfigspace}\includegraphics[width=\kernelfigwidth]{new_kernel/qu2eb_iker_rad_lat80_lon30.pdf} &
\hspace{\kernelfigspace}\includegraphics[width=\kernelfigwidth]{new_kernel/qu2eb_rker_con_lat80_lon30.pdf} &
\hspace{\kernelfigspace}\includegraphics[width=\kernelfigwidth]{new_kernel/qu2eb_iker_con_lat80_lon30.pdf} \\
$b=0^\circ$&
\hspace{\kernelfigspace}\includegraphics[width=\kernelfigwidth]{new_kernel/qu2eb_rker_rad_lat0_lon90.pdf} &
\hspace{\kernelfigspace}\includegraphics[width=\kernelfigwidth]{new_kernel/qu2eb_iker_rad_lat0_lon90.pdf} &
\hspace{\kernelfigspace}\includegraphics[width=\kernelfigwidth]{new_kernel/qu2eb_rker_con_lat0_lon90.pdf} &
\hspace{\kernelfigspace}\includegraphics[width=\kernelfigwidth]{new_kernel/qu2eb_iker_con_lat0_lon90.pdf} \\
&
\centering $ \textrm{Re} \left(\mathcal{M}_{G} \right) $ &
\centering $\textrm{Im} \left(\mathcal{M}_{G} \right) $ &
\centering $\textrm{Re}  \left(\mathcal{M}_{B}^* \right) $ &
\centering $\textrm{Im} \left(\mathcal{M}_{B}^* \right) $
\end{tabular}
\end{center}
  \caption{Real and imaginary parts of real space kernels for $Q/U$ to $E/B$ translation (and vice versa).  The function $\mathcal{M}_{G}$ is the Green's function (radiation kernel) that gives $E+iB$ for a delta function Stokes input $_{+2}X = Q+iU$ at the center.  On the other hand, $\mathcal{M}^*_{B}$ is the Green's function (radiation kernel) that gives $_{+2}X$ for a delta function scalar input $E+iB$ at the center.  The black circles denotes the position of the center around which the kernels have been evaluated while the black star marks the location of the North Pole. The four rows depict the kernels at different latitudes on the sphere.   The kernels are unchanged for center points at constant latitude.  $\mathcal{M}_G$ is invariant over the sphere because $E/B$ fields are coordinate independent.  In the convolution kernels, $\mathcal{M}_G$ and $\mathcal{M}_B$ are conjugated and switch roles.  The kernels have been evaluated with the band limit $\ell \in [2,192]$ and sampled at the HEALPix resolution parameter $N_{\rm side}=2048$. Each panel is approximately $16^{\circ} \times 16^{\circ}$ in size with grid lines every 2 degrees. } \label{fig:vis_kernel} 
\end{figure}
%
We compute the Euler angles $(\alpha, \beta, \gamma)$ given the angular coordinates of any two HEALPix pixels and use these to evaluate the convolution and radiation kernels. To provide an intuition for how these kernels vary as a function of position of the central pixel we depict in \fig{fig:vis_kernel} the respective kernels at a few different locations on the sphere.
While the kernels are evaluated in the band limit $\ell \in [2,192]$, for illustration these functions are sampled at a very high Healpix resolution parameter of $N_{\rm side}=2048$. All the plots have been rotated such that the central location marked by the black circle are in the centre of the figure. The grid spacing is 2 degrees.

The kernel $\mathcal{M}_G$ is the Green's function of the operator that transforms Stoke parameters to coordinate independent $E/B$.  The coordinate independence implies that the real and imaginary parts of the kernel do not vary with changes in the galactic latitude and longitude of the central pixel. In particular these functions are not distorted when a part of the domain overlaps with the poles, as can be seen in the first two rows of \fig{fig:vis_kernel}. From \eq{eq:qu2eb_radiation_concise}, we can see the $E/B$ patterns that Kronecker $\delta$-functions in the Stokes parameter pixels produce:
%
\beq
\bmat E= \textrm{Re}(\mathcal{M}_G) \\ B =\textrm{Im}(\mathcal{M}_G)  \emat  \leftarrow \bmat Q=\frac{\delta(\hat{n} - \hat{n}_q)}{\Delta \Omega}\\ U=0 \emat ;\qquad
\bmat E= -\textrm{Im}(\mathcal{M}_G) \\ B = + \textrm{Re}(\mathcal{M}_G)  \emat  \leftarrow \bmat Q=0 \\ U=\frac{\delta(\hat{n} - \hat{n}_q)}{\Delta \Omega} \emat.
\eeq
%
On the other hand, $\mathcal{M}^*_B$ is the Green's function of the operator that transforms $E/B$ to coordinate-dependent Stokes parameters $Q/U$.   The kernel $\mathcal{M}^*_B$ does not change with the central longitude, but varies as a function of galactic latitude. This latitude-dependent shape carries the  coordinate dependence of the Stokes parameters. From \eq{eq:eb2qu_radiation}, we $\mathcal{M}^*_B$ arises from $\delta$-functions in $E/B$:
%
\beq
\bmat Q= \textrm{Re}(\mathcal{M}^*_B) \\ U = \textrm{Im}(\mathcal{M}^*_B)  \emat  \leftarrow \bmat E=\frac{\delta(\hat{n} - \hat{n}_e)}{\Delta \Omega}\\ B=0 \emat;\qquad
\bmat Q= -\textrm{Im}(\mathcal{M}^*_B) \\ U = +\textrm{Re}(\mathcal{M}^*_B)  \emat  \leftarrow \bmat E=0 \\ B=\frac{\delta(\hat{n} - \hat{n}_e)}{\Delta \Omega} \emat\,.
\eeq
%
Recall that the complex conjugates of these functions switch roles to form the convolution kernels (For example, $\mathcal{M}_G^*$ is the convolution kernel for $E/B \rightarrow Q/U$.).  It is also interesting to note that the radiation kernel ($\mm_{G}$) and the convolution kernel ($\mm_{B}$) become progressively identical as one approaches the equator as seen in  \fig{fig:vis_kernel}, which is a consequence of $\gamma \simeq -\alpha$ in the vicinity of the equator. The equator is the place on the sphere most akin to the flat sky case, where the radiation and convolution kernels are expected to be identical.

%--------------------------------------------------------
\subsection{Purifying Stokes parameters $Q$/$U$ for $E$/$B$ modes} \label{sec:purify_stokes_qu}
We can only measure the total Stokes vector, a sum of the part that corresponds to scalar $E$ and the part that corresponds to $B$.  The $E$/$B$ modes are orthogonal to each other in the sense that their respective operators are orthogonal to each other as discussed in \sec{sec:mat_pol_intro}. It is possible to decompose the Stokes vector \vp{} into one \vp{\rm E} that purely contributes to $E$ modes and another \vp{\rm B} that purely contribute to the $B$ modes of polarization. In this section we derive the real space operators which operate on the total Stokes vector and yield this decomposition, without ever having to explicitly evaluate the scalar $E/B$ modes. The algebra is more involved, but the derivation is similar to that discussed in \sec{sec:qu2eb}, so we refrain from presenting the detailed calculations here and outline only the key points. We use the harmonic space projection operators $\tilde O_{E/B}$, defined in \eq{eq:har_eb_op}, to derive the respective real space operators. The Stokes parameters corresponding to each scalar mode are given by the following expressions:
%
\beqry
\bar{P}_E &=&  [\bar T^{-1}  {{}_2\mathcal{Y}} \, \tilde T  \tilde O_E \tilde T^{-1} {{}_2\mathcal{Y}^{\ddagger}}\, \bar T] \bar{P}  \,, \\
&=& [\frac{1}{4} \bar T^{\dagger }  {{}_2\mathcal{Y}} \, \tilde T  \tilde O_E  \tilde T^{\dagger}  {{}_2\mathcal{Y}^{\ddagger}}\, \bar T ]\bar{P}  \,, \nonumber \\
&=&  \bar O_{E} \bar{P} \,,\nonumber \\
\bar{P}_B &=&  [\bar T^{-1}  {{}_2\mathcal{Y}} \, \tilde T  \tilde O_B \tilde T^{-1} {{}_2\mathcal{Y}^{\ddagger}} \bar T]\bar{P}  \,, \\
&=& [\frac{1}{4} \bar T^{\dagger }  {{}_2\mathcal{Y}}\, \tilde T  \tilde O_B \tilde T^{\dagger} {{}_2\mathcal{Y}^{\ddagger}}\, \bar T] \bar{P}   \,, \nonumber\\
&=&  \bar O_{B} \bar{P} \,. \nonumber
\eeqry
%
We contract over all the matrix operators to arrive at the the real space operators. On working through the algebra it can be shown that the real space operators have the following form:
%
\beq
\bar O_{E/B} = 0.5 \Delta \Omega \Bigg\lbrace \bmat \mathcal{I}_{r} & \mathcal{I}_{i} \\  -\mathcal{I}_{i}  & \mathcal{I}_{r} \emat \pm \bmat \mathcal{D}_{r} & \mathcal{D}_{i} \\  \mathcal{D}_{i}  & - \mathcal{D}_{r} \emat \Bigg\rbrace \,,\\
\eeq
where $\mathcal{I}_{r}, \mathcal{D}_{r}$ and $\mathcal{I}_{i}, \mathcal{D}_{i}$ are the real and imaginary parts of the following complex functions:
%
\begin{subequations}
\beqry
\mathcal{I} (\hat{n}_e,\hat{n}_q) &=& \mathcal{I}_{r} + i \mathcal{I}_{i} = \sum_{\ell m} {_{-2}Y}_{\ell m}(\hat n_e) {_{-2}Y}^*_{\ell m}(\hat n_q) \,, \\
\mathcal{D}(\hat{n}_e,\hat{n}_q)  &=& \mathcal{D}_{r} + i\mathcal{D}_{i} = \sum_{\ell m} {_2Y}_{\ell m}(\hat n_e) {_{-2}Y}^*_{\ell m}(\hat n_q) \,.
\eeqry
\end{subequations}
%
These functions can be further simplified using the identity of spin spherical harmonics given in \eq{eq:sum_spin_shf}. Specifically it can be shown that these functions reduce to the following mathematical forms:
%
\beqrys \label{eq:fn_i}
\mathcal{I}(\hat{n}_e, \hat{n}_q) &=& \sum_{\ell} \sqrt{\frac{2\ell+1}{ 4 \pi}}{_{-2}Y}_{\ell2}(\beta_{qe}, \alpha_{qe}) ~ \rm{e}^{i2 \gamma_{qe}} \label{eq:healpix-compatible-i} = \mathcal{I}_r + i \mathcal{I}_i \,, \\
\mathcal{I}_r + i \mathcal{I}_i &=& \Big [ \cos(2 \alpha_{qe} +  2\gamma_{qe}) + i \sin(2 \alpha_{qe} +  2 \gamma_{qe}) \Big]   {{}_{\mi}f}(\beta_{qe},\ell_{\rm min},\ell_{\rm max}) \,,
\eeqrys
%
%
\beqrys \label{eq:fn_d}
\mathcal{D}(\hat{n}_q, \hat{n}_e) &=& \sum_{\ell} \sqrt{\frac{2\ell+1}{ 4 \pi}}{_2Y}_{\ell 2}(\beta_{qe}, \alpha_{qe}) ~ \rm{e}^{- i2 \gamma_{qe}} \label{eq:healpix-compatible-m} =\mathcal{D}_r + i \mathcal{D}_i \,, \\
\mathcal{D}_r + i \mathcal{D}_i &=&  \Big [ \cos(2 \alpha_{qe} - 2\gamma_{qe}) + i \sin(2 \alpha_{qe} -  2 \gamma_{qe}) \Big]   {{}_{\md}f}(\beta_{qe},\ell_{\rm min},\ell_{\rm max}) \,,
\eeqrys
%
where the radial functions are given by:
%
\beq
{{}_{\mdi}f}(\beta,\ell_{\rm min},\ell_{\rm max}) = \sum_{\ell=\ell_{\rm min}}^{\ell_{\rm max}} \sqrt{\frac{2\ell+1}{ 4 \pi}} {{}_{ \mdi}f}_{\ell}(\beta) \label{eq:f2_rad_ker}\,,
\eeq
%
where the functions ${{}_{\mdi}f}_{\ell}(\beta)$ are expressed in terms of $P_{\ell}^2$ Legendre polynomials and are given by the following explicit mathematical forms:
 %
 \beqry
 _{\mdi}f_{\ell}(\beta) &=& 2 \frac{(\ell-2)!}{(\ell+2)!}  \sqrt{\frac{2\ell +1 }{4 \pi}} \Bigg[ - P_{\ell}^{2} (\cos  \beta) \left( \frac{\ell-4}{\sin^2 \beta} + \frac{1}{2}\ell(\ell-1) \pm \frac{2 (\ell-1) \cos \beta}{\sin^2 \beta} \right) \nonumber \\ 
&+& P_{\ell-1}^2 (\cos \beta) \left( (\ell+2) \frac{\cos \beta}{\sin^2 \beta} \pm \frac{2 (\ell+2)}{ \sin^2 \beta } \right) \Bigg] \,. \label{eq:rad_ker_quequbqu}
 \eeqry
 %
Finally the Stokes parameters corresponding to the respective scalar fields can be computed by evaluating the following expressions:
 %
\beqry \label{eq:op_qu2equbqu}
\bmat Q_e \\ U_e  \emat_{E/B} &=& \sum_{q=1}^{N_{\rm pix}} \Bigg\lbrace {{}_{\mi}f}(\beta_{qe},\ell_{\rm min},\ell_{\rm max}) \bmat \cos(2 \alpha_{qe} + 2\gamma_{qe}) & \sin(2\alpha_{qe} +2 \gamma_{qe}) \\  -\sin(2\alpha_{qe} +2 \gamma_{qe})  & \cos(2 \alpha_{qe} + 2 \gamma_{qe}) \emat  \bmat Q_q \\ U_q  \emat  \\ &\pm& {}_{\md}f(\beta_{qe},\ell_{\rm min},\ell_{\rm max}) \bmat \cos(2 \alpha_{qe} - 2\gamma_{qe}) &  \sin(2\alpha_{qe} - 2 \gamma_{qe}) \\  \sin(2\alpha_{qe} - 2 \gamma_{qe})  & - \cos(2 \alpha_{qe} - 2 \gamma_{qe}) \emat  \bmat Q_q \\ U_q  \emat \Bigg\rbrace  \frac{\Delta\Omega}{2}  \,, \nonumber 
\eeqry
%
where all the symbols have their usual meaning. The above expression can be cast in the further simplified form,
%
\begin{subequations}
\beqry
{}_{+2}X_{E/B}(\hat{n}_e) &=& 0.5 \Delta \Omega\sum_{q=1}^{N_{\rm pix}}  {{}_{\mi}f}(\beta_{qe}) e^{-i2 (\alpha_{qe} + \gamma_{qe})} {}_{+2}X(\hat{n}_q)  \pm {{}_{\md}f}(\beta_{qe}) e^{i2 (\alpha_{qe} - \gamma_{qe})} {}_{+2}X(\hat{n}_q)^* \,, \nonumber \\
&=& 0.5 \Bigg\lbrace \Delta \Omega \sum_{q=1}^{N_{\rm pix}}  {}_{+2}X(\hat{n}_q)  \, \mathcal{I}_G(\hat{n}_q) \pm {}_{+2}X(\hat{n}_q)^* \, \mathcal{D}_G(\hat{n}_q) \Bigg\rbrace \hspace{.5cm } \textrm{\emph{  Radiation kernel}} \nonumber \,, \\ \vspace{-1cm} \\
&=& 0.5 \Bigg\lbrace \mathcal{I}_B \star {}_{+2}X\pm \mathcal{D}_B \star {}_{+2}X^* \Bigg\rbrace(\hat{n}_e)   \hspace{.5cm } \textrm{\emph {  Convolution kernel}} \,, \label{eq:qu2equbqu_convolution}
\eeqry
\end{subequations}
%
where all the symbols have their usual meaning and the explicit multipole dependence of the real space operators has been suppressed for brevity. Note that when the operators are expressed in terms of the Euler angles $(\alpha_{qe},\beta_{qe},\gamma_{qe})$ they can be interpreted as the Greens functions and  we denote them by $\mathcal{I}_G=\mathcal{I}^*$ and $\mathcal{D}_G=\mathcal{D}$. When expressed as function of Euler angles $(\alpha_{eq},\beta_{eq},\gamma_{eq})$ corresponding to the inverse rotations they can be interpreted as some convolving beam and we denote them by $\mathcal{I}_B=\mathcal{I}$ and $\mathcal{D}_B=\mathcal{D}$. Note that unlike in the case of the operators $\mm_G$ and $\mm_B$ which have different shapes owing to their dependence on Euler angles $\alpha$ and $\gamma$ respectively, the operators $D_G$ and $D_B$ are identical since $(\alpha_{qe}-\gamma_{qe}) = (\alpha_{eq}-\gamma_{eq})$, while $\mathcal{I}_{G}$ and $\mathcal{I}_B$ are related by conjugation since  $(\alpha_{qe}+\gamma_{qe}) = -(\alpha_{eq}+\gamma_{eq})$.

The operator $\mathcal{I}$ is Hermitian and is a band limited version of the delta function owing to the identity: $\lim_{\ell_{\rm max} \rightarrow \infty} \mathcal{I} = \delta(\hat{n}_i - \hat{n}_j)$. For all practical purposes $\mathcal{I}$ acts like an identity operator as ascertained by the following set of identities: (i) $\mathcal{I} \mathcal{I}=\mathcal{I}$ ; (ii) $\mathcal{D} \mathcal{I}=\mathcal{D}$. $\mathcal{D}$ is a complex but symmetric matrix and $\mathcal{D}^*$ is its inverse in this band limited sense: $\mathcal{D}^* \mathcal{D}=\mathcal{I}$. Using these properties\footnote{While testing the real space operator identities one encounters terms like $\mathcal{D} \mathcal{I}^*,\mathcal{I}^*\mathcal{I}$ and $\mathcal{I}\mathcal{I}^*$ which cannot be simply interpreted but they always occur in pairs with opposite signs that exactly cancel each other.} of the operators $\mathcal{I}$ and $\mathcal{D}$ , one can verify that the real space operators satisfy the following identities:
%
\begin{subequations}
\beqry
\bar O_E \, \bar O_E &=& \bar O_E;\qquad \bar O_B \, \bar O_B = \bar O_B \,, \\
\bar O_E \, \bar O_B &=& 0 \,,\label{eq:real_ortho}\\
\bar O_E + \bar O_B &=& \mathcal{I} \,,
\eeqry
\end{subequations}
%
which are the real space analogues of their harmonic space counterparts discussed in \sec{sec:mat_pol_intro}. Thus they are exactly orthogonal and idempotent. Note that unlike in the harmonic case, the sum of the operators is the band limited identity operator $\mathcal{I}$. This non-exactness is representative of the loss of information resulting from making this transformation on measured data with some imposed band limit. If we were to force the sum of the operators to be exactly an identity matrix, we would compromise the orthogonality property of $\bar{O}_E$ and $\bar{O}_B$, which is exact (\eq{eq:real_ortho}) and a more crucial property of the operators.
%
\begin{figure}[!t]
  \begin{center}
  \begin{tabular}{m{8ex}m{\kernelfigwidth}m{\kernelfigwidth}|m{\kernelfigwidth}m{\kernelfigwidth}}
$b=90^\circ$&
\hspace{\kernelfigspace}\includegraphics[width=\kernelfigwidth]{new_kernel/qu2ebqu_rker_D_lat90_lon45.pdf} &
\hspace{\kernelfigspace}\includegraphics[width=\kernelfigwidth]{new_kernel/qu2ebqu_iker_D_lat90_lon45.pdf} &
\hspace{\kernelfigspace}\includegraphics[width=\kernelfigwidth]{new_kernel/qu2ebqu_rker_I_lat90_lon45.pdf} &
\hspace{\kernelfigspace}\includegraphics[width=\kernelfigwidth]{new_kernel/qu2ebqu_iker_I_lat90_lon45.pdf} \\
$b=87^\circ$&
\hspace{\kernelfigspace}\includegraphics[width=\kernelfigwidth]{new_kernel/qu2ebqu_rker_D_lat87_lon45.pdf} &
\hspace{\kernelfigspace}\includegraphics[width=\kernelfigwidth]{new_kernel/qu2ebqu_iker_D_lat87_lon45.pdf} &
\hspace{\kernelfigspace}\includegraphics[width=\kernelfigwidth]{new_kernel/qu2ebqu_rker_I_lat87_lon45.pdf} &
\hspace{\kernelfigspace}\includegraphics[width=\kernelfigwidth]{new_kernel/qu2ebqu_iker_I_lat87_lon45.pdf} \\
$b=80^\circ$&
\hspace{\kernelfigspace}\includegraphics[width=\kernelfigwidth]{new_kernel/qu2ebqu_rker_D_lat80_lon30.pdf} &
\hspace{\kernelfigspace}\includegraphics[width=\kernelfigwidth]{new_kernel/qu2ebqu_iker_D_lat80_lon30.pdf} &
\hspace{\kernelfigspace}\includegraphics[width=\kernelfigwidth]{new_kernel/qu2ebqu_rker_I_lat80_lon30.pdf} &
\hspace{\kernelfigspace}\includegraphics[width=\kernelfigwidth]{new_kernel/qu2ebqu_iker_I_lat80_lon30.pdf} \\
$b=0^\circ$&
\hspace{\kernelfigspace}\includegraphics[width=\kernelfigwidth]{new_kernel/qu2ebqu_rker_D_lat0_lon90.pdf} &
\hspace{\kernelfigspace}\includegraphics[width=\kernelfigwidth]{new_kernel/qu2ebqu_iker_D_lat0_lon90.pdf} &
\hspace{\kernelfigspace}\includegraphics[width=\kernelfigwidth]{new_kernel/qu2ebqu_rker_I_lat0_lon90.pdf} &
\hspace{\kernelfigspace}\includegraphics[width=\kernelfigwidth]{new_kernel/qu2ebqu_iker_I_lat0_lon90.pdf} \\
&
\centering $\textrm{Re} \left(\mathcal{D} \right)$ &
\centering $\textrm{Im} \left(\mathcal{D} \right)$ &
\centering $\textrm{Re} \left(\mathcal{I} \right)$ &
\centering $\textrm{Im} \left(\mathcal{I} \right)$
  \end{tabular}
  \end{center}
  \caption{Like Figure \ref{fig:vis_kernel}, but for the kernels that purify Stokes parameter into their $E/B$ parts.} \label{fig:vis_kernel_DI}
\end{figure}
%

The kernels $\mathcal{D}$ and $\mathcal{I}$ vary significantly as a function of galactic latitude of the central pixel, as seen in \fig{fig:vis_kernel_DI}. These kernels show a two fold symmetry in the vicinity of the poles.  Here the Euler angle $\gamma \approx 0$ here and therefore $e^{i2(\alpha \pm \gamma)} \approx e^{i2\alpha}$. Note that in this region, the azimuthal profile of the real and imaginary part of these kernels is identical to $-\mathcal{M}_G$.  The imaginary part of the band limited delta function $\mathcal{I}$ contributes just as much as the real part in these regions. On transiting to lower latitudes, however, $\mathcal{D}$ quickly transitions to having a four fold symmetry while $\mathcal{I}$ transitions to being dominated by the real part and behaves more like the conventional delta function. This transition can be most easily understood in the flat sky limit where $\gamma \approx -\alpha$ which leads to the resultant 4 fold symmetry seen for $\mathcal{D}$ owing to $e^{i2(\alpha - \gamma)} \approx e^{i4\alpha}$ and $\mathcal{I}$ being dominated by the real part owing to $e^{-i2(\alpha + \gamma)} \approx 1 + i0$. Since the flat sky approximation has most validity in the proximity of the equator these limiting tendencies of the respective kernels are seen in the bottom row of \fig{fig:vis_kernel_DI} which depict the kernels evaluated at the equator $b=0^{\circ}$. The middle two row depict the kernels evaluated at intermediate latitudes: $b=87^{\circ}$ and $b=80^{\circ}$ and serve to indicate the rate of this transition. As before, these kernels are invariant under changes in longitude of the central pixel with the latitude fixed.
