\section{Introduction}
During recombination, the Cosmic Mircowave Background undergoes Thomson scattering that leaves it with $\sim 5$ percent linear polarization.  The polarization signal contains information about the plasma velocity and provides cosmological constraints independent from the signal in temperature \citep{1997NewA....2..323H}.  The standard analysis technique converts the Stokes parameters of polarization into scalar ($E$) and pseudo-scalar ($B$) modes, which are easier to compare to theory.  The pseudo-scalar $B$ mode is particularly important because it cannot be generated by primordial scalar pertubations.  Thus any $B$-modes that appear must be from:
% Need citations for all...
(1) primordial tensor perturbations (gravitational waves), which reveal the energy scale of inflation
\citep{1997PhRvD..56..596H,1997PhRvL..78.2054S};
(2) lensing of $E$-modes, which helps to reveal the distribution of matter in the universe;
(3) foregrounds (especially Galactic synchrotron and dust emission)
\citep{2016A&A...586A.133P};
(4) unknown systematic problems in the data;
(5) unknown, exotic phenemena like cosmic birefringence or primordial magnetic fields
\citep{1996ApJ...469....1K,1999PhRvL..83.1506L,2004ApJ...616....1C,2014MNRAS.438.2508P}.
 
 
The formalism for converting the Stokes parameters to scalar quantities is well established \citep{1997PhRvD..55.7368K,1997PhRvD..55.1830Z}.  The reason we take a detailed look at it here is that with renewed focus on foreground contamination to the $B$-mode signal, we wanted to gain an intuition for what $E/B$ mode patterns arise from physical polarized structures in the galaxy.  We also wanted to gain insight into the problem of masking and the non-locality of the $E/B$ fields compared to the Stokes parameters.

The spin-0 scalar $E$/$B$ modes relate to the spin-2 complex Stokes parameters via the spin-raising and -lowering operators ($\eth,\bar \eth$), which are second derivatives evaulated locally.  However, in practice we compute $E$/$B$ modes to a specified band limit, and this makes them non-local functions of the polarization field.  In other words, the $E$/$B$ modes at a point can get contributions from all over the sky.

(stuff about foregrounds, masking, ambiguous modes with references)
 
Zaldarriaga explored the spatial real-space kernels in the flat-sky approximation \citep{Zaldarriaga2001a}.

This paper is organized in the following manner: In \sec{sec:pol-primer} we present a primer on the description of CMB polarization on the sphere. Here we begin with a heuristic argument that makes transparent the real space construction of E/B modes on the sphere. We then discuss the standard harmonic space procedures for this operation. Finally we introduce the matrix-vector notation which yields a more concise description of the harmonic space procedures. In \sec{sec:real_space_operators} we derive and discuss the real space operators that transform Q/U to E/B and vice versa. In \sec{sec:visualize_operator} we evaluate these real space operators and present visualizations of these functions. We also derive a real space operator that decomposes the Stokes Q/U  parameters into components that correspond to E and B modes respectively.  Here we also discuss the locality of the real space E \& B operators. In \sec{sec:numerical_implementation} we implement these operators to evaluate E \& B  maps from the Stokes parameters Q \& U and compare these maps and their spectra from those derived using Healpix. We conclude with a discussion and the scope of this new method of analyzing CMB polarization in \sec{sec:discussion}.
