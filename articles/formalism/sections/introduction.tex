\section{Introduction}
During recombination, the Cosmic Mircowave Background undergoes Thomson scattering that leaves it with $\sim 5$ percent linear polarization.  The polarization signal contains information about the plasma velocity and provides cosmological constraints independent from the signal in temperature anisotropies \citep{1997NewA....2..323H}.  The standard analysis technique involves translating the Stokes parameters of polarization into scalar ($E$) and pseudo-scalar ($B$) modes, since the statistics of these coordinate independent scalar fields are predicted by theory.   

The dominant contribution to the $E$-modes of polarization are sourced by primordial scalar perturbations, however these do not generate B-modes of polarization at first order. The $B$-modes of CMB polarization can be generated by various sources:
% Need citations for all...
primordial tensor perturbations (gravitational waves)%, which reveal the energy scale of inflation
\citep{1997PhRvD..56..596H,1997PhRvL..78.2054S};
weak gravitational lensing of $E$-modes resulting from photon deflections induced by the potential of intervening large scale structure;
foregrounds (especially Galactic synchrotron and dust emission)
\citep{2016A&A...586A.133P};
uncorrected systematic problems in the data \citep{2003PhRvD..67d3004H,2008PhRvD..77h3003S};
and unknown, exotic phenemena like cosmic birefringence or primordial magnetic fields
\citep{1996ApJ...469....1K,1999PhRvL..83.1506L,2004ApJ...616....1C,2014MNRAS.438.2508P}.

The pseudo-scalar $B$-modes particularly on large angular scales ($\sim 2$ degrees) are expected to have a significant contribution from primordial tensor perturbations generated during inflation. Hence a measurement of the large angular scale B-modes will yield information about the statistical properties of tensor perturbations which will finally lead to important constraints on models of inflation and the measurement of $B$-modes on small angular scales ($\sim$ few arcminute) will yield information on the clustering of matter across cosmic ages \citep{Abazajian2015, Kamionkowski2016,Abazajian2016,Hu2002c,Wehus2016}.
  
The primary aim of the current CMB experiments is to make precise measurements of the CMB polarizations. While the E-modes of polarization have been measured reasonably well by a number of experiments (\cite{2018RPPh...81d4901S} gives a recent review), the ongoing and future CMB experiments aim to measure the B-modes of CMB polarization with unprecedented accuracy and angular resolution. The detectors are swiftly approaching the desired sensitivities enabling us to in principle measure tiny $B$-mode signals ($r\gtrsim 0.001$) \cite{s4 science book, pico mission study}. However $B$-modes generated by galactic foreground are expected to be a few orders of magnitude higher than $B$-mode amplitude measurable by these detectors.  Precise modelling and subtraction of this large foreground contribution poses a major challenge for a robust unravelling of the minuscule $B$-mode signal. 
 
The formalism for converting the Stokes parameters to scalar quantities is well established \citep{1997PhRvD..55.7368K,1997PhRvD..55.1830Z}. The spin-0 scalar $E$/$B$ modes relate to the spin-2 complex Stokes parameters via the spin-raising and -lowering operators ($\eth^2,\bar \eth^2$), which are derivatives evaluated locally.  However, in practice the $E$/$B$ modes are computed to some specified band limit and this makes them non-local functions of the polarization field.  In other words, band limited $E$/$B$ modes evaluated at a point receive contributions from all over the sky. In this work we aim to gain real space insights into the non-locality of the $E/B$ fields compared to the Stokes parameters. With renewed focus on foreground contamination to the $B$-mode signal we aim to gain an intuition for $E/B$ mode patterns resulting from physical polarized structures in the galaxy. These real space insights may yield new ideas for minimizing foreground contamination which may not be possible using conventional approaches. 

\revisit{Some of the ideas presented in this work bear resemblance to those in Zaldarriaga (2001) \citep{Zaldarriaga2001a}, however the approach adopted here is more mathematically rigorous and consequently leads to updates in interpretation of the results and substantial differences in detail. This has enabled us to develop a real space polarization analysis tool, the details of which are beyond the scope of this article but will be discussed in a subsequent publication.}

This paper is organized in the following manner: In \sec{sec:pol-primer} we present a primer on the description of CMB polarization on the sphere, beginning with a heuristic argument that makes transparent the real space construction of $E$/$B$ modes.  We discuss the standard harmonic-space procedures for this operation. Finally, we introduce the matrix-vector notation which yields a more concise description of the harmonic space procedures. In \sec{sec:qu2eb} and \sec{sec:eb2qu} we derive and discuss the real space operators that transform $Q$/$U$ to $E$/$B$ and vice versa. In \sec{sec:visualize_operator} we evaluate these real space operators and present visualizations of these functions. In \sec{sec:purify_stokes_qu} we derive a real space operator that decomposes the Stokes $Q$/$U$  parameters into components that correspond to $E$ and $B$ modes respectively and present its visualizations.  In \sec{sec:radial_locality} we study the locality of the real space operators and explore its band limit dependence. In \sec{sec:generalized_operators} we present a systematic method of generalizing the real space operators by controlling the non-locality while recovering the standard power spectra and we discuss the connection to the standard spin raising and lower operators. In \sec{sec:pol_filaments} we discuss $E/B$ mode signatures of foreground filaments. In \sec{sec:discussion}, we conclude with a summary and discuss the prospects of this new method of analyzing CMB polarization maps.
