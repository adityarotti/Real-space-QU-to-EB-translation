\section{Discussion}\label{sec:discussion}

\revisit{A similar equation for real space $E$ \& $B$ operators was derived in \cite{Zaldarriaga2001a}, however those results were derived for the flat sky case and did not explicitly derive the radial kernel.} \comment{A discussion on this should be in the conclusions.}

In this article we have cast the standard CMB polarization analysis operations in a vector matrix notation. Using this concise notation we derive the real space operators that translate the Stokes vector \vp{} to  the vector of scalars \vs and vice versa. We explicitly demonstrated that this real space operation can be simply interpreted as a convolution over the complex field $[Q - i U]$ (or $[E + iB]$) with an effective complex beam which is fully expressed in terms of the $Y_{\ell 2}$ spherical harmonic functions. We also use this vector matrix notation to derive real space operators which allow the direct decomposition of the full Stokes vector \vp{} into the vector \vp{E} and \vp{B} that correspond to the respective scalar modes. 

Given the effective beam interpretation of these real space operators we derive the harmonic coefficients of these effective beams at the north galactic pole. Using these harmonic coefficients we provide a prescription for computing the convolution kernels at any position on the sphere using the standard Healpix built in functions. The procedure is equivalent to parallel transporting the beam at the north pole to any desired location on the sphere. We implement the prescription to compute the  kernel at different location on the sphere and provide simple explanations in terms of Euler angles for the observed variations.

These real space convolution kernels provide a spatially intuitive way of understanding the construction of the scalar modes. We explicitly show that the kernels separates into an band limit independent azimuthal operation around any given direction which is primarily responsible for requisite decomposition, while the band limit dependent radial weights can be interpreted as some isotropic smoothing operation. These radial weights primarily determine the non-local dependence of the construction of the respective fields at any location on the Stoke field. We define the parameter $\beta_0$ as a means to characterize the non-locality and show that $\beta_0$ scales $\propto \ell_{\rm max}^{-1}$. We show that this non-locality parameter also characterized the non-locality of the $\bar{O}_{E/B}$ operators. 

Finally we present the generalized real space operators $\bar{O}'$, which are derived by allowing the radial function to vary from its default form. We derive constraints on the modifications to these radial function by demanding the inverse operator to be well defined. We argue that these modifications to the radial kernel can be be interpreted as a some smoothing smoothing operation on the scalar fields with a circularly symmetric instrument beam. We also show that as long as these radial function are invertible, the standard spectra can always be recovered from these modified $E'$ \& $B'$ maps. The main advantage of modifying these radial function is the ability to generate more locally defined $E$ and $B$ mode maps. This could potentially be useful in reducing foreground contamination on large angular scales in a full sky $E/B$ analysis. Also defining more locally constructed scalar fields $E$ \& $B$ can be used to circumvent the power leakage nuisance. We explore and demonstrate the working of these ideas in the next paper in this series. 
%\revisit{The discussion till now gives the impression that using the localized convolution kernels is no different from from using the default kernel and altering the spherical harmonic coefficients of expansion of the relavant fields by appropriately operating on them with the  effective beam functions $g_{\ell}$. To appreciate the difference between these two, it is important to realize that in general one can make a choice of a radial function which may not have a band limited description. In such a case these two method of evaluating the relevant fields is not identical. An example of this claim is depicted in \fig{fig:example_gbeta}.\\
%Another important thing to realize is that the harmonic coefficients derived from default full sky operations get some contributions from different portions of sky. For instance evaluating the E and B fields in the vicinity of the poles is are prone to receiving significant contributions from strong foregrounds near the equator. Correcting the harmonic coefficients of expansion with the effective beam function does not cancel these non-local contribution. On the contrary by performing the convolution with the localized real space kernels, the regions which contribute to the local field evaluations are predetermined by the choice of the radial function.} 