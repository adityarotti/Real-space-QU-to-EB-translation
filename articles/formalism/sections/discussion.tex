\section{Discussion}\label{sec:discussion}

We introduced a vector-matrix notation which allows for concise book keeping of all the standard operations involved in the analysis of CMB polarization maps. This notation simplifies the derivation of the real space operators. We presented a first derivation of the real space operators on the sphere that transforms the Stokes vector \vp{} to into a vector of scalars \vs{} and vice versa. We also presented real space operators that directly decompose the full Stokes vector \vp{} into vector \vp{E} and \vp{B} that correspond to the respective scalar modes.

These real space operators provide a spatially intuitive way of understanding the different decompositions of the Stokes vector on the sphere. We explicitly  demonstrated that all the real space operators have the common characteristic  of being separable into a azimuthal part that is band limit independent and radial part that incorporates all the band limit dependence. The azimuthal part of the operator is primarily responsible for the requisite spinorial decomposition, while the radial weights determine the non-local dependence of the construction of the resultant fields on the original fields. Using the self similarity property of the radial functions we define a non-locality parameter and show that it is proportional to $\ell_{\rm max}^{-1}$. The scalar decomposition of the Stokes parameters being of primary interest,  we empirically derived the non-locality parameter $\beta_0$ for this operation and showed that the proportionality constant $\ell_{0}=22$ provided a good prediction for the angular distance at which the ${}_{\mm}f$ falls below 1\% of its maxima as seen in \fig{fig:rad_ker_decay}.

Our careful study of the real space operators reveals the dual interpretation of the operators as either the Green's function or a complex convolving beam depending on whether the kernels are expressed in terms of the forward or inverse rotation Euler angles. While the convolution interpretation is a familiar one, the Green's function interpretation is a new way of looking at this operation. This new interpretation allows us to think of ${}_{+2}X$ as some spin-2 charge which radiates out a complex spin-0 scalar field $E+iB$. The resultant complex scalar maps can be then understood as arising from superposition of the radiating field emanating from all the spin charges on the sphere. The $E/B$ mode maps are merely the real and imaginary parts of this resultant complex scalar field.

\revisit{A similar equation for real space $E$ \& $B$ operators was derived in \cite{Zaldarriaga2001a}, however those results were derived for the flat sky case and did not explicitly derive the radial kernel.} \comment{A discussion on this should be in the conclusions.}

Understanding the non-locality of the real space operators, allowed us to generalize the real space operators $\bar{O}'$ and its inverse ${\bar O}'^{-1}$, which transform between the spin-2 and spin-0 representations of the CMB polarization. We presented a method of systematically generalizing these operators by introducing the harmonic space operator $\tilde{\mathcal{G}}$ which in effect causes the radial function to vary from its default form. We show that these modifications to the radial kernel can be interpreted as a smoothing operation on the scalar fields with a circularly symmetric instrument beam. We argue that one can choose any arbitrary form for $\tilde{\mathcal{G}}$ as long as one adheres to the constraint that $\tilde{\mathcal{G}}^{-1}$ and $\tilde{\mathcal{G}}^{-2}$ are well defined operators.  These constraints are necessary to ensure that all the standard CMB spectra are recoverable from the modified $E'$/$B'$ maps emerging from the modified operators on the Stokes parameter maps. Noting that that the standard spin raising $\eth^2$ and spin lowering $\bar{\eth}^2$ operators are special cases of these generalized operators, we presented a band limited representation of these operators. 

\revisit{Discussion on results from polarized filaments filaments, compare and contrast with those in \cite{Zaldarriaga2001a}.}

There are several advantages and alternative analysis routes that emerge from employing these real space operators for analysis of CMB polarization maps. To begin with, one does not need to evaluate the spin harmonic functions as the real space operators only rely on computing the Euler angles which can be done on the fly and the $P_{\ell}^{2}$ functions need to be tabulated only once for analysis at some predetermined resolution. Note that this real space analysis method trivially generalizes to analyzing maps of arbitrary spin. Especially given the Green's function interpretation of these operators, their implementation is trivially parallelizable, since the $E/B$ contribution from the Stokes charge in each pixel can be evaluated independently. Also note that the evaluation of contribution from charges in the masked portion of the sky can be easily omitted, unlike in the harmonic analysis. Given the Green's function interpretation of the  real space operators and their locality, suggests the possibility of converting the Stokes parameter time ordered data to scalar $E/B$ time ordered data given the pointing model for the observations, which can then be subsequently converted to maps of $E/B$ modes using standard map making techniques. Since the real space operators offer the ability of tuning the locality of $E/B$ maps, this can be potentially exploited to reduce foreground contamination from distant parts of the sky. For instance the real space operators can be defined such that the locality of their radial kernels is varied on different portions of the sky, dictated by say the foreground morphology, resulting in some scalar $E'/B'$ maps.  While this idea seems promising, the usefulness of this implementation will depend on whether the standard $E/B$ mode spectra are recoverable from scalar maps derived in this fashion. Note that it would be fairly non-trivial to think of such implementations using conventional harmonic space methods.  

Results demonstrating the equivalence of these real space methods to the conventional harmonic space methods and from explorations of some of the ideas discussed above will be presented in the second part of this series of papers. 

%\revisit{The discussion till now gives the impression that using the localized convolution kernels is no different from from using the default kernel and altering the spherical harmonic coefficients of expansion of the relavant fields by appropriately operating on them with the  effective beam functions $g_{\ell}$. To appreciate the difference between these two, it is important to realize that in general one can make a choice of a radial function which may not have a band limited description. In such a case these two method of evaluating the relevant fields is not identical. An example of this claim is depicted in \fig{fig:example_gbeta}.\\
%Another important thing to realize is that the harmonic coefficients derived from default full sky operations get some contributions from different portions of sky. For instance evaluating the E and B fields in the vicinity of the poles is are prone to receiving significant contributions from strong foregrounds near the equator. Correcting the harmonic coefficients of expansion with the effective beam function does not cancel these non-local contribution. On the contrary by performing the convolution with the localized real space kernels, the regions which contribute to the local field evaluations are predetermined by the choice of the radial function.}
 
 

 
