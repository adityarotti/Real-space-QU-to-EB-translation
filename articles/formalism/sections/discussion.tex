\section{Discussion}\label{sec:discussion}
In this work, we presented a first derivation of the real space operators on the sphere that transform the Stokes polarization vector to into a vector of scalars and vice versa.  We also presented real space operators that directly decompose the full Stokes vector \vp{} into vectors \vp{E} and \vp{B} that correspond to the respective scalar modes.  To facilitate these derivations we introduced a vector-matrix notation which allows for concise book keeping of all the standard operations involved in the analysis of  polarization maps (or any spin-2 fields) on the sphere. This real-space analysis method trivially generalizes to maps of arbitrary spin.

These real space operators offer a spatially intuitive way of understanding the different decompositions of the Stokes vector on the sphere. We explicitly  demonstrated that all the real space operators are separable into azimuthal and radial parts. While the azimuthal part of the operators is primarily responsible for handling the spin decomposition, the radial weights determine the non-local dependence of the resulting fields on the original fields.  Only the radial part of the kernel depends on the band limit. 
The radial parts of the operator kernels are roughly self-similar in the sense that the radial kernels evaluated with some band limit are related to other radial kernels (evaluated with a different band limit) by an approximate rescaling of the function. We use this property to define a non-locality parameter $\beta_0$ as the angular distance at which the amplitude of the radial kernel falls below one percent of its maxima and it  is a function of the maximum multipole $\ell_{\rm max}$ available for the analysis. We empirically show that the non-locality parameter is approximately given by: $\beta_0 = \mathrm{min}(180^{\circ}, 180^{\circ} \frac{\ell_{0}}{\ell_{\rm max}})$ with $\ell_{0}=22$ for the operator that converts Stokes Q/U to scalars E/B and vice versa (see \fig{fig:rad_ker_decay}). An analysis in  \cite{Zaldarriaga2001a} treated real space $E/B$ operators in the flat sky.  It did not explicitly derive the radial part of the kernel, but argued on geometric grounds that it should fall with angular separation as $\beta^{-2}$.  We find that this agrees with the average behavior of radial functions after averaging oscillations and note that some such averaging always takes place in practice due to pixelization of the signal.  However, for precision reconstruction of $E/B$ on the sphere, the oscillations must be taken into account.

Our careful study of the real space operators show that they can be expressed either as Green's functions or as convolving beam functions.  The convolution interpretation is not a totally new concept.  It guides the discussion in \cite{Zaldarriaga2001a,Chiueh2002} and closely relates to the popular spokes and pinwheel descriptions of the $E/B$ modes. However, the radiation/Green's function interpretation of the operators is a new one and is discussed here for the first time in detail on the sphere. These two different interpretations of the operators emerge from the expression of the kernels in terms of the forward or inverse rotation Euler angles.  The mathematical forms of the Green's function and convolution kernels swap roles (and are conjugated) when transforming back to $Q/U$ from $E/B$.

The Green's function interpretation provides some useful insights into these operations. In particular it allows us to think of ${}_{+2}X=Q+iU$ as some spin-2 charge which radiates out a complex spin-0 scalar field $E+iB$. The resulting complex scalar maps can be then understood as arising from superposition of the radiated spin-0 scalar fields emanating from all the spin charges on the sphere.  The $E/B$ mode maps are merely the real and imaginary parts of this field. Results that demonstrate the equivalence of these real space methods to the conventional harmonic space methods will be presented in the next paper in this series.

Deeper understanding of the non-locality of the real space operators has allowed us to generalize the real space operators that transform between the spin-2 and spin-0 representations of the CMB polarization. We presented a systematic procedure to modify and generalize the construction of the scalar polarization fields and to control the radial kernels, specifying them with few restrictions.  We argue that these modifications to the radial kernel have the same effect as a smoothing operation on the $E/B$ fields by a circularly symmetric beam.  Therefore it is trivial to recover the standard CMB angular power spectra from the modified scalar polarization maps resulting from the modified kernels.  We noted that the standard spin-raising ($\eth^2$) and spin-lowering ($\bar{\eth}^2$) operators are special cases of these generalized operators which allowed us to present a band limited representation of these operators. 

Modified real-space operators with compact kernels could open several alternative analysis routes in the future.  No spin-harmonic transforms are necessary as the real space operators only rely on computing the Euler angles which can be done on the fly.  The radial functions (depending on $P_{\ell}^{2}$) need be tabulated only once at some determined resolution.  Especially given the Green's function interpretation of these operators, their implementation is trivially parallelizable over a compact domain, since the $E/B$ contribution from the Stokes charges in each pixel can be evaluated independently.  Alternatively, the spatially-varying convolution kernel could be applied a polarized effective beam, as implemented in \cite{2011ApJS..193....5M} in a parallel scheme.

The real space kernels could in principle be incorporated into the pointing matrices for map making, allowing maps of $E/B$ to be made directly from instrument data, without the need for Stokes parameter maps as intermediate products.
  The pointing matrix---projecting the maps into the time-ordered data at a point---requires the convolution version of the kernel, while the transpose pointing matrix---projecting the time-ordered data to the map---requires the Green's function kernel.
  The method would be similar to pixel-based strategies to deconvolve an instrument beam during map making \citep[e.g.][]{2010ApJS..187..212C}. Such a strategy for map making would result in the pointing matrix in being much less sparse, hence making a practical implementation significantly challenging. Using the compact radial kernels might help with restoring some of the sparseness of the pointing matrix, however a real world implementation of this method requires a more careful feasibility study.
  %However, such a strategy for map making would force the pointing matrix to be much less sparse and it may not be practical.
 
Since the real space operators let us tune the locality of $E/B$ maps, this can be potentially exploited to eliminate foreground contamination from distant parts of the sky. Such applications can be difficult to implement using conventional harmonic space methods.  For instance, the real space operators can be defined such that the locality of their radial kernels is varied on different portions of the sky, dictated by say the foreground morphology, resulting in some modified scalar $E'/B'$ maps.  While this idea seems interesting, the usefulness of this implementation will depend on whether the standard $E/B$ mode spectra are easily recoverable in this fashion.  We will explore some of these possible analysis directions in the future papers of this series.

Finally, the toolbox of real-space operators gives us more intuition about the $E$- and $B$-mode structure of polarized gas and dust filaments in the Milky Way, an important foreground for inflationary science. We demonstrate that filaments with finite length or changing radius of curvature result in $B$-mode patterns in addition to the $E$-mode pattern already expected.  We therefore predict a characteristic $B$-mode pattern from filament sources that should be observable in future polarization measurements.
