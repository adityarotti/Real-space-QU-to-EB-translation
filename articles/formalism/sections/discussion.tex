\section{Discussion}\label{sec:discussion}

\begin{itemize}
\item Summarize what has been achieved in this peice of work.
\item Whats new as compared to work done by Zaldariaga?
\item Not having to compute spin spherical harmonics.
\item Potentially allows for TOD to E/B map generation.
\item Compact kernels and understanding of why mask apodization works.
\item Can you deconvolve anisotropic compact kernels? Meaning different radial cut-off in different portions of the sky. This will be useful to dealing with foregrounds. Its not obvious if this can be achieved in harmonic space.  
\item The relation to galaxy shear E/B estimates.
\item Understanding E/B signatures of magentized filaments.
\end{itemize}

\revisit{A similar equation for real space $E$ \& $B$ operators was derived in \cite{Zaldarriaga2001a}, however those results were derived for the flat sky case and did not explicitly derive the radial kernel.} \comment{A discussion on this should be in the conclusions.}

In this work we present real space operators on the sphere that transform maps of Stoke Q/U parameters to scalars E/B. We  also present real space operators that decompose maps of Stokes Q/U parameters into the parts that only contribute to E/B modes. We introduced a vector-matrix notation which allows for concise book keeping of all the operations involved and simplifies the derivation of the real space operators. 

These real space operators provide a spatially intuitive way of understanding construction of the scalar modes. We explicitly  demonstrated that all the operators can be separated into a band limit independent azimuthal operation and band limit dependent radial weights. The azimuthal part of the operator is primarily responsible for the requisite spinorial decomposition, while the radial weights determine the non-local dependence of the construction of the resultant fields. We define the $\beta_0$ parameter using an empirical relation to characterize the non-locality and demonstrate that it scales $\propto \ell_{\rm max}^{-1}$.

Our careful study of the real space operators reveals the dual interpretation of the operators as either the Greens function or a convolving beam depending on whether the kernels are expressed in terms of the forward rotation Euler angles or the inverse rotation Euler angles. While the convolution interpretation is familiar to all, the Green's function interpretation is a new one arising from this work. In particular the Green's function interpretation allows us to think of ${}_{+2}X$ as some spin-2 charge which radiates out a complex spin-0 scalar field $E+iB$. The resultant E/B maps can be then understood as arising from superposition of the radiating field from all the spin charges on the sphere.

 We explicitly demonstrated that this real space operation can be simply interpreted as a convolution over the complex field $[Q - i U]$ (or $[E + iB]$) with an effective complex beam which is fully expressed in terms of the $Y_{\ell 2}$ spherical harmonic functions. We also use this vector matrix notation to derive real space operators which allow the direct decomposition of the full Stokes vector \vp{} into the vector \vp{E} and \vp{B} that correspond to the respective scalar modes. 



Finally we present the generalized real space operators $\bar{O}'$, which are derived by allowing the radial function to vary from its default form. We derive constraints on the modifications to these radial function by demanding the inverse operator to be well defined. We argue that these modifications to the radial kernel can be be interpreted as a some smoothing smoothing operation on the scalar fields with a circularly symmetric instrument beam. We also show that as long as these radial function are invertible, the standard spectra can always be recovered from these modified $E'$ \& $B'$ maps. The main advantage of modifying these radial function is the ability to generate more locally defined $E$ and $B$ mode maps. This could potentially be useful in reducing foreground contamination on large angular scales in a full sky $E/B$ analysis. Also defining more locally constructed scalar fields $E$ \& $B$ can be used to circumvent the power leakage nuisance. We explore and demonstrate the working of these ideas in the next paper in this series. 
%\revisit{The discussion till now gives the impression that using the localized convolution kernels is no different from from using the default kernel and altering the spherical harmonic coefficients of expansion of the relavant fields by appropriately operating on them with the  effective beam functions $g_{\ell}$. To appreciate the difference between these two, it is important to realize that in general one can make a choice of a radial function which may not have a band limited description. In such a case these two method of evaluating the relevant fields is not identical. An example of this claim is depicted in \fig{fig:example_gbeta}.\\
%Another important thing to realize is that the harmonic coefficients derived from default full sky operations get some contributions from different portions of sky. For instance evaluating the E and B fields in the vicinity of the poles is are prone to receiving significant contributions from strong foregrounds near the equator. Correcting the harmonic coefficients of expansion with the effective beam function does not cancel these non-local contribution. On the contrary by performing the convolution with the localized real space kernels, the regions which contribute to the local field evaluations are predetermined by the choice of the radial function.}
 
 
\section{Intuition for the polarization structure of magnetized filaments}
\begin{figure}
  \includegraphics[width=0.5\columnwidth]{line.pdf}
  \includegraphics[width=0.5\columnwidth]{spiral.pdf}
  \caption{
    The polarization signals of toy filament structures.
    In a filament organized perfectly along a magnetic field line, the polarization will be perpendicular to the filament direction.  The $E/B$ modes of filaments are in some ways easier to think about than the Stokes parameters.
    Left panels: in a straight filament, the E-mode is positive along the filament and at the ends, but negative along the sides.  B-modes are only non-zero at the ends.  Right panels: in a curved filament, the E-mode is again positive along the filament.  Outside the filament, the $E$-mode is more negative on the interior of the curve than the exterior.  The $B$-modes are again non-zero only at the ends, and are akin to the straight filament case.
    In all images, the longitude increases to the left (sky convention).}
  \label{fig:polfilaments}
\end{figure}

Naturally describes the reason for a positive T/E correlation.  Figure~\ref{fig:polfilaments}.

 
