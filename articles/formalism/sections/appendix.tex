\section{Appendix}
%--------------------------------------------------------
%--------------------------------------------------------
\subsection{Product of spin spherical harmonics}\label{sec:ylm_prod}
The spin spherical harmonics are related to the Wigner D functions via the following relations,
%
\beq
D_{-s m}^{\ell}(\alpha,\beta,\gamma) = \sqrt{\frac{4 \pi}{2 \ell+1}} {}_{s}Y_{\ell m }(\beta,\alpha)e^{-is\gamma} \,,
\eeq
%
where $\alpha, \beta~\&~\gamma$ can be thought of as Euler angles for some rotation.

The product of two different spherical harmonic functions can be expressed in terms of the Wigner D functions and simplified using their identities. In particular we are interested in products of spherical harmonic function of the following kind,
%
\begin{subequations}
\beqry
\sum_{m} {}_{s_1}Y_{\ell m }(\theta_e,\phi_e) {}_{s_2}Y^*_{\ell m }(\theta_q,\phi_q) &=& \frac{2\ell+1}{4 \pi}  \sum_{m}D^{\ell}_{-s_1 m }(\phi_e,\theta_e,0) D^{*\ell}_{-s_2 m }(\phi_q,\theta_q,0) \,,\\
&=& \frac{2\ell+1}{4 \pi}  \sum_{m}D^{\ell}_{-s_1 m }(\phi_e,\theta_e,0) D^{\ell}_{m -s_2 }(0,-\theta_q,-\phi_q) \,, \\
&=& \frac{2\ell+1}{4 \pi} D^{\ell}_{-s_1 -s_2 }(\alpha_{qe},\beta_{qe},\gamma_{qe}) \,, \\
&=& \sqrt{\frac{2\ell+1}{4 \pi}} {}_{s_1}Y_{\ell -s_2}(\beta_{qe},\alpha_{qe}) e^{- i s_1 \gamma_{qe}}
\eeqry
\end{subequations}
%
where we have used some standard identities of the Wigner D functions to transition between then  equations \cite{varshalovich}. Note that the Euler angles $(\alpha,\beta,\gamma) = (0,-\theta_q,-\phi_q)$ correspond to a rotation that aligns the local cartesian coordinate at $\hat{n}_q$ with that at the pole and the Euler angles $(\alpha,\beta,\gamma) = (\phi_e,\theta_e,0)$ correspond to rotations that align the local cartesian coordinates at the pole with those at the location $\hat{n}_e$. Hence the net rotation operation is that of aligning the local cartesian coordinates at location $\hat{n}_q$ with those at location $\hat{n}_e$ and therefore the final results are expressed in terms of Euler angles: $(\alpha_{qe},\beta_{qe},\gamma_{qe})$.

Since the following equation holds true,
\beq
\sum_{m} {}_{s_1}Y_{\ell m }(\theta_e,\phi_e) {}_{s_2}Y^*_{\ell m }(\theta_q,\phi_q) = \sum_{m} {}_{-s_1}Y^*_{\ell m }(\theta_e,\phi_e) {}_{-s_2}Y_{\ell m }(\theta_q,\phi_q) \,,
\eeq
this sum over product of spin spherical harmonic functions can be equally expressed in terms of the Euler angles corresponding to the inverse rotations. Using the same algebra as given above, it is possible to show that 


%%--------------------------------------------------------
%%--------------------------------------------------------

%%--------------------------------------------------------
%%--------------------------------------------------------
%\subsection{ Mathematical properties of spin spherical harmonics}\label{sec:ylm_mathprop}
%The sum over $m$ index of product of two spherical harmonic functions of spin $s_1$ and $s_2$ respectively, is given by the following expression \cite{varshalovich},
%\beq \label{eq:sum_spin_shf}
% \sum_{m}{_{s_1}Y^*_{\ell m}}(\hat{n}_i){_{s_2}Y_{\ell m}}(\hat{n}_j) = \sqrt{\frac{2\ell+1}{4 \pi}} _{s_2}Y^{-s_1}_{\ell}(\beta,\alpha) e^{- i s_2 \gamma} \,,
%\eeq
%where $\alpha, ~\beta ~\&~ \gamma$ correspond to the Euler angles that specify the rotation matrix which transforms the local cartesian coordinates defined at $\hat{n}_i$ such that it aligns with the local cartesian coordinate system at $\hat{n}_j$.
%
%The spin spherical harmonics satisfy the following orthogonality relations,
%%
%\beq
%\int  {_sY_{\ell m}}(\hat{n}){_sY^*_{\ell' m'}}(\hat{n}) d\Omega = \delta_{\ell \ell'} \delta_{\rm m m'} \,, \label{eq:ylmortho1}
%\eeq
%%
%where $s$ denotes the spin of the spherical harmonic coefficients. The numerical validity of \eq{eq:ylmortho1} is only limited by the rate at which these functions are sampled on the sphere and hence this identity can be made arbitrarily accurate by choosing a sufficiently high sampling rate.
%%While working with CMB polarization one is often dealing with spin-2 spherical harmonics. Here we derive some relation which we use while evaluating the real space projection operators,
%%\beq
%%\sum_{\ell m} {_2Y}_{\ell m}(\hat n_i) {_2Y}^*_{\ell m}(\hat n_j) = 
%%\eeq
%
%The spin spherical harmonic functions satisfy the following completeness relation,
%%
%\beq
%\sum_{\ell m}{_sY_{\ell m}}(\hat{n}_i){_sY^*_{\ell m}}(\hat{n}_j) = \delta(\hat{n}_i - \hat{n}_j) \label{eq:ylm_prop1} \,,
%\eeq
%%
%Note that the numerical validity of \eq{eq:ylmortho2} is strictly true only when the sums over the indices $(\ell, m)$ run to infinity. This is never true in practice, since the measured data invariable are band limited owing to the finite resolution of the experiments. Hence this relation is only approximately true and in more realistic scenario takes up the following function form,
%%
%\beqry
%\sum_{\ell=\ell_{\rm min}, m}^{\ell_{\rm max}}{_{s_1}Y^*_{\ell m}}(\hat{n}_i){_{s_2}Y_{\ell m}}(\hat{n}_j) &\approx& \delta(\hat{n}_i - \hat{n}_j) \label{eq:ylmortho2} \,, \\
%\sum_{\ell=\ell_{\rm min}, m}^{\ell_{\rm max}}{_{s_1}Y^*_{\ell m}}(\hat{n}_i){_{s_2}Y_{\ell m}}(\hat{n}_j) &=& \sum_{\ell=\ell_{\rm min}}^{\ell_{\rm max}} \sqrt{\frac{2 \ell+1}{4 \pi}} {}_{s_2}Y^{-s_1}_{\ell}(\beta,\alpha) e^{-i s_2 \gamma} \,, \nonumber
%\eeqry
%%
%where $\alpha, \beta ~\&~ \gamma$ are the Euler angles. A specific case of this function with $(s=2, \ell_{\rm min}=2, \ell_{\rm max}=96)$ is depicted in the last two columns of \fig{fig:vis_kernel}.
%%--------------------------------------------------------
%%--------------------------------------------------------
%
%%--------------------------------------------------------
%%--------------------------------------------------------
%\subsection{The flat sky limit}\label{sec:flat_sky}
%%--------------------------------------------------------
%%--------------------------------------------------------