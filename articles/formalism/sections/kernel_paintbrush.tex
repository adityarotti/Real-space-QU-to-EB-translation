\section{Understanding polarization signatures of magnetized filaments}
% 
\begin{figure}[t]
\includegraphics[width=0.5\columnwidth]{line.pdf}
\includegraphics[width=0.5\columnwidth]{spiral.pdf}
\includegraphics[width=1.0\columnwidth]{colorbar.pdf}
\caption{ The polarization signals of toy filament structures. In a filament organized perfectly along a magnetic field line, the polarization will be perpendicular to the filament direction.  The $E/B$ modes of filaments are in some ways easier to think about than the Stokes parameters. Left panels: in a straight filament, the E-mode is positive along the filament and at the ends, but negative along the sides.  B-modes are only non-zero at the ends.  Right panels: in a curved filament, the E-mode is again positive along the filament.  Outside the filament, the $E$-mode is more negative on the interior of the curve than the exterior.  The $B$-modes are again non-zero only at the ends, and are akin to the straight filament case. In all images, the longitude angle increases to the left (East in sky convention).  All plot are on a common, arbitrary color scale.}
\label{fig:polfilaments}
\end{figure}
%

The real space kernels give us a better intuitive understanding of the $E/B$ modes associated with physical objects.  For example, a simple model for a magnetize filament has the magnetic field threaded along a linear gas overdensity.  Precession of the dust grains around the magnetic field leads to a net polarization perpendicular to the magnetic field (and perpendicular to the filament overall).  For a filament aligned North--South, the polarization will be horizontal or $Q<0$, $U=0$ (left pane of \fig{fig:polfilaments}).  The Green's function kernels for horizontal polarization are rotated by 90 degrees relative to the components of $\cal{M}_G$ in \fig{fig:vis_kernel}.

The kernel can be thought of as the orientable nib of a calligraphy pen or paintbrush that we can trace along the filament.  The positive components for the $E$ part of the Green's function align and reinforce along the filament, and so the filament is highlighted as a segment with $E>0$.  Since the overdensity will also have emission in total intensity, this naturally predicts a positive $TE$ correlation for magnetized filaments.  The $E$ pattern is somewhat negative along the outside of the filament, also a consequence of the kernel shape.

The $B$ part of the Green's function, traced along the filament, cancels itself except at the filament ends.  This results in a non-zero $B$ pattern for the filament.  For a North--South filament, the $B$-mode pattern is positive on the North-East and South-West, and negative in the North-West and South-East.


The non-zero $B$ result is somewhat surprising given that the polarization pattern is symmetric to both horizontal and vertical reflections through the filament center.  However, unlike a circular ring, this filament is not a configuration with a definite parity.  Because the scalar polarization descriptions are coordinate independent, the $E/B$ patterns do not depend on the orientation of the filament.  A filament inclined at $45^\circ$ will have a similar $E/B$ pattern, but different reflection symmetries.  

%\revisit{A stacking analysis of Planck data \citep{2016A&A...586A.141P} sees $E>0$ along filaments (selected from intensity data), but no $B$-mode signal is above the noise.  We predict that it should be there in higher fidelity data.}
\revisit{A stacking analysis of Planck data \citep{2016A&A...586A.141P} sees $E>0$ along filaments (selected from intensity data), but claim no $B$-mode signal. We predict that a B-mode signal from filaments should be present. A naive filament stacking analysis is only suited for making E-mode detections. Detecting B-modes from filaments requires a more careful edge stacked filaments analysis. While the detectability of B-mode signature from filaments from Planck data calls for a more careful assessment, it should be detectable in higher fidelity data.}

The intuition from the real-space kernels holds also when we distort the shape of the filament.  If the filament were bent around into a circle, the positive and negative parts of the $B$ pattern will cancel, and we are left with a hoop of pure $E$ pattern (note that this cancellation would not happen for an ellipse or any loop of non-contant radius of curvature).  The same general description holds for a spiral-shaped filament, which can can be viewed as distortion of the straight filament.  The filament is highlighted by positive $E>0$.  The $E$-pattern is more negative on the interior of a curve than on the exterior, and the concentric rings of filamentary structure make an increasingly negative $E$ value inside.  The $B$-pattern is again concentrated at the ends of the filament in an oriented pair of positive/negative fluctuations. \revisit{The spiral filament showcased in \fig{fig:polfilaments} has a constant radius of curvature and thats the reason for not seeing a B-mode signal in intermediate regions of the spiral. For filaments that don't have a constant radius of curvature one expects to generate more B-mode power.} \comment{This is an important point right? So even if the magnetic field is perfectly aligned with the filament, one can generate B-mode signal in intermediate regions of the filament just by having the filament with changing radius of curvature. There is no reason for filaments to have a constant radius of curvature. Maybe its important to make this demonstration?}

\revisit{Note that we have made the simplifying assumption that the polarization headless vectors are perpendicular to the magnetic field which is always assumed parallel to the tangent to the filament at any given point and this results in generating dominantly an E-mode signal while the B-mode signal is present only near the edges. By similar arguments it can be shown that that polarization pattern will be dominated by B-modes if the polarization vector is aligned at $+45$ (or $-45$) degrees to the tangent to the filament while the E-mode signal would only be present at the edges. Therefore a careful study of the $E/B$ mode power in filaments can provide insights into the orientation of the magnetic field with respect to the length of the filament.}

\revisit{Our conclusions differ subtly from those presented in \cite{Zaldarriaga2001a}, where in it is argued that that filaments cannot produce B-mode signals for configurations in which the polarization vector is parallel or perpendicular to the length of the filament. While this conclusion is true for filaments of infinite length, in reality we expect filaments that are observed on the sky to have a finite length and in this case one expects a definite B-mode pattern from the edges of the filament. One also does not expect that filaments have a constant radius of curvature and this will result in an additional generation of B-mode power.} \comment{Move to conclusions?}
 

