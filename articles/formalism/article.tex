\documentclass[a4paper,11pt]{article}
\usepackage{jcappub}

\usepackage[latin1]{inputenc}
\usepackage{amsmath, amsfonts, amssymb, hyperref, graphicx, multirow, subfigure, bbold}
\hypersetup{colorlinks=True,citecolor=blue}
\usepackage[usenames,dvipsnames,svgnames,table]{xcolor}
\usepackage[normalem]{ulem}
\usepackage{caption}
\renewcommand{\thesubfigure}{(\roman{subfigure})}
%\renewcommand{\thesubfigure}{}


\newcommand{\beq}{\begin{equation}}
\newcommand{\eeq}{\end{equation}}
\newcommand{\beqry}{\begin{eqnarray}}
\newcommand{\eeqry}{\end{eqnarray}}
\newcommand{\beqrys}{\begin{subequations}\begin{eqnarray}}
\newcommand{\eeqrys}{\end{eqnarray}\end{subequations}}
\newcommand{\fqu}{\begin{bmatrix} Q \\ U \end{bmatrix}}
\newcommand{\xpm}{\begin{bmatrix} _{+2}X \\ _{-2}X \end{bmatrix}}
\newcommand{\equ}{\begin{bmatrix} Q \\ U \end{bmatrix}_{E}}
\newcommand{\bqu}{\begin{bmatrix} Q \\ U \end{bmatrix}_{B}}
\newcommand{\eblm}{\begin{bmatrix} a^E \\ a^{B} \end{bmatrix}}
\newcommand{\qutox}{\begin{bmatrix} \mathbb{1} & i\mathbb{1}  \\  \mathbb{1} & -i\mathbb{1}   \end{bmatrix}}
\newcommand{\qutoxd}{\begin{bmatrix} \mathbb{1} & \mathbb{1}  \\  -i\mathbb{1} & i\mathbb{1}   \end{bmatrix}}
\newcommand{\ymat}{\begin{bmatrix} _{+2}Y & 0  \\  0 & _{-2}Y  \end{bmatrix}}
\newcommand{\ymatc}{\begin{bmatrix} _{+2}Y^* & 0  \\  0 & _{-2}Y^*  \end{bmatrix}}
\newcommand{\yzmat}{\begin{bmatrix} _{0}Y & 0  \\  0 & _{0}Y  \end{bmatrix}}
\newcommand{\yzmatc}{\begin{bmatrix} _{0}Y^* & 0  \\  0 & _{0}Y^*  \end{bmatrix}}
\newcommand{\ope}{\hat{O}_{E}}
\newcommand{\opb}{\hat{O}_{B}}
\newcommand{\bmat}{\begin{bmatrix}}
\newcommand{\emat}{\end{bmatrix}}
\newcommand{\mdi}{\mathcal{D}/\mathcal{I}}
\newcommand{\mm}{\mathcal{M}}
\newcommand{\md}{\mathcal{D}}
\newcommand{\mi}{\mathcal{I}}
\def\vp#1{$\bar{P}_{#1}$ }
\def\vs{$\bar{S}~$}
\def\eq#1{{Eq.~(\ref{#1})}}
\def\sec#1{{Sec.~\ref{#1}}}
\def\fig#1{{Fig.~\ref{#1}}}
\def\app#1{{Appendix~\ref{#1}}}
\def\ymat#1{\begin{bmatrix} _{+2}Y_{#1} & 0  \\  0 & _{-2}Y_{#1}  \end{bmatrix}}
\def\ymatc#1{\begin{bmatrix} _{+2}Y^{T*}_{#1} & 0  \\  0 & _{-2}Y^{T*}_{#1}  \end{bmatrix}}
\def\yzmat#1{\begin{bmatrix} _{0}Y_{#1} & 0  \\  0 & _{0}Y_{#1}  \end{bmatrix}}
\def\yzmatc#1{\begin{bmatrix} _{0}Y^*_{#1} & 0  \\  0 & _{0}Y^*_{#1}  \end{bmatrix}}

% Text comments and highlights
\newcommand{\st}{\sout}
\newcommand{\revisit}{\textcolor{red}}
\newcommand{\comment}{ $\Rightarrow$ \textcolor{green}  }
\newcommand{\rfedit}{\textcolor{magenta}  }
\newcommand{\soothe}{\textcolor{BlueViolet}  }

\graphicspath{ {"./figures/"//} }

\begin{document}
%\title{Quantifying the locality of operators that translate Stokes $Q$ \& $U$ to scalars $E$ \& $B$}
%\title{Real space operators for Stokes $Q$ \& $U$ to scalars $E$ \& $B$ translation}
\title{Real-space computation of $E$/$B$-mode maps I: Formalism and Compact Kernels}
\author[]{Aditya Rotti and}
\author[]{Kevin Huffenberger}
\affiliation[]{Department of Physics, Florida State University, Keen Physics Building, 77 Chieftan Way, Tallahassee, Florida, U.S.A.}
\emailAdd{adityarotti@gmail.com}
\emailAdd{khuffenberger@fsu.edu}

\abstract{
  \noindent 
  \revisit{  We derive full-sky, real-space operators that convert between polarization Stokes $Q$/$U$ parameters to the coordinate-independent, scalar $E$/$B$ modes that are widely used in Cosmic Microwave Background and cosmic shear analysis.  The covolution kernels split naturally into angular and radial parts, and we show explicitly how the spatial extent of the convolution kernel depends on the targeted band-limit.  We show that a arbitrary radial dependence can produce $E$/$B$-like maps and that these are simply filtered versions of the standard $E$/$B$ maps.  This allows us to compute $E$/$B$ maps in real space with a compactly-supported kernel, an approach that can guarantee the avoidance of known foreground regions and can be employed in a massively parallel scheme at high-resolution.  We can compute power spectra using standard techniques, and recover the power spectrum of the sky with a simple window function.  
    %
     We cast the standard CMB polarization analysis operators in a matrix-vector notation which facilitates the derivations and shows that the kernels relate directly to spin-0 $Y_{\ell 2}$ spherical harmonic functions.  This new notation also allows us to derive real space operators which decompose the measured Stokes parameters into their even and odd-parity parts, without ever evaluating the scalar $E$/$B$ fields themselves.  This paper is the first part in a series of papers that explore real-space computation of polarization modes and their applications.
  } 
  \\
  \\
%  We derive fully sky real space operators to translate Stokes Q \& U parameters to scalars $E$ \& $B$ and vice versa. We explicitly show that these local real space operator are fully characterized by the spin-0 $Y_{\ell 2}$ spherical harmonic functions. These real space operators make transparent the association of radial and tangential patterns of polarization with the E modes and that of  the clockwise and anti-clockwise pinwheel patterns of polarization with B-modes. We cast the standard CMB polarization analysis operators in a matrix-vector notation which elucidate these derivations. Using this new notation also allows us to derive real space operators which decompose the measured Stokes parameters into those corresponding to E-modes and B-modes respectively, without ever evaluating the scalar fields themselves. We use these analytical derivations us to quantify the non-local nature of the relation between the two representations of the polarization field. We present a prescription for generalizing these operators, which allow the non-local behavior of these operators to be a tunable parameter. We derive constraints on these generalized operators to ensure reliable recovery of the standard $E$ and $B$ fields and the corresponding power spectra. \revisit{This is part 1 of a series of papers where we have derived the real space operators. In the following papers we will explore some of their applications.} \comment{Needs better blending.}
}

\maketitle
\begingroup
\let\clearpage\relax
\section{Introduction}

In this work we follow the convection in which bar-ed variables correspond to those in real space, while the tilde-ed variables correspond to those in harmonic space \cite{Zaldarriaga2001a}. 

This paper is organized in the following manner: In \sec{sec:pol-intro} we present a primer on the description of CMB polarization on the sphere and introduce the matrix notation which provides a more concise description of the same. In \sec{sec:real_space_operators} we introduce the necessary tools  and discuss the derivations of the real space operators. In \sec{sec:visualize_operator} we evaluate the real space operators and present visualizations of these functions. Here we discuss the locality of the real space E \& B operators. In \sec{sec:numerical_implementation} we implement these operators to evaluate E \& B  maps from the Stokes parameters Q \& U and compare these maps and their spectra from those derived using Healpix. We conclude with a discussion and the scope of this new method of analyzing CMB polarization in \sec{sec:discussion}.

\section{E/B description of CMB polarization} \label{sec:pol-intro}

%--------------------------------------------------------
%--------------------------------------------------------
\subsection{Polarization primer}\label{sec:pol-primer}
The CMB polarization is measured in terms of Stokes Q and U parameters. These measurements can be combined to form the complex spin 2 polarization field as follows,
%
\beqry \label{eq:spin-pol}
_{\pm 2}\bar{X}(\hat{n}) &=& Q(\hat{n}) \pm i U (\hat{n}) \nonumber \\ &=& \sum_{\ell m}  {_{\pm 2}} \tilde{X}_{\ell m}  {_{\pm 2}}Y_{\ell m} (\hat{n}) \,.
\eeqry
%
Since these measured quantities depend on the local coordinate system making it is cumbersome to work with them. To overcome this, one describes the CMB polarization field in terms of a scalar field denoted by $E(\hat{n}) $ and a pseudo scalar field $B(\hat{n}) $ \cite{Kamionkowski1997}. These scalar fields are related to the spin-2 polarization field $_{\pm 2}X(\hat{n})$ through the following relation,
%
\beq \label{eq:ebdef}
\mathcal{E}(\hat{n}) = -\frac{1}{2} \big[ \bar{\eth}^2 _{+ 2}X(\hat{n})  +  \eth^2 _{- 2}X(\hat{n}) \big] ~\,;~\mathcal{B}(\hat{n}) = -\frac{1}{2i} \big[ \bar{\eth}^2 _{+ 2}X(\hat{n})  -  \eth^2 _{- 2}X(\hat{n}) \big] \,,
\eeq
%
where $\eth$ and $\bar{\eth}$ denote the spin raising and lowering operators respectively. These $E$ and $B$ fields are spin-0 fields similar to the temperature anisotropies and hence their value are independent of the coordinate system definitions. The spin raising and lowering operators have the following properties \cite{goldberg67},
%
\beqrys \label{eq:spinopylm} 
\eth _s Y_{lm}(\hat{n}) &=& \sqrt{(\ell-s)(\ell+s+1)} _{s+1} Y_{lm}(\hat{n}) \,, \\
\bar{\eth} _s Y_{lm}(\hat{n}) &=& -\sqrt{(\ell+s)(\ell-s+1)} _{s-1} Y_{lm}(\hat{n}) \,, 
\eeqrys
%
where $_s Y_{lm}(\hat{n}) $ denote the spin-s spherical harmonics.

Using \eq{eq:ebdef} and the properties of the spin raising and lowering operators given in \eq{eq:spinopylm} it can be shown that the scalar fields $\mathcal{E}/\mathcal{B}$ are defined via the following set of equations,
%
\beq \label{eq:pseudo}
\mathcal{E}(\hat{n}) = \sum_{\ell m} a^{E}_{\ell m} \sqrt{\frac{(\ell+2)!}{(\ell-2)!}} Y_{\ell m} (\hat{n}) ~\,;~ \mathcal{B}(\hat{n})  =\sum_{\ell m} a^{B}_{\ell m} \sqrt{\frac{(\ell+2)!}{(\ell-2)!}} Y_{\ell m} (\hat{n}) \,,
\eeq
%
where the harmonic coefficients of  $\mathcal{E}/\mathcal{B}$ fields are related to the spin harmonic coefficients of the polarization field through the following equations,
%
\beq\label{eq:x2eb}
a^{E}_{\ell m} = -\frac{1}{2} \Big[ {}_{+2}X_{\ell m} + _{-2}X_{\ell m} \Big] ~\,;~a^{B}_{\ell m} = -\frac{1}{2i} \Big[ {}_{+2}X_{\ell m} - _{-2}X_{\ell m} \Big] 
\eeq
%
In the remainder of this article, we will work with the scalar $E$ and pseudo scalar $B$ fields as defined by the following expressions, 
%
\beq \label{eq:pseudo}
E(\hat{n}) = \sum_{\ell m} a^{E}_{\ell m} Y_{\ell m} (\hat{n}) ~\,;~ B(\hat{n})  =\sum_{\ell m} a^{B}_{\ell m} Y_{\ell m} (\hat{n}) \,.
\eeq
%
Note that the two set of fields $\mathcal{E}/\mathcal{B}$ and $E/B$ differ, since their spherical harmonic coefficients of expansion differ by the factor of $\sqrt{\frac{(\ell+2)!}{(\ell-2)!}}$.
\filbreak	
\subsection{Matrix notation} \label{sec:mat_pol_intro}
In this section we cast the relation introduced in Sec.~\ref{sec:pol-primer} in matrix notation\footnote{While we work with the matrix and vector sizes given in terms of Healpix pixelization parameter $\rm N_{\rm pix}$, all the relations are equally valid in the continuum limit attained by allowing $\rm N_{\rm pix}\rightarrow \infty$}. This representation will make transparent the derivation of the real space operators we discuss in the following sections. We adopt a convention in which real space quantities are denoted by bar-ed variable while those in harmonic space are denoted by tilde-ed variables.\\
We begin by introducing the matrices encoding the spin spherical harmonic basis vectors,
%
\beq
_sB= \bmat _{+s}Y & 0 \\ 0 & _{-s}Y \emat _{2 \rm N_{\rm pix} \times 2 \rm N_{\rm alms}} \,,
\eeq
%
where $s$ denotes the spin of the basis functions. For this work we will only be working with cases $s \in [0,2]$. In this notation, each column can be mapped to a specific harmonic basis function marked by the pair of indices:$(\ell,m)$ and each row maps to a specific position on the sphere. Note that this matrix is in general not a square matrix. Generally the number of columns is determined by the scheme used to discretely represent the sphere and the number of rows is set by the number of basis functions of interest (often determined the band limit).

We now define the different data vectors and their representation in real and harmonic space as follows,
%
\beqrys
\bar{S} &=& \bmat E \\ B  \emat_{2 \rm N_{\rm pix} \times 1} ~~~~;~~ \bar{X} = \bmat _{+2}X \\ _{-2}X \emat_{2 \rm N_{\rm pix} \times 1} ~~;~~\bar{P} =\fqu_{\tiny {2 \rm N_{\rm pix} \times 1}} \,, \\
\tilde{S} &=& \bmat a^{E} \\ a^{B} \emat _{2 \rm N_{\rm alms} \times 1}  ~~; ~~ \tilde{X} = \bmat _{+2} \tilde{X} \\ _{-2} \tilde{X} \emat_{2 \rm N_{\rm alms} \times 1} \,.
\eeqrys
%
The different symbols have the same meaning as that discussed in \sec{sec:pol-primer}, except that the subscript $_{\ell m}$ for the spherical harmonic coefficients of expansion is suppressed to avoid clutter in notation.

Next we define the operators which govern the transformations between different representations of the polarization field as follows,
%
\beqrys
\bar T &=& \qutox_{2 \rm N_{\rm pix} \times 2 \rm N_{\rm pix}} ~~;~~ \bar T^{-1} = \frac{1}{2} \bar T^{\dagger} \,, \\
\tilde T &=& -\qutox_{2 \rm N_{\rm alms} \times 2 \rm N_{\rm alms}} ~~;~~ \tilde T^{-1} = \frac{1}{2} \tilde T^{\dagger} \,. 
\eeqrys
%
Using the data vectors and the all the operators defined in this section we now write down, in compact notation, the forward and inverse relations between different representations of the polarization field as follows,
%
\beqrys \label{eq:pol_data_relns}
\bar{X} &=& \bar T * \bar{P} ~~;~~\bar{P} = \frac{1}{2} \bar T^{\dagger} * \bar{X} \,, \\
\bar X &=&  {_2B} * \tilde X  ~~;~~ \tilde X ={_2B}^{\dagger} * \bar X  \,, \\
\tilde{X} &=& \tilde T * \tilde{S} ~~;~~ \tilde{S} = \frac{1}{2}\tilde T^{\dagger} * \tilde{X} \,.\\ 
\bar S &=&  {_0B} * \tilde S ~~;~~  \tilde S =  {_0B}^{\dagger} * \bar S \,.
%\tilde X &=&  {_2B}^{\dagger} * \bar X ~~;~~ \tilde{X} = \tilde T * \tilde{S} \,, \\
%\bar{S} &=& {_0B}*\tilde S ~~;~~ \tilde{S} = \frac{1}{2}\tilde T^{\dagger} * \tilde{X} \,.\\
%\bar{X} &=& \bar T * \bar{P} ~~;~~ \tilde{X} = \tilde T * \tilde{S} \,, \\
%\bar{P} &=& \frac{1}{2} \bar T^{\dagger} * \bar{X} ~~;~~ \tilde{S} = \frac{1}{2}\tilde T^{\dagger} * \tilde{X} \,. \\
\eeqrys
%
Next we introduce the harmonic space operators, which project the harmonic space data vector to E or B subspace,
%
\beqrys
\tilde O_E &=& \bmat \mathbb{1} & \mathbb{0} \\ \mathbb{0} & \mathbb{0} \emat _{2 \rm N_{\rm alms} \times 2 \rm N_{\rm alms} }   ~~;~~ \tilde S_E = \tilde O_E* \tilde S \,,\\
\tilde O_B &=& \bmat \mathbb{0} & \mathbb{0} \\ \mathbb{0} & \mathbb{1} \emat _{2 \rm N_{\rm alms} \times 2 \rm N_{\rm alms} } ~~; ~~ \tilde S_B = \tilde O_B *\tilde S \,.
\eeqrys
%
Note that these harmonic space matrices are idempotent, orthogonal to each other and their sum is an identity matrix as can be explicitly seen via the following relations, 
%
\beqrys\label{eq:eb_har_proj}
\tilde O_E * \tilde O_E&=& \tilde O_E ~~;~~  \tilde O_B * \tilde O_B= \tilde O_B \,,\\
 \tilde O_E * \tilde O_B&=& \mathbb{0} \,, \\
 \tilde O_E + \tilde O_B&=& \mathbb{1} \,.
\eeqrys
%
Here it is important to note that these relations are exact in harmonic space.  \revisit{In the following sections our aim is to derive real space analogues of these harmonic space operators. }
%%%%%%%%%%%%%%%%%%%%%%%%%%%%%%%%%%%
\section{Real space operators} \label{sec:real_space_operators}
In this section we derive the real space operators which translate the Stokes parameters Q \& U to E \& B fields and vice versa. We also derive real space operators for directly(without first evaluating the E \& B field themselves) decomposing the Stokes  parameters Q \& U in to Stokes parameters that correspond to the E \& B fields respectively. We extensively make use of the matrix notation introduces in \sec{sec:mat_pol_intro} for these derivations.

All the results are most conveniently expressed as functions of the Euler angles  $\alpha, ~\beta~\&~ \gamma$ on the sphere. Generically, the Euler angles define the rotations that transforms the local cartesian coordinate system defined at the sphere position $\hat{n}_i \equiv (\theta_i,\phi_i)$ such that it aligns with the local cartesian coordinate system at the location $\hat{n}_j \equiv (\theta_j,\phi_j)$ \cite{varshalovich}. The Euler angles can be evaluated as the following functions of the angular coordinates of the points $\hat{n}_i \equiv (\theta_i , \phi_i)$ and $\hat{n}_j \equiv (\theta_j, \phi_j)$,
%
\beqrys \label{eq:fn_euler}
\cos(\beta) &=& \sin(\theta_i)\sin(\theta_j) \cos(\phi_i -\phi_j) + \cos(\theta_i)\cos(\theta_j) \,,\\
\tan(\alpha) &=& \frac{\sin(\phi_i - \phi_j) \sin(\theta_i) \sin(\theta_j)}{\cos(\theta_i) \cos(\beta) - \cos(\theta_j)} \,, \\
\tan(\gamma) &=& \frac{\sin(\phi_i - \phi_j) \sin(\theta_i) \sin(\theta_j)}{\cos(\theta_j) \cos(\beta) - \cos(\theta_i)}\,,
\eeqrys
%
where $\beta$ denotes the angular distance between the two points $\hat{n}_i$ \& $\hat{n}_j$ on the sphere, while the angles $\alpha$ \& $\gamma$ define rotations which co-align the coordinate axes of the two local coordinate systems. While evaluating the above functions we follow the convention that $\beta$ lies in the domain $[0, \pi]$ and that the angles $\alpha ~\&~ \gamma$ lie in the domain $[-\pi,\pi]$. It is important to assign the proper signs to $\alpha ~\&~ \gamma$ by duly accounting for the signs of the term in the numerator and the denominator. 
%--------------------------------------------------------
%--------------------------------------------------------
\subsection{Evaluating E \& B fields from measured Stokes parameters Q \& U}\label{sec:qu2eb}
In \sec{sec:pol-primer} we discussed how the scalar fields E \& B are derived from the Stokes parameters Q \& U. To reiterate, this process involved taking the spin harmonic transform of the complex spin-2 field $({}_{\pm2} \bar X)$, taking linear combinations of the resultant coefficients of expansion $({}_{\pm 2} \tilde X_{\ell m})$ and evaluating the forward spin-0 transform to derive the scalar E \& B fields. Here we derive the real space convolution kernels on the sphere which can be used to directly evaluate the scalar E \& B fields on the sphere.  
We use the relations given in \revisit{\eq{eq:pol_data_relns}}, to write down an equation relating the real space vector of scalars $\bar{S}^{\dagger}=[E, B]$ to the polarization vector $\bar{P}^{\dagger}=[Q, U]$ as given below,
%
\beqrys
\bar{S} &=& {_0B} *\tilde T^{-1}* {_2B^{\dagger}} *\bar T *\bar{P} = \frac{1}{2} {_0B} *\tilde T^{\dagger} {_2B^{\dagger}} *\bar T *\bar{P}   \,, \\
&=&  \bar O *\bar{P} \,.
\eeqrys
%
The explicit form of the real space operator $\bar O$ can be derived by contracting over all the matrix operators. This procedure of contracting over the operators is explicitly worked out in the following set of equations,
%
\beqrys
\bar{O} &=& \frac{1}{2} {_0B} *\tilde T^{\dagger} *{_2B^{\dagger}} *\bar T \,, \\
&=& -0.5 \yzmat{i} \qutoxd \ymatc{j} \qutox   \,, \\
&=& -0.5 \begin{bmatrix} \sum ({}_{0}Y_i ~{}_{2}Y^{T*}_j  +  {}_{0}Y_i~ {}_{-2}Y^{T*}_j) & {\rm i}  \sum ({}_{0}Y_i~ {}_{2}Y^{T*}_j - {}_{0}Y_i ~{}_{-2}Y^{T*}_j)  \\  - {\rm i} \sum  ({}_{0}Y_i {}_{2}Y^{T*}_j - {}_{0}Y_i {}_{-2}Y^{T*}_j) & \sum ({}_{0}Y_i {}_{2}Y^{T*}_j + {}_{0}Y_i {}_{-2}Y^{T*}_j)  \end{bmatrix} \,, \label{eq:qu2eb_ker_1}
\eeqrys
%
where the symbol ${}_{0}Y_i$ is used to denote the matrix ${}_{0}Y_{\hat{n}_i \times \ell m} \equiv {}_{0}Y_{\ell m}(\hat{n}_i)$, the symbol ${}_{\pm 2}Y^{T*}_j$ is used to denote the matrix ${}_{\pm 2}Y^*_{\ell m \times \hat{n}_j} \equiv {}_{\pm 2}Y^*_{\ell m}(\hat{n}_j)$ and the summation is over the multipole indices $\ell,m$. Using the conjugation properties of the spin spherical harmonic functions it can be shown that the following relation holds true,
%
\beq
 \left [\sum_{\ell m} {}_{0}Y_{\ell m}(\hat{n}_i){}_{+2}Y^*_{\ell m}(\hat{n}_j)\right]^* = \sum_{\ell m} {}_{0}Y_{\ell m}(\hat{n}_i){}_{-2}Y^*_{\ell m}(\hat{n}_j) \,.
 \eeq
 %
 where the terms on either side of the equation are those that appear in \eq{eq:qu2eb_ker_1}.
Therefore the different parts of the real space operators  are completely specified in terms of the complex function,
%
\beqrys
\mathcal{M}( \hat{n}_i, \hat{n}_j)  &=& \mathcal{M}_{r} + i \mathcal{M}_{i}  \,,\nonumber \\ 
&=&\sum_{\ell m} {{_0}Y}_{\ell m}(\hat n_i) {{_2}Y}^*_{\ell m}(\hat n_j) = \sum_{\ell} \sqrt{\frac{2\ell+1}{ 4 \pi}}{{_0Y}^*_{\ell 2}}(\beta_{ij},\alpha_{ij})\,,\\
&=&  \Big [ \cos(2 \alpha_{ij}) - i \sin(2 \alpha_{ij} ) \Big]   \sum_{\ell=\ell_{\rm min}}^{\ell_{\rm max}} {\frac{2\ell+1}{ 4 \pi}} \sqrt{\frac{(\ell-2)!}{(\ell+2)!}}P_{\ell 2} (\cos\beta_{ij}) \,, \label{eq:rad_ker_queb} \\
&=&  \Big [ \cos(2 \alpha_{ij}) - i \sin(2 \alpha_{ij} ) \Big] f(\beta_{ij},\ell_{\rm min},\ell_{\rm max}) \,, 
\eeqrys
%
where we have used the property of summation over spin spherical harmonics (see \eq{eq:sum_spin_shf}) listed in Appendix \ref{sec:ylm_mathprop}. Here we first note that this function does not depend on the Euler angle $\gamma$. This function has a part which depends only on the Euler angle $\alpha$ and this part of the function has no multipole dependence.  except the factor of 2 which arises because the polarization field is a spin-2 field. The other part of the function $f(\beta,\ell_{\rm min},\ell_{\rm max})$ depends only on the Euler angle $\beta$ and completely incorporates the multipole dependence of the function. $f(\beta,\ell_{\rm min},\ell_{\rm max})$ will be often be referred to as the radial kernel. The radial kernel is what determines the locality of the operator which translates the Stokes parameters Q \& U to the scalars E \& B.  

Finally the real space operator can be cast in this simple form,
%
\beq\label{eq:op_qu2eb}
\bar O =\bmat  -\mathcal{M}_{r} & -\mathcal{M}_{i} \\  \mathcal{M}_{i}  & -\mathcal{M}_{r} \emat_{2 N_{\rm pix} \times 2 N_{pix}} = -f(\beta_{ij},\ell_{\rm min},\ell_{\rm max})\bmat \cos(2 \alpha_{ij}) & \sin(2\alpha_{ij})\\  -\sin(2 \alpha_{ij})  & \cos(2 \alpha_{ij}) \emat \,,
\eeq
%
where i,j indices map to the location $\hat{n}_i$ and $\hat{n}_j$ on the sphere. \revisit{A similar equation for real space E \& B operators was derived in \cite{Zaldarriaga2001a}, however those results are derived for the flat sky case and do not explicitly derive the radial kernel.} \comment{Maybe a discussion on this should be in the conclusions.}

The scalar fields E \& B can now be directly derived from the measured Stokes Q \& U parameters by evaluating the following expression,
%
\beq \label{eq:qu2eb_convolution}
\bmat E_i \\ B_i  \emat= -f(\beta_{ij},\ell_{\rm min},\ell_{\rm max})\bmat \cos(2 \alpha_{ij}) & \sin(2\alpha_{ij})\\  -\sin(2 \alpha_{ij})  & \cos(2 \alpha_{ij}) \emat  \bmat Q_j \\ U_j  \emat \Delta \Omega\,,
\eeq
%
where we have used the Einstein summation convention: repeated indices are summed over. The factor $\Delta \Omega$ accounts for the finite pixel size and is important for proper normalization. \revisit{This has an elegant interpretation: to derive the E and/or B field at any given position we need to find the cosine quadrupole transform and the sine quadrupole transform of the Stokes Q \& U parameters on circles around this position, weigh the transform by the value of the function $f(\beta,\ell_{\rm min},\ell_{\rm max})$, $\beta$ being the radius of the circle and sum up the results with appropriate signs, to construct the respective scalar fields.}
\rfedit{While the azimuthal operations do not depend on the choice of basis functions, the radial kernel is completely determined by the choice of the basis functions. One can now think of constructing alternate basis functions which have different radial fall off.}
%--------------------------------------------------------
%--------------------------------------------------------
\subsection{Evaluating Stokes parameters Q \& U fields from E \& B fields}\label{sec:eb2qu}
The real space operator which translates E \& B fields to Stokes parameters Q \& U is derived using a similar procedure. The inverse operator is given by the following expression,
%
\beqry
\bar{P} &=& \bar{T}^{-1} *{_2B} *\tilde T *{_0B^{\dagger}}\bar{S} = \frac{1}{2} \bar{T}^{\dagger} *{_2B} *\tilde T *{_0B^{\dagger}}\bar{S}   \\
&=&  \bar O^{-1} *\bar{S}
\eeqry
%
We do not provide the explicit calculations here, since the real space inverse operator can be derived by contracting over all the matrix operators using a procedure nearly identical to that discussed in the previous section. The inverse operator is given by the following expression,
%
\beq
{\bar O}^{-1}=\bmat - \mathcal{M}_{r} & \mathcal{M}_{i} \\  -\mathcal{M}_{i}  & - \mathcal{M}_{r} \emat_{2 N_{\rm pix} \times 2 N_{pix}} =-f(\beta_{ij},\ell_{\rm min},\ell_{\rm max})\bmat \cos(2 \alpha_{ij}) & -\sin(2\alpha_{ij})\\  \sin(2 \alpha_{ij})  & \cos(2 \alpha_{ij}) \emat \,,
\eeq
%
where all the symbols have the same meaning as discussed in \sec{sec:qu2eb}.
Note that the kernel is different by a mere change in sign on the off-diagonals of the block matrix as compared to \eq{eq:op_qu2eb}.
We can evaluate the Stokes Q \& U parameters from the scalar E \& B  fields by evaluating the following expression,
%
\beq
\bmat Q_i \\ U_i  \emat=-f(\beta_{ij})\bmat \cos(2 \alpha_{ij}) & -\sin(2\alpha_{ij})\\  \sin(2 \alpha_{ij})  & \cos(2 \alpha_{ij}) \emat  \bmat E_j \\ B_j  \emat \Delta\Omega \,,
\eeq
%
where again the Einstein summation convention is implied and all the symbols have their usual meaning.

\comment{How does the radial kernel reduce to unity on evaluating the the operator on to its inverse ? }
%--------------------------------------------------------
%--------------------------------------------------------
\subsection{Decomposing Q \& U Stokes parameters into those corresponding to E \& B modes respectively}
The Stokes Q \& U parameters can be decomposed into the scalar modes E \& B and vice verse, as seen in the previous sections. The E \& B modes are orthogonal to each other. It is possible to decompose the Stokes Q \& U parameters into those that purely contribute to E modes and those that purely contribute to the B mode of polarization. We can only measure the total Stokes parameters which is a sum of the Stokes Q \& U corresponding to the respective scalar modes.  In this section we derive the real space operators which directly decompose the total measured Stokes Q \& U parameters to Stokes parameters corresponding to the scalar fields E \& B respectively, \textit{without ever having to evaluate the E \& B modes explicitly}. Again the procedure is analogous to that discussed in \sec{sec:qu2eb}, though the algebra is a little more involved. Here we use the harmonic space E/B projection operators $\tilde O_{E/B}$, defined in \eq{eq:eb_har_proj}, to derive the respective real space operators. It can be shown that the Stokes parameters corresponding to each scalar mode are given by the following expressions,
%
\beqry
\bar{P}_E &=&  [\bar T^{-1} * {_2B} *\tilde T * \tilde O_E* \tilde T^{-1}* {_2B^{\dagger}} *\bar T] *\bar{P}  \,, \\
&=& [\frac{1}{4} \bar T^{\dagger } * {_2B} *\tilde T * \tilde O_E* \tilde T^{\dagger} * {_2B^{\dagger}} *\bar T ]*\bar{P}  \,, \nonumber \\
&=&  \bar O_{E}*\bar{P} \,,\nonumber \\
\bar{P}_B &=&  [\bar T^{-1}* {_2B}* \tilde T* \tilde O_B* \tilde T^{-1}* {_2B^{\dagger}}* \bar T]*\bar{P}  \,, \\
&=& [\frac{1}{4} \bar T^{\dagger } * {_2B} *\tilde T * \tilde O_B* \tilde T^{\dagger} *{_2B^{\dagger}} *\bar T] *\bar{P}   \,, \nonumber\\
&=&  \bar O_{B}*\bar{P} \,. \nonumber
\eeqry
%
We contract over all the matrix operators to arrive at the the real space operators. On simplification it can be shows that the real space operator takes up the following form,
%
\beq
\bar O_{E/B} = 0.5 \bmat \mathcal{I}_{r} \pm \mathcal{D}_{r} & -\mathcal{I}_{i} \pm \mathcal{D}_{i} \\  -\mathcal{I}_{i} \pm \mathcal{D}_{i}  & \mathcal{I}_{r} \mp \mathcal{D}_{r} \emat_{2 N_{\rm pix} \times 2 N_{pix}} \,,\\
\eeq
where $\mathcal{I}_{r} ~\&~ \mathcal{D}_{r}$ and $\mathcal{I}_{i} ~\&~ \mathcal{D}_{i}$ are the real and complex parts of the following complex functions,
\beqry
\mathcal{I} &=& \mathcal{I}_{r} + i \mathcal{I}_{i} = \sum_{\ell m} {_2Y}_{\ell m}(\hat n_i) {_2Y}^*_{\ell m}(\hat n_j) \,, \nonumber \\
\mathcal{D}  &=& \mathcal{D}_{r} + i\mathcal{D}_{i} = \sum_{\ell m} {_2Y}_{\ell m}(\hat n_i) {_{-2}Y}^*_{\ell m}(\hat n_j) \,.\nonumber
\eeqry
%
These functions can be further simplified using the properties of spin spherical harmonics listed in Appendix~\ref{sec:ylm_mathprop}. Specifically it can be shown that these functions reduce to the following mathematical forms,
%
\beqrys \label{eq:fn_i}
\mathcal{I}(\hat{n}_i, \hat{n}_j) &=& \sum_{\ell} \sqrt{\frac{2\ell+1}{ 4 \pi}}{_2Y}_{\ell -2}(\beta_{ij}, \alpha_{ij}) ~ \rm{e}^{- i2 \gamma_{ij}} \label{eq:healpix-compatible-i} = \mathcal{I}_r + i \mathcal{I}_i \,, \\
\mathcal{I}_r + i \mathcal{I}_i &=& \Big [ \cos(2 \alpha_{ij} +  2\gamma_{ij}) - i \sin(2 \alpha_{ij} +  2 \gamma_{ij}) \Big]   _{-2}f(\beta_{ij},\ell_{\rm min},\ell_{\rm max}) \,,
\eeqrys
%
%
\beqrys \label{eq:fn_d}
\mathcal{D}(\hat{n}_i, \hat{n}_j) &=& \sum_{\ell} \sqrt{\frac{2\ell+1}{ 4 \pi}}{_2Y}_{\ell +2}(\beta_{ij}, \alpha_{ij}) ~ \rm{e}^{- i2 \gamma_{ij}} \label{eq:healpix-compatible-m} =\mathcal{D}_r + i \mathcal{D}_i \,, \\
\mathcal{D}_r + i \mathcal{D}_i &=&  \Big [ \cos(2 \alpha_{ij} - 2\gamma_{ij}) + i \sin(2 \alpha_{ij} -  2 \gamma_{ij}) \Big]   _{+2}f(\beta_{ij},\ell_{\rm min},\ell_{\rm max}) \,,
\eeqrys
%
where the functions,
%
\beq
{}_{\pm2}f(\beta,\ell_{\rm min},\ell_{\rm max}) = \sum_{\ell=\ell_{\rm min}}^{\ell_{\rm max}} \sqrt{\frac{2\ell+1}{ 4 \pi}} _{ \pm 2}{f}_{\ell}(\beta) \label{eq:f2_rad_ker}\,,
\eeq
%
can be expressed in terms of $P_{\ell}^2$ Legendre polynomials and are given by the following explicit mathematical forms,
 %
 \beqry
 _{\pm 2}f_{\ell}(\beta) &=& 2 \frac{(\ell-2)!}{(\ell+2)!}  \sqrt{\frac{2\ell +1 }{4 \pi}} \Bigg[ - P_{\ell}^{2} (\cos  \beta) \left( \frac{\ell-4}{\sin^2 \beta} + \frac{1}{2}\ell(\ell-1) \pm \frac{2 (\ell-1) \cos \beta}{\sin^2 \beta} \right) \nonumber \\ 
&+& P_{\ell-1}^2 (\cos \beta) \left( (\ell+2) \frac{\cos \beta}{\sin^2 \beta} \pm \frac{2 (\ell+2)}{ \sin^2 \beta } \right) \Bigg] \,. \label{eq:rad_ker_quequbqu}
 \eeqry
 %
Finally, the Stokes parameters corresponding to the respective scalar fields can be derived by evaluating the following expression, 
 %
\beqry
\bmat Q_i \\ U_i  \emat_{E/B} &=&0.5 \Bigg\lbrace {}_{-2}f(\beta_{ij},\ell_{\rm min},\ell_{\rm max}) \bmat \cos(2 \alpha_{ij} + 2\gamma_{ij}) & \sin(2\alpha_{ij} +2 \gamma_{ij}) \\  \sin(2\alpha_{ij} +2 \gamma_{ij})  & \cos(2 \alpha_{ij} + 2 \gamma_{ij}) \emat  \bmat Q_j \\ U_j  \emat   \\ &\pm& {}_{+2}f(\beta_{ij},\ell_{\rm min},\ell_{\rm max}) \bmat \cos(2 \alpha_{ij} - 2\gamma_{ij}) & - \sin(2\alpha_{ij} - 2 \gamma_{ij}) \\  -\sin(2\alpha_{ij} - 2 \gamma_{ij})  & - \cos(2 \alpha_{ij} - 2 \gamma_{ij}) \emat  \bmat Q_j \\ U_j  \emat \Bigg\rbrace \Delta\Omega \,,\nonumber
\eeqry
%
where all the symbols have their usual meaning.

 
  %%%% NEW D/I


\begin{figure}
  \begin{center}
  \begin{tabular}{m{8ex}m{\kernelfigwidth}m{\kernelfigwidth}|m{\kernelfigwidth}m{\kernelfigwidth}}
$b=90^\circ$&
\hspace{\kernelfigspace}\includegraphics[width=\kernelfigwidth]{new_kernel/qu2ebqu_rker_D_lat90_lon45.pdf} &
\hspace{\kernelfigspace}\includegraphics[width=\kernelfigwidth]{new_kernel/qu2ebqu_iker_D_lat90_lon45.pdf} &
\hspace{\kernelfigspace}\includegraphics[width=\kernelfigwidth]{new_kernel/qu2ebqu_rker_I_lat90_lon45.pdf} &
\hspace{\kernelfigspace}\includegraphics[width=\kernelfigwidth]{new_kernel/qu2ebqu_iker_I_lat90_lon45.pdf} \\
$b=87^\circ$&
\hspace{\kernelfigspace}\includegraphics[width=\kernelfigwidth]{new_kernel/qu2ebqu_rker_D_lat87_lon45.pdf} &
\hspace{\kernelfigspace}\includegraphics[width=\kernelfigwidth]{new_kernel/qu2ebqu_iker_D_lat87_lon45.pdf} &
\hspace{\kernelfigspace}\includegraphics[width=\kernelfigwidth]{new_kernel/qu2ebqu_rker_I_lat87_lon45.pdf} &
\hspace{\kernelfigspace}\includegraphics[width=\kernelfigwidth]{new_kernel/qu2ebqu_iker_I_lat87_lon45.pdf} \\
$b=80^\circ$&
\hspace{\kernelfigspace}\includegraphics[width=\kernelfigwidth]{new_kernel/qu2ebqu_rker_D_lat80_lon30.pdf} &
\hspace{\kernelfigspace}\includegraphics[width=\kernelfigwidth]{new_kernel/qu2ebqu_iker_D_lat80_lon30.pdf} &
\hspace{\kernelfigspace}\includegraphics[width=\kernelfigwidth]{new_kernel/qu2ebqu_rker_I_lat80_lon30.pdf} &
\hspace{\kernelfigspace}\includegraphics[width=\kernelfigwidth]{new_kernel/qu2ebqu_iker_I_lat80_lon30.pdf} \\
$b=0^\circ$&
\hspace{\kernelfigspace}\includegraphics[width=\kernelfigwidth]{new_kernel/qu2ebqu_rker_D_lat0_lon90.pdf} &
\hspace{\kernelfigspace}\includegraphics[width=\kernelfigwidth]{new_kernel/qu2ebqu_iker_D_lat0_lon90.pdf} &
\hspace{\kernelfigspace}\includegraphics[width=\kernelfigwidth]{new_kernel/qu2ebqu_rker_I_lat0_lon90.pdf} &
\hspace{\kernelfigspace}\includegraphics[width=\kernelfigwidth]{new_kernel/qu2ebqu_iker_I_lat0_lon90.pdf} \\
&
\centering $\textrm{Re} \left(\mathcal{D} \right)$ &
\centering $\textrm{Im} \left(\mathcal{D} \right)$ &
\centering $\textrm{Re} \left(\mathcal{I} \right)$ &
\centering $\textrm{Im} \left(\mathcal{I} \right)$
  \end{tabular}
  \end{center}
  \caption{Like Figure \ref{fig:vis_kernel}, but for the kernels that purify Stokes parameter into their $E/B$ parts.} \label{fig:vis_kernel_DI}
\end{figure}

%\begin{figure} \label{fig:mixing_kernel}
  %%%%  OLD VERSION %%%%
%  \centering
%\captionsetup[subfigure]{labelformat=empty}
%\subfigure{\includegraphics[width=0.125\columnwidth]{new_kernel/qu2eb_rker_rad_lat90_lon45.pdf}}\hspace{-2mm}
%\subfigure{\includegraphics[width=0.125\columnwidth]{new_kernel/qu2eb_iker_rad_lat90_lon45.pdf}}\hspace{-2mm}
%\subfigure{\includegraphics[width=0.125\columnwidth]{new_kernel/qu2eb_rker_con_lat90_lon45.pdf}}\hspace{-2mm}
%\subfigure{\includegraphics[width=0.125\columnwidth]{new_kernel/qu2eb_iker_con_lat90_lon45.pdf}}\hspace{-2mm}
%\subfigure{\includegraphics[width=0.125\columnwidth]{new_kernel/qu2ebqu_rker_D_lat90_lon45.pdf}}\hspace{-2mm}
%\subfigure{\includegraphics[width=0.125\columnwidth]{new_kernel/qu2ebqu_iker_D_lat90_lon45.pdf}}\hspace{-2mm}
%\subfigure{\includegraphics[width=0.125\columnwidth]{new_kernel/qu2ebqu_rker_I_lat90_lon45.pdf}}\hspace{-2mm}
%\subfigure{\includegraphics[width=0.125\columnwidth]{new_kernel/qu2ebqu_iker_I_lat90_lon45.pdf}}\hspace{-2mm}

%\subfigure{\includegraphics[width=0.125\columnwidth]{new_kernel/qu2eb_rker_rad_lat87_lon45.pdf}}\hspace{-2mm}
%\subfigure{\includegraphics[width=0.125\columnwidth]{new_kernel/qu2eb_iker_rad_lat87_lon45.pdf}}\hspace{-2mm}
%\subfigure{\includegraphics[width=0.125\columnwidth]{new_kernel/qu2eb_rker_con_lat87_lon45.pdf}}\hspace{-2mm}
%\subfigure{\includegraphics[width=0.125\columnwidth]{new_kernel/qu2eb_iker_con_lat87_lon45.pdf}}\hspace{-2mm}
%\subfigure{\includegraphics[width=0.125\columnwidth]{new_kernel/qu2ebqu_rker_D_lat87_lon45.pdf}}\hspace{-2mm}
%\subfigure{\includegraphics[width=0.125\columnwidth]{new_kernel/qu2ebqu_iker_D_lat87_lon45.pdf}}\hspace{-2mm}
%\subfigure{\includegraphics[width=0.125\columnwidth]{new_kernel/qu2ebqu_rker_I_lat87_lon45.pdf}}\hspace{-2mm}
%\subfigure{\includegraphics[width=0.125\columnwidth]{new_kernel/qu2ebqu_iker_I_lat87_lon45.pdf}}\hspace{-2mm}

%\subfigure{\includegraphics[width=0.125\columnwidth]{new_kernel/qu2eb_rker_rad_lat80_lon30.pdf}}\hspace{-2mm}
%\subfigure{\includegraphics[width=0.125\columnwidth]{new_kernel/qu2eb_iker_rad_lat80_lon30.pdf}}\hspace{-2mm}
%\subfigure{\includegraphics[width=0.125\columnwidth]{new_kernel/qu2eb_rker_con_lat80_lon30.pdf}}\hspace{-2mm}
%\subfigure{\includegraphics[width=0.125\columnwidth]{new_kernel/qu2eb_iker_con_lat80_lon30.pdf}}\hspace{-2mm}
%\subfigure{\includegraphics[width=0.125\columnwidth]{new_kernel/qu2ebqu_rker_D_lat80_lon30.pdf}}\hspace{-2mm}
%\subfigure{\includegraphics[width=0.125\columnwidth]{new_kernel/qu2ebqu_iker_D_lat80_lon30.pdf}}\hspace{-2mm}
%\subfigure{\includegraphics[width=0.125\columnwidth]{new_kernel/qu2ebqu_rker_I_lat80_lon30.pdf}}\hspace{-2mm}
%\subfigure{\includegraphics[width=0.125\columnwidth]{new_kernel/qu2ebqu_iker_I_lat80_lon30.pdf}}\hspace{-2mm}
%\subfigure[$ \textrm{Re} \left(\mathcal{M}_{G} \right) $]{\includegraphics[width=0.125\columnwidth]{new_kernel/qu2eb_rker_rad_lat0_lon90.pdf}}\hspace{-2mm}
%\subfigure[$\textrm{Im} \left(\mathcal{M}_{G} \right) $]{\includegraphics[width=0.125\columnwidth]{new_kernel/qu2eb_iker_rad_lat0_lon90.pdf}}\hspace{-2mm}
%\subfigure[$\textrm{Re}  \left(\mathcal{M}_{B}^* \right) $]{\includegraphics[width=0.125\columnwidth]{new_kernel/qu2eb_rker_con_lat0_lon90.pdf}}\hspace{-2mm}
%\subfigure[$\textrm{Im} \left(\mathcal{M}_{B}^* \right) $]{\includegraphics[width=0.125\columnwidth]{new_kernel/qu2eb_iker_con_lat0_lon90.pdf}}\hspace{-2mm}
%\subfigure[$\textrm{Re} \left(\mathcal{D} \right)$]{\includegraphics[width=0.125\columnwidth]{new_kernel/qu2ebqu_rker_D_lat0_lon90.pdf}}\hspace{-2mm}
%\subfigure[$\textrm{Im} \left(\mathcal{D} \right)$]{\includegraphics[width=0.125\columnwidth]{new_kernel/qu2ebqu_iker_D_lat0_lon90.pdf}}\hspace{-2mm}
%\subfigure[$\textrm{Re} \left(\mathcal{I} \right)$]{\includegraphics[width=0.125\columnwidth]{new_kernel/qu2ebqu_rker_I_lat0_lon90.pdf}}\hspace{-2mm}
%\subfigure[$\textrm{Im} \left(\mathcal{I} \right)$]{\includegraphics[width=0.125\columnwidth]{new_kernel/qu2ebqu_iker_I_lat0_lon90.pdf}}\hspace{-2mm}
%\caption{This panel of figure depicts the real and imaginary parts of the real space kernels $\mathcal{M}_{G}$, $\mathcal{M}_{B}$, $\mathcal{D}$ \& $\mathcal{I}$ respectively. These kernels have been evaluated with the band limit: $\ell \in [2,192]$ and sampled at the Healpix resolution parameter NSIDE=2048 for visual appeal. The size of each panel is approximately $16^{\circ} \times 16^{\circ}$ and the grid lines are marked at 2 degree separations. The black circles denotes the position of the center around which the kernels have been evaluated while the black star marks the location of the north galactic pole. The four rows depict the kernels at different location on the sphere and the galactic coordinates of the central pixel are specified in each panel.}
%from top to bottom rows are as follows $[b,\ell] = [0^{\circ},0^{\circ}], [87^{\circ},0^{\circ}], [87^{\circ},30^{\circ}], [80^{\circ},30^{\circ}], [0^{\circ},90^{\circ}]$.}
%\label{fig:vis_kernel}
% \end{figure}
%

%\textit{Evaluating the local kernels: }\revisit{Let us consider the case when one of the coordinates coincides with the north pole $\hat{z}=(0,0)$ (this refers to the point $\theta_0 \rightarrow 0$ while moving along the longitude $\phi_0=0$). In this case the Euler angles in the $z-y1-z2$ convention are simply given by: $(\alpha,\beta,\gamma) =(\phi_i,\theta_i,0)$, where $(\theta_i,\phi_i)$ denote the coordinates of the location $\hat{n}_i$.} Since the Euler angle $\gamma=0$ when rotations are defined with respect to the pole, the respective kernels simplify to the following forms,
%%
%\begin{subequations}
%\beqry 
%\mathcal{M}(\hat{z},\hat{n}_i) &=&  \sum_{\ell} {{}_{0}}a_{\ell 2} \, {{}_{0}}Y_{\ell 2}(\hat{n}_i) \,;\\
%\mathcal{I}(\hat{z},\hat{n}_i) &=& \sum_{\ell} {{}_{-2}}a_{\ell 2}\, {{}_{-2}}Y_{\ell 2}(\hat{n}_i) ~~\,;~~
%\mathcal{D}(\hat{z},\hat{n}_i) =\sum_{\ell} {{}_{2}}a_{\ell 2} \, {{}_{2}}Y_{\ell 2}(\hat{n}_i) \,,
%\eeqry
%\end{subequations}
%%
%where ${}_{s}a_{\ell 2} = \sqrt{\frac{2 \ell+1}{ 4 \pi}} ~~\forall ~~ s \in [0,-2,+2]$.
%The convolution kernels centered around any other location $\hat{n}_j = (\theta_j,\phi_j)$ are simply given by evaluating the respective spherical harmonic sums: $\sum_{\ell m} {}_{s}a_{\ell m}\,{}_{s}Y_{\ell m}(\hat{n}_i)$ using the rotated harmonic coefficients given by: ${}_s a_{\ell m} = D^{\ell}_{m 2}(\phi_j , \theta_j, 0) {{}_s}a_{\ell 2}$, where $D^{\ell}_{2 m}$ are the Wigner-D functions.
%These rotation operations can be carried out using inbuilt Healpix routine \textit{rotate\_alm}, while the convolution kernels can be synthesized by evaluating the respective spherical harmonic sums using the \textit{alm2map} routine. 
%\comment{Make parallels with instrument beam analysis here ? Or is it trivial since its obvious that all convolution problems can be cast in this form.}
 
%

The kernels $\mathcal{D}$ and $\mathcal{I}$ vary significantly as a function of galactic latitude of the central pixel, as seen in \fig{fig:vis_kernel_DI}. These kernels show a two fold symmetry in the vicinity of the poles.  Here the Euler angle $\gamma \approx 0$ here and therefore $e^{i2(\alpha \pm \gamma)} \approx e^{i2\alpha}$. Note that in this region, the azimuthal profile of the real and imaginary part of these kernels is identical to $-\mathcal{M}_G$.  The imaginary part of the band limited delta function $\mathcal{I}$ contributes just as much as the real part in these regions. On transiting to lower latitudes, however, $\mathcal{D}$ quickly transitions to having a four fold symmetry while $\mathcal{I}$ transitions to being dominated by the real part and behaves more like the conventional delta function. This transition can be most easily understood in the flat sky limit where $\gamma \approx -\alpha$ which leads to the resultant 4 fold symmetry seen for $\mathcal{D}$ owing to $e^{i2(\alpha - \gamma)} \approx e^{i4\alpha}$ and $\mathcal{I}$ being dominated by the real part owing to $e^{-i2(\alpha + \gamma)} \approx 1 + i0$. Since the flat sky approximation has most validity in the proximity of the equator these limiting tendencies of the respective kernels are seen in the bottom row of \fig{fig:vis_kernel_DI} which depict the kernels evaluated at the equator $b=0^{\circ}$. The middle two row depict the kernels evaluated at intermediate latitudes: $b=87^{\circ}$ and $b=80^{\circ}$ and serve to indicate the rate of this transition. As before, these kernels are invariant under changes in longitude of the central pixel with the latitude fixed.

%%%%%%%%%%%%%%%%%%%%%%%%%%%%%%%%%%%%%%%%%%%%%%%%%%%%%%%%%%%%%%%%%%%%%%%%%%
%\subsection{Quantifying the non-locality of E \& B modes} \label{sec:radial_locality}
\subsection{The non-locality of the real space operators} \label{sec:radial_locality}
%
\begin{figure}[t]
\centering
\includegraphics[width=0.8\columnwidth]{beta_kernel.pdf}
\caption{The figure depicts the radial part of the convolution kernels. These radial function have been evaluated with the band limit fixed at $\ell \in [2,24]$. The vertical dashed line marks the approximate Healpix pixel size of a NSIDE=8, which is the lowest resolution that allows access to $\ell_{\rm max}=24$.}
\label{fig:beta_kernel}
\end{figure}
%
Above, we have explored in detail the  azimuthal dependence of the real space kernels.  Here we probe the radial dependence, which both  determines the non-locality of the operators and encodes all their multipole dependencies. For illustration, \fig{fig:beta_kernel} shows the radial kernels ${_{\mm}f}, {_{\md}f},{_{\mi}f}$, evaluated using the respective multipole sums given in \eq{eq:rad_ker_queb} and \eq{eq:f2_rad_ker} in the band limit $\ell \in [2,24]$. We choose a low band limit to evaluate these radial kernels to highlight some key features of their radial profile.
%
\begin{figure}[t]
\subfigure[]{\includegraphics[width=0.325\columnwidth]{rad_ker_fn_of_ellmax.pdf}}\hfill
\subfigure[]{\includegraphics[width=0.325\columnwidth]{rad_ker_d_fn_of_ellmax.pdf}}\hfill
\subfigure[]{\includegraphics[width=0.325\columnwidth]{rad_ker_i_fn_of_ellmax.pdf}}\hfill
\centering
\subfigure[]{\includegraphics[width=0.325\columnwidth]{rad_ker_fn_of_ellmax_zaldariagga_scaling.pdf}}
\subfigure[]{\includegraphics[width=0.325\columnwidth]{rad_ker_i_fn_of_ellmax_zaldariagga_scaling.pdf}}

\caption{This figures depicts the radial functions for the different kernels and varying band limit set by fixed $\ell_{\rm min}=2$ and varying $\ell_{\rm max}$ as indicated by their legends. The left panel depicts ${_{\mm}f}(\beta,\ell_{\rm min},\ell_{\rm max})$, the middle panels depicts ${}_{\md}f(\beta,\ell_{\rm min},\ell_{\rm max})$ and the right panel depicts ${}_{\mi}f(\beta,\ell_{\rm min},\ell_{\rm max})$ respectively. All the curves are normalized such that their maxima is set to unity. The horizontal solid black line marks the location where the amplitude of the respective kernels fall below 1\% of its maximum. The thin slanted dashed gray lines indicate a power law fit (by eye) to the envelope of the radial functions. The thick black short vertical dashed lines indicate the transition points as predicted by the empirically derived relation for the non-locality parameter $\beta_0={\rm min}(180^\circ,180^\circ \times 22/\ell_{\rm max})$.}
%While the envelopes for function ${_{\mm}f}(\beta)~\&~ {}_{\mi}f(\beta)$ are fit well by the power law $\propto \beta^{-1.5}$, the envelope for the function ${}_{\md}f(\beta)$ is seen to have a slightly steeper slope $\propto \beta^{-1.65}$.}
\label{fig:rad_ker_decay}
\end{figure}
%

The function ${_{\mm}f}$ is the radial part of the kernel that translates the Stokes parameters $Q$/$U$ to scalars $E$/$B$ and vice versa.  It oscillates and as a consequence of the spin-2 symmetry, it vanishes as $\beta \rightarrow 0$ and $\beta \rightarrow \pi$, since the Euler angles $\alpha$, $\gamma$ are not uniquely defined at those separations.
%This nature of ${_{\mm}f}$ is critical to ensure that the derived fields have the necessary spin properties.

The radial part of the kernel that decomposes the  Stokes parameters into parts that correspond to $E$ and $B$ modes  are also  necessarily non-local.  The function ${_{\mi}f}$ is the radial part of the band limited delta function $\mi$.  It expectedly has its maxima at $\beta=0$ and decays with increasing angular separation.  ${_{\md}f}$ has a vanishing value in the region where $\beta \rightarrow 0$ however it does not vanish at $\beta \rightarrow \pi$ as seen in \fig{fig:beta_kernel}.

\paragraph{Band limit dependence.} 
%It is clear from previous discussions that the scalar field $E$ \& $B$ constructed at a location depends on the Stokes field in the surrounding regions. 
To quantify this non-locality, we study the radial extent of the kernels and its dependence on the maximum multipole accessible. We evaluate the radial functions for different values of $\ell_{\rm max}$, while keeping the lowest multipole fixed at $\ell_{\rm min}=2$. 

The resultant set of radial function are depicted in \fig{fig:rad_ker_decay}. While the amplitude of these radial function scales up as $\propto \ell_{\rm max}^2$, for clarity in the plot their global maxima are normalized to unity.  This normalization highlights the key feature, that on increasing $\ell_{\rm max}$ the radial kernels shift left, attaining their global maxima at progressively small angular distances $\beta$.  At intermediate values of $\beta$, the envelope of the radial functions is fit well by a power law $ \propto \beta^{-n}$ as seen in \fig{fig:rad_ker_decay}.
In fact these finding are neatly summarized in the observation that the radial functions computed by evaluating the multipole sums to different maximum multipoles are approximately self-similar and follow this telescoping and scaling property: $${}_rf(\beta,2,\ell_{\rm max}) \approx \Big[\frac{\ell_{\rm max}}{\ell'_{\rm max}}\Big]^2{}_rf(\beta'=\frac{\ell'_{\rm max}}{\ell_{\rm max}} \beta ,2,\ell'_{\rm max}) \,,$$ where ${}_rf , r \in [\mm, \md,\mi]$ denotes all the different radial functions. \revisit{We can now understand the amplitude scaling and the shifting left of the radial kernels on increasing the maximum multipole by studying this telescoping property. Specifically the global maxima of the kernel with maximum multipole $\ell_{max}$ is amplified/supressed by a factor $(\ell_{\rm max}/\ell'_{\rm max})^2$ and the oscillating part is compressed/expanded by a factor $(\ell'_{\rm max}/\ell_{\rm max})^{-1}$ with respect to the kernel defined by the maximum multipole $\ell'_{\rm max}$.}

To quantify the non-locality of the scalar modes $E/B$, we can define a characteristic angular radius of the region from which the kernels get most of their contribution.   We define a non-locality parameter $\beta_{0}$ as the angular distance beyond which the function ${_{\mm}f}(\beta,\ell_{\rm min}=2,\ell_{\rm max})$ transitions to being consistently below 1\% of its maximum.
%For $\ell_{\rm max}=24$, the maximum multipole accessible on a Nside=8 Healpix map, the non-locality parameter $\beta_0=180^{\circ}$ as the radial function never falls monotonously below 1\% of its global maxima as seen in \fig{}. Using this fact and the self similar property of the radial functions, we define the following empirical relation: $\beta_o= {\rm min}(180,180 \frac{24}{\ell_{\rm max}})$, as a means of estimating the non-locality parameter given the maximum multipole $\ell_{\rm max}$ accessible for analysis.
The empirical relation:
\beq
\beta_0= {\rm min}\left(180^\circ,180^\circ \frac{\ell_{0}}{\ell_{\rm max}} \right) \,,
\eeq
with $\ell_{0}=22$ provides a reasonable estimate of this transition point for ${}_{\mm}f$ as seen in \fig{fig:rad_ker_decay}. Setting $\ell_{0}=10$ and $\ell_{0}=32$ predicts the transition points for the functions ${}_{\mi}f$ and  ${}_{\md}f$ respectively.%\comment{How sensitive is this transition point to $\ell{\rm min}$?} 

In the flat sky limit it has been argued that form of the radial function has to be ${}_\mm f(\beta)\simeq\beta^{-2}$ so as to ensures that the Fourier modes for Stokes parameters $Q/U$ are related  to those of $E/B$ modes merely by rotations and with no scale dependent factor \cite{Zaldarriaga2001a}. Here we numerically studied the formally derived radially function ${}_\mm f(\beta)$ which reveals an oscillatory nature whose amplitude decays on increasing angular distance. \revisit{While we find that the the positive envelope of the function scales as $\beta^{-1.55}$, the radial function resulting from taking a moving average of the radial function does show a scaling of $\beta^{-2}$ at intermediate angular separations. Hence we conclude that the form of the radial function argued in \cite{Zaldarriaga2001a} corresponds to the sliding averaged full sky radial function.}
\comment{K: What do we get in a rolling window of size $\beta_0/5$ or something to average over a few oscillations?}


%--------------------------------------------------------
%--------------------------------------------------------
\section{Generalized operators}
The azimuthal of the convolution kernel $\mathcal{M}$ could have been been argued to have the form $e^{-i2 \alpha}$ by requiring to construct a spin-0 field given some spin-2 fields. In this sense there is no freedom in the choice of the azimuthal dependence of the convolution kernels. The radial part of this kernel however is determined by the basis functions. It is possible to generalize these convolution kernels by choosing alternate forms for the radial functions, without affecting the parity properties of the scalar fields $E$ \& $B$.

We can characterize different forms of the radial kernel by introducing the following harmonic space operator,
%
\beq
\tilde{\mathcal{G}} = {\begin{bmatrix} g_{\ell}^E & 0  \\  0 & g_{\ell}^B \end{bmatrix}} \,,
\eeq
%
where the functions $g_{\ell}^E$ and $g_{\ell}^B$ represent the harmonic representation of the modified radial functions and can in the most general case be chosen to be different for $E$ and $B$ modes. To simplify the discussion and without loosing generality we proceed by setting $g_{\ell}^E = g_{\ell}^B= g_{\ell}$. Once we have made a choice for these harmonic functions, we can define the real space operator $\bar{O}'$ which translates Stokes $Q$ \& $U$ to scalars $E$ \& $B$ and the inverse operator $\bar{O}'^{-1}$ in the following manner,
%
\begin{subequations} \label{eq:gen_qu2eb}
\beqry
{\bar O}' &=& {{}_0\mathcal{Y}} *\tilde T^{-1}*\tilde{\mathcal{G}}* {{}_2\mathcal{Y}^{\dagger}} *\bar T \,,\\
{\bar O}'^{-1}&=& \bar{T}^{-1} *{{}_2\mathcal{Y}}* \tilde{\mathcal{G}}^{-1} *\tilde T *{{}_0\mathcal{Y}^{\dagger}}
\eeqry
\end{subequations}
%
where we have used the primed notation  to distinguish these generalized operators from the default operators defined in \sec{sec:qu2eb} and \sec{sec:eb2qu}. Note that for an arbitrary choice of $\tilde{\mathcal{G}}$ only one of the operators in \eq{eq:gen_qu2eb} is well defined, since $\tilde{\mathcal{G}}^{-1}$ may be ill defined. If we require both the forward and inverse operators to be well defined, then we are constrained in choosing $\tilde{\mathcal{G}}$ such that it has a valid  inverse. The radial part of these generalized convolution kernels is given by the following expressions,
%
\begin{subequations}
\beqry
G_{QU \rightarrow EB}(\beta) &=& G(\beta) = \sum _{\ell=2} ^{\ell_{\rm max}} g_{\ell}\frac{2 \ell+1}{4 \pi} \sqrt{\frac{(\ell-2)!}{(\ell + 2)!}} P_{\ell}^2(\cos{\beta}) \, \label{eq:mod_rad_forward} \\
G_{EB \rightarrow QU}(\beta) &=& G^{-1}(\beta) = \sum _{\ell=2} ^{\ell_{\rm max}} g_{\ell}^{-1}\frac{2 \ell+1}{4 \pi} \sqrt{\frac{(\ell-2)!}{(\ell + 2)!}} P_{\ell}^2(\cos{\beta}) \,,\label{eq:mod_rad_inverse}
\eeqry
\end{subequations}
%
where $g_{\ell}$ are the same multipole function as those appearing in $\tilde{\mathcal{G}}$. Given this general definition for the radial function $G(\beta)$, note that the default radial function ${{}_{\mm}f}$ is just a special case resulting from the choice $\tilde{\mathcal{G}}=\mathbb{1}$ ($g_{\ell}=1$). Note that for this choice of $\tilde{\mathcal{G}}$ the inverse is trivial $\tilde{\mathcal{G}}^{-1}=\tilde{\mathcal{G}}$ and therefore $G^{-1}(\beta) = G(\beta)$.

While defining these generalized operators, it seems more natural to choose the real space function $G(\beta)$ as compared to choosing the multipole function $g_{\ell}$. Using the orthogonality property of associated Legendre polynomials it can be shown that the harmonic function $g_{\ell}$ is given by the following integral over the radial function $G(\beta)$,
%
\beq
g_{\ell} = 2 \pi \sqrt{\frac{(\ell-2)!}{(\ell+2)!}} \int _{0}^{\pi} G(\beta) P_{\ell}^{2}(\cos{\beta}) d\cos{\beta} \,. \label{eq:gb2bl}
\eeq
%
Here it is important to note that the radial function $G(\beta)$ has to be chosen such that it vanishes at $\beta=0$ and $\beta=\pi$. 
%One way to understand this is that the associated Legendre polynomials $P_{\ell}^2 \propto \sin^2{\beta}$ vanish at these values of the abscissa and hence cannot be used to describe functions which don't have this property.  Another way to understand this requirements is that at these locations the coordinate dependence of the Stokes parameters cannot be integrated out, since the azimuthal angle is ill defined and hence the convolution kernel needs to have vanishing contribution from these locations. \revisit{The normalization of these functions is not critically important, since any choice defines a convention. It is important to ensure consistency with the convention once a choice has been made.} \comment{find a better location for this}
Note that in contrast to the radial function $G(\beta)$ an instrumental beam function appropriately normalized has the property $B(\beta) \rightarrow 1$ as $\beta \rightarrow 0$. We clarify that the $B(\beta)$ refers to the effective beam acting to smooth the scalar $E$ \& $B$ mode maps. A circularly symmetric beam $B(\beta)$ defined at the pole can be expressed in the Legendre polynomial $P_{\ell}^0$ basis as follows,
%
\beq
B(\beta) = \sum_{\ell=0}^{\ell_{\rm max}} \frac{2 \ell+1}{4 \pi} b_{\ell} P_{\ell}^{0} (\cos{\beta})\,,
\eeq
%
where $b_{\ell}$ denote the coefficients of expansion.
%
\begin{figure}[!t] 
\centering
\subfigure[\label{fig:bl_gbeta}]{\includegraphics[width=0.32\columnwidth]{Gbeta_for_different_gl_lmax1536.pdf}}
\subfigure[\label{fig:glbl}]{\includegraphics[width=0.32\columnwidth]{gl_for_different_gbeta_lmax1536.pdf}}
\subfigure[\label{fig:gl_bbeta}]{\includegraphics[width=0.32\columnwidth]{Bbeta_for_gl_lmax1536.pdf}}
\caption{\textit{Left:} The vertical dashed gray line depicts the approximate pixel size $\Delta_{\rm pix} = \sqrt{\frac{4 \pi}{N_{\rm pix}}}$ of a Nside=512 Healpix map. The green line depicts the default radial kernel $f(\beta)$ defined in \eq{eq:qu2eb_gen_kernel}. The blue and orange lines depict the modified radial function resulting the beam harmonics $b_{\ell}$ corresponding to Gaussian beams with fwhm=15 \& 12 arcminutes respectively. The red curve depicts an example modified radial function: $G(\beta)=\mathcal{N} \beta^n \exp{\left[ -\left( \frac{\beta-\beta0}{\sqrt{2} \sigma} \right)^s \right]}$ with parameters set to the following values $[n=1;\, \beta_0=0 ;\, \sigma = 2\Delta_{\rm pix} ;\, s=1.5]$. The black dashed curve depicts the band limited reconstruction of the modified radial function $G(\beta)$. We intentionally have plotted $\beta G(\beta)$ to clearly depict the high $\beta$ behavior of these functions. \textit{Middle: } This figure depicts the harmonic representation of the respective radial functions as indicated by the legend. The dashed curves of the corresponding color depict the inverse of the harmonic functions. \textit{Right:} This figure depicts the beam function $B(\beta)$ evaluated from interpreting the respective harmonic functions as those corresponding to an instrument beam.}
\label{fig:example_gbeta}
\end{figure}
%

Though the real space behavior of these two function $G(\beta)$ and $B(\beta)$ has important differences, in harmonic space they play identical roles. Therefore it is possible to interpret the beam harmonic coefficients as those representing some modified radial kernel. \fig{fig:glbl} depicts the harmonic functions $g_{\ell} (b_{\ell})$ for the respective radial kernel and beams.  The modified radial kernel resulting from Gaussian beams with ${\rm fwhm} =15' \,\&\, 12'$ are depicted in \fig{fig:bl_gbeta} as blue and orange curves respectively. Note that instruments  beams tend to increase the non-locality parameter $\beta_0$, indicated by the shifting right of the maxima of the respective kernels, as one may have expected. The red curve depicts a modified radial kernel which by construction has a very small $\beta_0$.  Similarly it is possible to interpret the harmonic representation $g_{\ell}$ of the radial function $G(\beta)$ as those corresponding to some instrument beam function. The beam function corresponding to the default radial kernel ($g_{\ell}=1$) is merely a band limited representation of the delta function depicted by the green curve \fig{fig:gl_bbeta}, while the red curve depicts the same for the modified radial kernel.

%In \sec{sec:local_conv_eb} we constructed localized convolution kernels by multiplying $R(\beta)$ with an apodized version of the step function $\theta_{\rm apo}(r_{\rm cutoff})$. The oscillation seen in the spectra in \fig{fig:eb-spectra_rad_cutoff} can be explained to be due to this effective beam characterized by $g_{\ell}^2$ operating on the power spectra. The effective beam can be evaluated by computing the multipole function $g_{\ell}$ as follows,
%%
%\beq
%g_{\ell} = 2 \pi \sqrt{\frac{(\ell-2)!}{(\ell+2)!}} \int _{0}^{r_{\rm cutoff}} R(\beta) \theta_{\rm apo}(r_{\rm cutoff})  P_{\ell}^{2}(\cos{\beta}) d\cos{\beta} \,, \label{eq:gb2bl} \,,
%\eeq
%%
%where the upper limit of the integration is set to $r_{\rm cutoff}$ since the function $\theta_{\rm apo}(r_{\rm cutoff})$ vanishes for $\beta>r_{\rm cutoff}$. The function $b^2_{\ell}-1$ matches the oscillation seen in \fig{fig:eb-spectra_rad_cutoff} as  seen in \fig{fig:match_cl_oscillations} where the two results have been over plotted.
%%
%\begin{figure}[!t] 
%\centering
%\subfigure[]{\includegraphics[width=0.98\columnwidth]{analytical_cl_oscillations_vs_data.pdf}}
%\caption{The thin lines depicts the same spectral differences as those seen in \fig{fig:eb-spectra_rad_cutoff}, while the thick lines of the corresponding color depict the function $g_{\ell}^2 -1$ as derived from evaluating \eq{eq:gb2bl} for different $r_{\rm cutoff}$}.
%\label{fig:match_cl_oscillations}
%\end{figure}
%%
%The apodized step function in this case transition from 1 at $\beta < r_{\rm cutoff} -w$ to 0 at $r_{\rm cutoff}$ over a width $w= 3^{\circ}$ with a cosine squared profile .

\subsection{Recovering the default $E$ and $B$ mode spectra}
The generalized convolution kernels defined in the previous section, when operated on the Stokes vector returns some scalar $E'$ and $B'$ mode maps,
%
\beq
\bar{S}' = \bar{O}' * \bar{P}
\eeq
%
which are not the standard $E$ and $B$ modes maps. Since the the harmonic representation $g_{\ell}$ of the radial function $G(\beta)$ can be simply interpreted as the harmonic coefficients of some beam,  the spectra of the scalar fields $E'$ and $B'$ are related to the spectra of the standard $E$ and $B$ fields via the following relation, 
 %
 \begin{subequations}
 \beqry
C_{\ell}^{EE,BB,EB} &= &C_{\ell}^{E'E',B'B',E'B'} /   g_{\ell}^2\,,\\
C_{\ell}^{TE,TB}  &=&  C_{\ell}^{TE',TB'} / g_{\ell}\,,
 \eeqry
 \end{subequations}
 %
 where $C_{\ell}$ denotes the angular power spectra and $T$ refers to the temperature anisotropy map. Therefore the standard $E$ and $B$ mode spectra can be recovered from the modified fields $E'$ and $B'$ and their accurate recovery only relies on the inverse of the harmonic functions $1/g_{\ell}$ being well behaved, which can be ensured by making a suitable choice for the radial function $G(\beta)$.
%One direct application of this freedom in choosing the radial kernel is that one can mitigate the issue of leakage of power from $E$ to $B$.  By constructing a radial function which tapers to zero at sufficiently small radii one can clearly discard pixels which are affected by masking. 
%--------------------------------------------------------
%--------------------------------------------------------

\section{Discussion}\label{sec:discussion}

\begin{itemize}
\item Summarize what has been achieved in this peice of work.
\item Whats new as compared to work done by Zaldariaga?
\item Not having to compute spin spherical harmonics.
\item Potentially allows for TOD to E/B map generation.
\item Compact kernels and understanding of why mask apodization works.
\item Can you deconvolve anisotropic compact kernels? Meaning different radial cut-off in different portions of the sky. This will be useful to dealing with foregrounds. Its not obvious if this can be achieved in harmonic space.  
\item The relation to galaxy shear E/B estimates.
\item Understanding E/B signatures of magentized filaments.
\end{itemize}

\revisit{A similar equation for real space $E$ \& $B$ operators was derived in \cite{Zaldarriaga2001a}, however those results were derived for the flat sky case and did not explicitly derive the radial kernel.} \comment{A discussion on this should be in the conclusions.}

In this work we present real space operators on the sphere that transform maps of Stoke Q/U parameters to scalars E/B. We  also present real space operators that decompose maps of Stokes Q/U parameters into the parts that only contribute to E/B modes. We introduced a vector-matrix notation which allows for concise book keeping of all the operations involved and simplifies the derivation of the real space operators. 

These real space operators provide a spatially intuitive way of understanding construction of the scalar modes. We explicitly  demonstrated that all the operators can be separated into a band limit independent azimuthal operation and band limit dependent radial weights. The azimuthal part of the operator is primarily responsible for the requisite spinorial decomposition, while the radial weights determine the non-local dependence of the construction of the resultant fields. We define the $\beta_0$ parameter using an empirical relation to characterize the non-locality and demonstrate that it scales $\propto \ell_{\rm max}^{-1}$.

Our careful study of the real space operators reveals the dual interpretation of the operators as either the Greens function or a convolving beam depending on whether the kernels are expressed in terms of the forward rotation Euler angles or the inverse rotation Euler angles. While the convolution interpretation is familiar to all, the Green's function interpretation is a new one arising from this work. In particular the Green's function interpretation allows us to think of ${}_{+2}X$ as some spin-2 charge which radiates out a complex spin-0 scalar field $E+iB$. The resultant E/B maps can be then understood as arising from superposition of the radiating field from all the spin charges on the sphere.

 We explicitly demonstrated that this real space operation can be simply interpreted as a convolution over the complex field $[Q - i U]$ (or $[E + iB]$) with an effective complex beam which is fully expressed in terms of the $Y_{\ell 2}$ spherical harmonic functions. We also use this vector matrix notation to derive real space operators which allow the direct decomposition of the full Stokes vector \vp{} into the vector \vp{E} and \vp{B} that correspond to the respective scalar modes. 



Finally we present the generalized real space operators $\bar{O}'$, which are derived by allowing the radial function to vary from its default form. We derive constraints on the modifications to these radial function by demanding the inverse operator to be well defined. We argue that these modifications to the radial kernel can be be interpreted as a some smoothing smoothing operation on the scalar fields with a circularly symmetric instrument beam. We also show that as long as these radial function are invertible, the standard spectra can always be recovered from these modified $E'$ \& $B'$ maps. The main advantage of modifying these radial function is the ability to generate more locally defined $E$ and $B$ mode maps. This could potentially be useful in reducing foreground contamination on large angular scales in a full sky $E/B$ analysis. Also defining more locally constructed scalar fields $E$ \& $B$ can be used to circumvent the power leakage nuisance. We explore and demonstrate the working of these ideas in the next paper in this series. 
%\revisit{The discussion till now gives the impression that using the localized convolution kernels is no different from from using the default kernel and altering the spherical harmonic coefficients of expansion of the relavant fields by appropriately operating on them with the  effective beam functions $g_{\ell}$. To appreciate the difference between these two, it is important to realize that in general one can make a choice of a radial function which may not have a band limited description. In such a case these two method of evaluating the relevant fields is not identical. An example of this claim is depicted in \fig{fig:example_gbeta}.\\
%Another important thing to realize is that the harmonic coefficients derived from default full sky operations get some contributions from different portions of sky. For instance evaluating the E and B fields in the vicinity of the poles is are prone to receiving significant contributions from strong foregrounds near the equator. Correcting the harmonic coefficients of expansion with the effective beam function does not cancel these non-local contribution. On the contrary by performing the convolution with the localized real space kernels, the regions which contribute to the local field evaluations are predetermined by the choice of the radial function.}
 
 
\section{Intuition for the polarization structure of magnetized filaments}
\begin{figure}
  \includegraphics[width=0.5\columnwidth]{line.pdf}
  \includegraphics[width=0.5\columnwidth]{spiral.pdf}
  \caption{
    The polarization signals of toy filament structures.
    In a filament organized perfectly along a magnetic field line, the polarization will be perpendicular to the filament direction.  The $E/B$ modes of filaments are in some ways easier to think about than the Stokes parameters.
    Left panels: in a straight filament, the E-mode is positive along the filament and at the ends, but negative along the sides.  B-modes are only non-zero at the ends.  Right panels: in a curved filament, the E-mode is again positive along the filament.  Outside the filament, the $E$-mode is more negative on the interior of the curve than the exterior.  The $B$-modes are again non-zero only at the ends, and are akin to the straight filament case.
    In all images, the longitude increases to the left (sky convention).}
  \label{fig:polfilaments}
\end{figure}

Naturally describes the reason for a positive T/E correlation.  Figure~\ref{fig:polfilaments}.

 

\section{Appendix}

%--------------------------------------------------------
%--------------------------------------------------------
\subsection{ Mathematical properties of spin spherical harmonics}\label{sec:ylm_mathprop}
The sum over $m$ index of product of two spherical harmonic functions of spin $s_1$ and $s_2$ respectively, is given by the following expression \cite{varshalovich},
\beq \label{eq:sum_spin_shf}
 \sum_{m}{_{s_1}Y^*_{\ell m}}(\hat{n}_i){_{s_2}Y_{\ell m}}(\hat{n}_j) = \sqrt{\frac{2\ell+1}{4 \pi}} _{s_2}Y^{-s_1}_{\ell}(\beta,\alpha) e^{- i s_2 \gamma} \,,
\eeq
where $\alpha, ~\beta ~\&~ \gamma$ correspond to the Euler angles that specify the rotation matrix which transforms the local cartesian coordinates defined at $\hat{n}_i$ such that it aligns with the local cartesian coordinate system at $\hat{n}_j$.

The spin spherical harmonics satisfy the following orthogonality relations,
%
\beq
\int  {_sY_{\ell m}}(\hat{n}){_sY^*_{\ell' m'}}(\hat{n}) d\Omega = \delta_{\ell \ell'} \delta_{\rm m m'} \,, \label{eq:ylmortho1}
\eeq
%
where $s$ denotes the spin of the spherical harmonic coefficients. The numerical validity of \eq{eq:ylmortho1} is only limited by the rate at which these functions are sampled on the sphere and hence this identity can be made arbitrarily accurate by choosing a sufficiently high sampling rate.
%While working with CMB polarization one is often dealing with spin-2 spherical harmonics. Here we derive some relation which we use while evaluating the real space projection operators,
%\beq
%\sum_{\ell m} {_2Y}_{\ell m}(\hat n_i) {_2Y}^*_{\ell m}(\hat n_j) = 
%\eeq

The spin spherical harmonic functions satisfy the following completeness relation,
%
\beq
\sum_{\ell m}{_sY_{\ell m}}(\hat{n}_i){_sY^*_{\ell m}}(\hat{n}_j) = \delta(\hat{n}_i - \hat{n}_j) \label{eq:ylm_prop1} \,,
\eeq
%
Note that the numerical validity of \eq{eq:ylmortho2} is strictly true only when the sums over the indices $(\ell, m)$ run to infinity. This is never true in practice, since the measured data invariable are band limited owing to the finite resolution of the experiments. Hence this relation is only approximately true and in more realistic scenario takes up the following function form,
%
\beqry
\sum_{\ell=\ell_{\rm min}, m}^{\ell_{\rm max}}{_{s_1}Y^*_{\ell m}}(\hat{n}_i){_{s_2}Y_{\ell m}}(\hat{n}_j) &\approx& \delta(\hat{n}_i - \hat{n}_j) \label{eq:ylmortho2} \,, \\
&=& \sum_{\ell=\ell_{\rm min}}^{\ell_{\rm max}} \sqrt{\frac{2 \ell+1}{4 \pi}} {}_{s_2}Y^{-s_1}_{\ell}(\beta,\alpha) e^{-i s_2 \gamma} \,, \nonumber
\eeqry
%
where $\alpha, \beta ~\&~ \gamma$ are the Euler angles relating the two directions $\hat{n}_i$ and $\hat{n}_j$. A specific case of this function with $(s=2, \ell_{\rm min}=2, \ell_{\rm max}=96)$ is depicted in the last two columns of \fig{fig:vis_kernel}.

%--------------------------------------------------------
\subsection{Local convolution kernels using standing Healpix routines}\label{sec:rot_ker_healpix}
Healpix uses Euler angles given in the z-y-z convention in contrast to the Euler angle definitions given in \sec{sec:euler} which are in z-y1-z2 convention. The two set of Euler angles are related by the following relations $(\alpha,\beta,\gamma)_{\rm z-y-z} = (\gamma,\beta,\alpha)_{\rm z-y1-z2}$ \cite{varshalovich}. We know the exact form of the harmonic coefficients for the kernels when they are centered around the north pole. Therefore the kernels centered around any arbitrary Healpix pixel $\hat{n}_i$ can be calculated by simply rotating these harmonic coefficients by the Euler angles $(0,\theta_i,\phi_i)_{\rm z-y-z}$ and synthesizing the kernel by evaluating the harmonic sum.
\endgroup

\bibliographystyle{JHEP}
\bibliography{ref}

\end{document}
