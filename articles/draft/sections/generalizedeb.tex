%--------------------------------------------------------
%--------------------------------------------------------
\comment{How does the radial kernel reduce to unity on evaluating the the operator on to its inverse ? This will be important to understand how to define alternate radial functions.}
\section{Generalized operators}
The azimuthal dependence of the convolution kernels which translate the Stokes parameters Q \& U to the scalar E \& B is determined by the spin properties of the field being operated upon and the spin of the resultant fields. Hence there is no freedom in the choice of function for the azimuthal dependence of the convolution kernels. The radial part of the function however is determined by the choice of the basis functions.  It is possible to generalize these convolution kernels by choosing alternate forms for the radial functions.

We can characterize different forms of the radial kernel by introducing the following harmonic space operator,
%
\beq
\tilde{\mathcal{G}} = {\begin{bmatrix} g_{\ell}^E & 0  \\  0 & g_{\ell}^B \end{bmatrix}} \,,
\eeq
%
where the functions $g_{\ell}^E$ and $g_{\ell}^B$ represent the harmonic representation of the modified radial functions and can in the most general case be chosen to be different for E and B modes. To simplify the discussion and without loosing this generality we proceed with the assumption $g_{\ell}^E = g_{\ell}^B= g_{\ell}$. We can chose this function to be any arbitrary function and it will allow us to define some convolution operator which either translates Stokes Q \& U parameters to scalars E \& B or vice verse. Given $\tilde{\mathcal{G}}$ the modified forward and inverse convolution kernels are given by the following expressions,
%
\begin{subequations} \label{eq:gen_qu2eb}
\beqry
{\bar O}' &=& {{}_0\mathcal{Y}} *\tilde T^{-1}*\tilde{\mathcal{G}}* {{}_2\mathcal{Y}^{\dagger}} *\bar T \,,\\
{\bar O}'^{-1}&=& \bar{T}^{-1} *{{}_2\mathcal{Y}}* \tilde{\mathcal{G}}^{-1} *\tilde T *{{}_0\mathcal{Y}^{\dagger}}
\eeqry
\end{subequations}
%
where we have used the primed notation  to distinguish these generalized operators from the default operators defined in \sec{sec:qu2eb} and \sec{sec:eb2qu}. Note that for an arbitrary choice of $\tilde{\mathcal{G}}$ only one of the operators in \eq{eq:gen_qu2eb} is well defined, since $\tilde{\mathcal{G}}^{-1}$ may be ill defined. If we require both the forward and inverse hold true, then we are constrained in choosing $\tilde{\mathcal{G}}$ such that its inverse is well defined.

The radial part of these generalized convolution kernels is now given by the following expressions,
%
\begin{subequations}
\beqry
G_{QU \rightarrow EB}(\beta) &=& G(\beta) = \sum _{\ell=\ell_{\rm min}} ^{\ell_{\rm max}} g_{\ell}\frac{2 \ell+1}{4 \pi} \sqrt{\frac{(\ell-2)!}{(\ell + 2)!}} P_{\ell}^2(\cos{\beta}) \, \label{eq:mod_rad_forward} \\
G_{EB \rightarrow QU}(\beta) &=& G^{-1}(\beta) = \sum _{\ell=\ell_{\rm min}} ^{\ell_{\rm max}} g_{\ell}^{-1}\frac{2 \ell+1}{4 \pi} \sqrt{\frac{(\ell-2)!}{(\ell + 2)!}} P_{\ell}^2(\cos{\beta}) \,,\label{eq:mod_rad_inverse}
\eeqry
\end{subequations}
%
where $g_{\ell}$ are the same multipole function as those appearing in $\tilde{\mathcal{G}}$. If one chooses different forms of $g_{\ell}$ for E and B modes then the radial function are defined accordingly. Given this general definition for the radial function $G(\beta)$, note that the default radial function $f(\beta)$ is just a special case resulting from the choice $\tilde{\mathcal{G}}=\tilde{\mathcal{I}}$ ($g_{\ell}=1$). Note that for this choice of $\tilde{\mathcal{G}}$ the inverse is trivial $\tilde{\mathcal{G}}^{-1}=\tilde{\mathcal{G}}$ and therefore $G^{-1}(\beta) = G(\beta)$.

While defining these alternate operators by modifying the radial part of the kernels, it seems more natural to make a choice on the real space function $G(\beta)$ as compared to choosing the multipole function $g_{\ell}$. Using the orthogonality property of associated Legendre polynomials it can be shown that the multipole function $g_{\ell}$ is given by the following integral once the radial function $G(\beta)$ is specified,
%
\beq
g_{\ell} = 2 \pi \sqrt{\frac{(\ell-2)!}{(\ell+2)!}} \int _{0}^{\pi} G(\beta) P_{\ell}^{2}(\cos{\beta}) d\cos{\beta} \,. \label{eq:gb2bl}
\eeq
%
Here it is important to note that the radial function has to be chosen such that $G(\beta)$ has to be such that it vanishes at $\beta=0$ and $\beta=\pi$. One way to understand this is that the associated Legendre polynomials $P_{\ell}^2$ vanish at these values of the abscissa and hence cannot be used to describe function which don't have this property.  Another way to understand this requirements is that at these locations the coordinate dependence of the Stokes parameters cannot be integrated out, since the azimuthal angle is ill defined and hence the convolution kernel needs to have vanishing contribution from these locations. \revisit{The normalization of these functions is not important, since that just defines a convention. One just needs to ensure to be consistent with the convention once a choice has been made.} \comment{find a better location for this}

In contrast,  the beam convolution function $B(\beta) \rightarrow 1$ as $\beta \rightarrow 0$ and a circularly symmetric beam's harmonic representation $b_{\ell}$ are the coefficients of expansion of $B(\beta)$ in the Legendre polynomial $P_{\ell}^0$ basis. \comment{Distinguishing E/B localization characterized by $g_{\ell}$ from E/B smoothing characterized by $b_{\ell}$.}

The discussion till now gives the impression that using the localized convolution kernels is no different from from using the default kernel and altering the spherical harmonic coefficients of expansion of the relavant fields by appropriately operating on them with the  effective beam functions $g_{\ell}$. To appreciate the difference between these two, it is important to realize that in general one can make a choice of a radial function which may not have a band limited description. In such a case these two method of evaluating the relevant fields is not identical. An example of this claim is depicted in \fig{fig:example_gbeta}.  
%
\begin{figure}[!t] 
\centering
\subfigure[]{\includegraphics[width=0.49\columnwidth]{example_gbeta.pdf}}
\subfigure[]{\includegraphics[width=0.49\columnwidth]{example_eff_bl.pdf}}
\caption{}
\label{fig:example_gbeta}
\end{figure}
%

Another important thing to realize is that the harmonic coefficients derived from default full sky operations get some contributions from different portions of sky. For instance evaluating the E and B fields in the vicinity of the poles is are prone to receiving significant contributions from strong foregrounds near the equator. Correcting the harmonic coefficients of expansion with the effective beam function does not cancel these non-local contribution. On the contrary by performing the convolution with the localized real space kernels, the regions which contribute to the local field evaluations are predetermined by the choice of the radial function.


%In \sec{sec:local_conv_eb} we constructed localized convolution kernels by multiplying $R(\beta)$ with an apodized version of the step function $\theta_{\rm apo}(r_{\rm cutoff})$. The oscillation seen in the spectra in \fig{fig:eb-spectra_rad_cutoff} can be explained to be due to this effective beam characterized by $g_{\ell}^2$ operating on the power spectra. The effective beam can be evaluated by computing the multipole function $g_{\ell}$ as follows,
%%
%\beq
%g_{\ell} = 2 \pi \sqrt{\frac{(\ell-2)!}{(\ell+2)!}} \int _{0}^{r_{\rm cutoff}} R(\beta) \theta_{\rm apo}(r_{\rm cutoff})  P_{\ell}^{2}(\cos{\beta}) d\cos{\beta} \,, \label{eq:gb2bl} \,,
%\eeq
%%
%where the upper limit of the integration is set to $r_{\rm cutoff}$ since the function $\theta_{\rm apo}(r_{\rm cutoff})$ vanishes for $\beta>r_{\rm cutoff}$. The function $b^2_{\ell}-1$ matches the oscillation seen in \fig{fig:eb-spectra_rad_cutoff} as  seen in \fig{fig:match_cl_oscillations} where the two results have been over plotted.
%%
%\begin{figure}[!t] 
%\centering
%\subfigure[]{\includegraphics[width=0.98\columnwidth]{analytical_cl_oscillations_vs_data.pdf}}
%\caption{The thin lines depicts the same spectral differences as those seen in \fig{fig:eb-spectra_rad_cutoff}, while the thick lines of the corresponding color depict the function $g_{\ell}^2 -1$ as derived from evaluating \eq{eq:gb2bl} for different $r_{\rm cutoff}$}.
%\label{fig:match_cl_oscillations}
%\end{figure}
%%
%The apodized step function in this case transition from 1 at $\beta < r_{\rm cutoff} -w$ to 0 at $r_{\rm cutoff}$ over a width $w= 3^{\circ}$ with a cosine squared profile .

\subsection{Recovering the default E and B mode spectra}
The convolution kernels defined using these modified radial functions returns some E' and B' mode maps,
%
\beq
\bar{S}' = \bar{O}' * \bar{P}
\eeq
%
 which are not the standard $E$ and $B$ modes maps. We would now like to relate the spectra of these modified $E'$ and $B'$ mode maps to the standard spectra. Even though the $G(\beta)$ maybe such that it doesn't have an accurate band limited  description, the spectra are always evaluated to a definite band limit. Therefore the harmonic representation $g_{\ell}$ of the radial function $G(\beta)$ can be used to appropriately normalize the spectra so as to match the standard E and B mode spectra.
 Specifically it can be shown that the spectra of $E'$ and $B'$ field is related to the spectra of the spectra of the standard E and B fields via the following simple relation,
 %
 \beq
 C_{\ell}^{E',B'} =  C_{\ell}^{E,B} g_{\ell}^2\,,
 \eeq
 %
%--------------------------------------------------------
%--------------------------------------------------------
