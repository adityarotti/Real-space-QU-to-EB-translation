%--------------------------------------------------------
%--------------------------------------------------------
\section{E/B description of CMB polarization} \label{sec:pol-intro}
%--------------------------------------------------------
\subsection{Polarization primer}\label{sec:pol-primer}
The CMB polarization is measured in terms of Stokes Q and U parameters. These measurements can be combined to form the complex spin 2 polarization field as follows,
%
\beqry \label{eq:spin-pol}
_{\pm 2}\bar{X}(\hat{n}) &=& Q(\hat{n}) \pm i U (\hat{n}) \nonumber \\ &=& \sum_{\ell m}  {_{\pm 2}} \tilde{X}_{\ell m}  {_{\pm 2}}Y_{\ell m} (\hat{n}) \,.
\eeqry
%
Since these measured quantities depend on the local coordinate system, it is cumbersome to work with them. To overcome this, one describes the CMB polarization field in terms of a scalar field denoted by $E(\hat{n}) $ and a pseudo scalar field $B(\hat{n}) $ \cite{Kamionkowski1997}. These scalar fields are related to the spin-2 polarization field $_{\pm 2}X(\hat{n})$ via the following relations,
%
\beq \label{eq:ebdef}
\mathcal{E}(\hat{n}) = -\frac{1}{2} \big[ \bar{\eth}^2 _{+ 2}X(\hat{n})  +  \eth^2 _{- 2}X(\hat{n}) \big] ~\,;~\mathcal{B}(\hat{n}) = -\frac{1}{2i} \big[ \bar{\eth}^2 _{+ 2}X(\hat{n})  -  \eth^2 _{- 2}X(\hat{n}) \big] \,,
\eeq
%
where $\eth$ and $\bar{\eth}$ denote the spin raising and lowering operators respectively. These $E$ and $B$ fields are spin-0 fields similar to the temperature anisotropies and hence their value are independent of the coordinate system definitions (except that the B-modes have an odd parity, meaning that they change sign under reflection $\hat{n} \rightarrow -\hat{n}$.). The spin raising and lowering operators have the following properties \cite{goldberg67},
%
\beqrys \label{eq:spinopylm} 
\eth _s Y_{lm}(\hat{n}) &=& \sqrt{(\ell-s)(\ell+s+1)} _{s+1} Y_{lm}(\hat{n}) \,, \\
\bar{\eth} _s Y_{lm}(\hat{n}) &=& -\sqrt{(\ell+s)(\ell-s+1)} _{s-1} Y_{lm}(\hat{n}) \,, 
\eeqrys
%
where $_s Y_{lm}(\hat{n}) $ denote the spin-s spherical harmonics.

Using \eq{eq:ebdef} and the properties of the spin raising and lowering operators given in \eq{eq:spinopylm} it can be shown that the scalar fields $\mathcal{E}/\mathcal{B}$ are defined via the following set of equations,
%
\beq \label{eq:pseudo}
\mathcal{E}(\hat{n}) = \sum_{\ell m} a^{E}_{\ell m} \sqrt{\frac{(\ell+2)!}{(\ell-2)!}} Y_{\ell m} (\hat{n}) ~\,;~ \mathcal{B}(\hat{n})  =\sum_{\ell m} a^{B}_{\ell m} \sqrt{\frac{(\ell+2)!}{(\ell-2)!}} Y_{\ell m} (\hat{n}) \,,
\eeq
%
where the harmonic coefficients of  $\mathcal{E}/\mathcal{B}$ fields are related to the harmonic coefficients of the spin-2 polarization field via the following equations,
%
\beq\label{eq:x2eb}
a^{E}_{\ell m} = -\frac{1}{2} \Big[ {}_{+2}X_{\ell m} + _{-2}X_{\ell m} \Big] ~\,;~a^{B}_{\ell m} = -\frac{1}{2i} \Big[ {}_{+2}X_{\ell m} - _{-2}X_{\ell m} \Big] 
\eeq
%
In the remainder of this article, we will work with the scalar $E$ and pseudo scalar $B$ fields as defined by the following expressions, 
%
\beq \label{eq:realeb}
E(\hat{n}) = \sum_{\ell m} a^{E}_{\ell m} Y_{\ell m} (\hat{n}) ~\,;~ B(\hat{n})  =\sum_{\ell m} a^{B}_{\ell m} Y_{\ell m} (\hat{n}) \,.
\eeq
%
Note that the $[E,B]$ fields are merely filtered versions of the fields $[\mathcal{E},\mathcal{B}]$, as their spherical harmonic coefficients of expansion differ by the factor of $\sqrt{\frac{(\ell+2)!}{(\ell-2)!}}$. We make this choice since the CMB spectra are more closely related to the fields E \& B.
%--------------------------------------------------------
%--------------------------------------------------------
\subsection{Matrix notation} \label{sec:mat_pol_intro}
In this section we cast the relation introduced in Sec.~\ref{sec:pol-primer} in matrix notation\footnote{While we work with the matrix and vector sizes given in terms of some pixelization parameter $\rm N_{\rm pix}$, all the relations are equally valid in the continuum limit attained by allowing $\rm N_{\rm pix}\rightarrow \infty$}. This representation will make transparent the derivation of the real space operators we discuss in the following sections. We adopt a convention in which real space quantities are denoted by bar-ed variable while those in harmonic space are denoted by tilde-ed variables.\\
We begin by introducing the matrices encoding the spin spherical harmonic basis vectors,
%
\beq
{}_{|s|}\mathcal{Y}= \bmat _{+s}Y & 0 \\ 0 & _{-s}Y \emat _{2 \rm N_{\rm pix} \times 2 \rm N_{\rm alms}} \,,
\eeq
%
where $s$ denotes the spin of the basis functions. For this work we will only be working with cases $s \in [0,2]$. In this notation, each column can be mapped to a specific harmonic basis function marked by the pair of indices:$(\ell,m)$ and each row maps to a specific position on the sphere. Note that this matrix is in general not a square matrix. The number of rows is determined by the scheme used to discretely represent the sphere and the number of columns is set by the number of basis functions of interest (often determined by the band limit).

We now define the different polarization data vectors and their representation in real and harmonic space as follows,
%
\beqrys
\bar{S} &=& \bmat E \\ B  \emat_{2 \rm N_{\rm pix} \times 1} ~~~~;~~ \bar{X} = \bmat _{+2}X \\ _{-2}X \emat_{2 \rm N_{\rm pix} \times 1} ~~;~~\bar{P} =\fqu_{\tiny {2 \rm N_{\rm pix} \times 1}} \,, \\
\tilde{S} &=& \bmat a^{E} \\ a^{B} \emat _{2 \rm N_{\rm alms} \times 1}  ~~; ~~ \tilde{X} = \bmat _{+2} \tilde{X} \\ _{-2} \tilde{X} \emat_{2 \rm N_{\rm alms} \times 1} \,.
\eeqrys
%
The different symbols have the same meaning as that discussed in \sec{sec:pol-primer}, except that the subscript $_{\ell m}$ for the spherical harmonic coefficients of expansion is suppressed if favor of cleaner notation.

Next we define the operators which govern the transformations between different representations of the polarization field as follows,
%
\beqrys
\bar T &=& \qutox_{2 \rm N_{\rm pix} \times 2 \rm N_{\rm pix}} ~~;~~ \bar T^{-1} = \frac{1}{2} \bar T^{\dagger} \,, \\
\tilde T &=& -\qutox_{2 \rm N_{\rm alms} \times 2 \rm N_{\rm alms}} ~~;~~ \tilde T^{-1} = \frac{1}{2} \tilde T^{\dagger} \,,
\eeqrys
%
where we have chosen the sign conventions so as to match those used in Healpix.
Using the data vectors and the matrix operators defined above we can now express, in compact notation, the forward and inverse relations between different representations of the polarization data vectors as follows,
%
\begin{subequations} \label{eq:pol_data_relns}
\beqry 
\bar{X} &=& \bar T * \bar{P} ~~;~~\bar{P} = \frac{1}{2} \bar T^{\dagger} * \bar{X} \,, \\
\bar X &=&  {{}_2\mathcal{Y}} * \tilde X  ~~;~~ \tilde X ={{}_2\mathcal{Y}}^{\dagger} * \bar X  \,, \\
\tilde{X} &=& \tilde T * \tilde{S} ~~;~~ \tilde{S} = \frac{1}{2}\tilde T^{\dagger} * \tilde{X} \,.\\ 
\bar S &=&  {{}_0\mathcal{Y}} * \tilde S ~~;~~  \tilde S =  {{}_0\mathcal{Y}}^{\dagger} * \bar S \,.
%\tilde X &=&  {{}_2\mathcal{Y}}^{\dagger} * \bar X ~~;~~ \tilde{X} = \tilde T * \tilde{S} \,, \\
%\bar{S} &=& {{}_0\mathcal{Y}}*\tilde S ~~;~~ \tilde{S} = \frac{1}{2}\tilde T^{\dagger} * \tilde{X} \,.\\
%\bar{X} &=& \bar T * \bar{P} ~~;~~ \tilde{X} = \tilde T * \tilde{S} \,, \\
%\bar{P} &=& \frac{1}{2} \bar T^{\dagger} * \bar{X} ~~;~~ \tilde{S} = \frac{1}{2}\tilde T^{\dagger} * \tilde{X} \,. \\
\eeqry
\end{subequations}
%
Next we introduce the harmonic space operators, which project the harmonic space data vector to E or B subspace,
%
\begin{subequations} \label{eq:har_eb_op}
\beqry
\tilde O_E &=& \bmat \mathbb{1} & \mathbb{0} \\ \mathbb{0} & \mathbb{0} \emat _{2 \rm N_{\rm alms} \times 2 \rm N_{\rm alms} }   ~~;~~ \tilde S_E = \tilde O_E* \tilde S \,,\\
\tilde O_B &=& \bmat \mathbb{0} & \mathbb{0} \\ \mathbb{0} & \mathbb{1} \emat _{2 \rm N_{\rm alms} \times 2 \rm N_{\rm alms} } ~~; ~~ \tilde S_B = \tilde O_B *\tilde S \,
\eeqry
\end{subequations}
%
Note that these harmonic space matrices are idempotent, orthogonal to each other and their sum is an identity matrix as can be explicitly seen via the following relations, 
%
\begin{subequations} \label{eq:har_op_prop}
\beqry
\tilde O_E * \tilde O_E&=& \tilde O_E ~~;~~  \tilde O_B * \tilde O_B= \tilde O_B \,,\\
 \tilde O_E * \tilde O_B&=& \mathbb{0} \,, \label{eq:op_eb_ortho}\\ 
 \tilde O_E + \tilde O_B&=& \mathbb{1} \,.
\eeqry
\end{subequations}
%
Note that the above relations for these harmonic space operators are exactly valid.  In the following sections we aim to derive the real space analogues of these harmonic space operators.
%%%%%%%%%%%%%%%%%%%%%%%%%%%%%%%%%%%
\section{Real space operators} \label{sec:real_space_operators}
In this section we derive the real space operators which translate the Stokes vector \vp{}  to the vector of scalars \vs  and vice versa. We also derive real space operators for direct decomposition of the Stokes vector \vp{} in to a vector \vp{\rm E} that correspond to E-modes and another vector \vp{\rm B} that corresponds to the B-modes of polarization, such that \vp{} = \vp{\rm E} + \vp{\rm B} (without ever evaluating the E \& B fields or their spherical harmonics). For carrying out these derivations we extensively use the matrix notation introduced in \sec{sec:mat_pol_intro}.

\textit{Euler angles:} All the real space operators we derive are most conveniently expressed as functions of the Euler angles  $\alpha, ~\beta~\&~ \gamma$ on the sphere. Owing to this fact we begin by briefly describing the Euler angles and present a way to think about them. The local cartesian coordinate system at any point on the sphere is defined such that the z-axis points along the radial direction, the x-axis is along the vector tangent to the local longitude pointing south and the y-axis is a vector tangent to the local latitude pointing east. The Euler angles define rotations that transforms the local cartesian coordinate system defined at the location $\hat{n}_0 \equiv (\theta_0,\phi_0)$ such that it aligns with the local cartesian coordinate system at the location $\hat{n}_i \equiv (\theta_i,\phi_i)$ \cite{varshalovich}. The Euler angles can be evaluated using the following functions of the angular coordinates of the two locations $\hat{n}_0 \equiv (\theta_0 , \phi_0)$ and $\hat{n}_i \equiv (\theta_i, \phi_i)$,
%
\beqrys \label{eq:fn_euler}
\tan{\alpha} &=& \frac{\sin{(\phi_i - \phi_0)} \sin{\theta_i}}{\sin{\theta_i}\cos{\theta_0} \cos{(\phi_i - \phi_0)} - \sin{\theta_0} \cos{\theta_i} } \,, \\
\cos{\beta} &=& \sin{\theta_i}\sin{\theta_0} \cos{(\phi_i -\phi_0)} + \cos{\theta_i}\cos{\theta_0} \,,\\
\tan{\gamma} &=& \frac{ - \sin{(\phi_i - \phi_0)} \sin{\theta_0}}{\sin{\theta_0}\cos{\theta_i} \cos{(\phi_i - \phi_0)} - \sin{\theta_i} \cos{\theta_0} } \,,
\eeqrys
%
where $\alpha$ defines the rotation about the z-axis, $\beta$ defines the rotation about the new y-axis (y1-axis) after the previous rotation and $\gamma$ defines the rotation about the final z-axis (z2-axis) after carrying out the previous two rotations. In more physical terms, these angles can be understood as follow: The rotation by $\alpha$ about the z-axis is such that it aligns the x-axis of the cartesian system at location $\hat{n}_0$ along the great circle in the direction of $\hat{n}_i$.  The rotation by $\beta$ about the y2-axis parallel transports the local cartesian coordinate system from location $\hat{n}_0$ to location $\hat{n}_i$, such that the z-axes of the two coordinate systems are parallel to each other. Finally the rotation by angle $\gamma$ about the z2-axis aligns the x \& y axes of the parallel transported system with those of the local cartesian system defined at $\hat{n}_i$. The simplest case is when one of the coordinates coincides with the north pole (more formally this refers to the point $\theta_0 \rightarrow 0$ while moving along the longitude $\phi_0=0$), say $\hat{n}_0=(\theta_0,\phi_0)=(0,0)$. In this case it is easy to check by substituting these coordinate values in  \eq{eq:fn_euler} that the Euler angles are: $(\alpha,\beta,\gamma) =(\phi_i,\theta_i,0)$. While evaluating the above functions we follow the convention that $\beta$ lies in the domain $[0, \pi]$ and that the angles $\alpha ~\&~ \gamma$ lie in the domain $[-\pi,\pi]$. It is important to assign the proper signs to $\alpha ~\&~ \gamma$ by duly accounting for the signs of the term in the numerator and the denominator. \revisit{While in following section we present results in terms of these Euler angles, the local kernels surrounding any pixel can be simply evaluated using standard Healpix functions. We give the details of this procedure in \app{sec:rot_ker_healpix}}.
%--------------------------------------------------------
%--------------------------------------------------------
\subsection{Evaluating E \& B fields from measured Stokes parameters Q \& U}\label{sec:qu2eb}
In \sec{sec:pol-primer} we described how the scalar fields E \& B are derived from the Stokes parameters Q \& U. To reiterate, this process involved taking the spin harmonic transform of the complex spin-2 field $({}_{\pm2} \bar X)$, taking linear combinations of the resultant coefficients of expansion $({}_{\pm 2} \tilde X_{\ell m})$ and evaluating the forward spin-0 transform to derive the scalar E \& B fields. Here we derive the real space convolution kernels on the sphere which can be used to directly evaluate the scalar E \& B fields on the sphere.  
We use the relations given in \eq{eq:pol_data_relns}, to write down an equation relating the real space vector of scalars \vs to the polarization vector \vp{} as given below,
%
\beqrys
\bar{S} &=& {{}_0\mathcal{Y}} *\tilde T^{-1}* {{}_2\mathcal{Y}^{\dagger}} *\bar T *\bar{P} = \frac{1}{2} {{}_0\mathcal{Y}} *\tilde T^{\dagger} *{{}_2\mathcal{Y}^{\dagger}} *\bar T *\bar{P}   \,, \\
&=&  \bar O *\bar{P} \,.
\eeqrys
%
The explicit form of the real space operator $\bar O$ can be derived by contracting over all the matrix operators. This procedure of contracting over the operators is explicitly worked out in the following set of equations,
%
\beqrys
\bar{O} &=& \frac{1}{2} {{}_0\mathcal{Y}} *\tilde T^{\dagger} *{{}_2\mathcal{Y}^{\dagger}} *\bar T \,, \\
&=& -0.5 \yzmat{i} \qutoxd \ymatc{j} \qutox   \,, \\
&=& -0.5 \begin{bmatrix} \sum ({}_{0}Y_i ~{}_{2}Y^{T*}_j  +  {}_{0}Y_i~ {}_{-2}Y^{T*}_j) & {\rm i}  \sum ({}_{0}Y_i~ {}_{2}Y^{T*}_j - {}_{0}Y_i ~{}_{-2}Y^{T*}_j)  \\  - {\rm i} \sum  ({}_{0}Y_i {}_{2}Y^{T*}_j - {}_{0}Y_i {}_{-2}Y^{T*}_j) & \sum ({}_{0}Y_i {}_{2}Y^{T*}_j + {}_{0}Y_i {}_{-2}Y^{T*}_j)  \end{bmatrix} \,, \label{eq:qu2eb_ker_1}
\eeqrys
%
where the symbol ${}_{0}Y_i$ is used to denote the sub-matrix ${}_{0}Y_{\hat{n}_i \times \ell m} \equiv {}_{0}Y_{\ell m}(\hat{n}_i)$, the symbol ${}_{\pm 2}Y^{T*}_j$ is used to denote the matrix ${}_{\pm 2}Y^*_{\ell m \times \hat{n}_j} \equiv {}_{\pm 2}Y^*_{\ell m}(\hat{n}_j)$ and the summation is over the multipole indices $\ell,m$. Using the conjugation properties of the spin spherical harmonic functions it can be shown that the following relation holds true,
%
\beq
 \left [\sum_{\ell m} {}_{0}Y_{\ell m}(\hat{n}_i){}_{+2}Y^*_{\ell m}(\hat{n}_j)\right]^* = \sum_{\ell m} {}_{0}Y_{\ell m}(\hat{n}_i){}_{-2}Y^*_{\ell m}(\hat{n}_j) \,.
 \eeq
 %
 where the terms on either side of the equation are those that appear in \eq{eq:qu2eb_ker_1}.
Therefore the different parts of the real space operator $\bar{O}$  are completely specified in terms of the complex function,
%
\beqrys
\mathcal{M}( \hat{n}_i, \hat{n}_j)  &=& \mathcal{M}_{r} + i \mathcal{M}_{i}  \,,\nonumber \\ 
&=&\sum_{\ell m} {{_0}Y}^*_{\ell m}(\hat n_i) {{_2}Y}_{\ell m}(\hat n_j) = \sum_{\ell} \sqrt{\frac{2\ell+1}{ 4 \pi}}{{_0Y}_{\ell 2}}(\beta_{ij},\alpha_{ij})\,,\\
&=&  \Big [ \cos(2 \alpha_{ij}) + i \sin(2 \alpha_{ij} ) \Big]   \sum_{\ell=\ell_{\rm min}}^{\ell_{\rm max}} {\frac{2\ell+1}{ 4 \pi}} \sqrt{\frac{(\ell-2)!}{(\ell+2)!}}P_{\ell 2} (\cos\beta_{ij}) \,, \label{eq:rad_ker_queb} \\
&=&  \Big [ \cos(2 \alpha_{ij}) + i \sin(2 \alpha_{ij} ) \Big] f(\beta_{ij},\ell_{\rm min},\ell_{\rm max}) \,, 
\eeqrys
%
where we have used the property of summation over spin spherical harmonics (see \eq{eq:sum_spin_shf}) listed in Appendix \ref{sec:ylm_mathprop}.  On simplifying \eq{eq:qu2eb_ker_1}, the local convolution kernel can be cast in this simple form,
%
\beq\label{eq:op_qu2eb}
\bar O =-\bmat  \mathcal{M}_{r} & \mathcal{M}_{i} \\  -\mathcal{M}_{i}  & \mathcal{M}_{r} \emat_{2 N_{\rm pix} \times 2 N_{pix}} = -f(\beta_{ij},\ell_{\rm min},\ell_{\rm max})\bmat \cos(2 \alpha_{ij}) & \sin(2\alpha_{ij})\\  -\sin(2 \alpha_{ij})  & \cos(2 \alpha_{ij}) \emat \,,
\eeq
%
where indices i,j map to the location $\hat{n}_i$ and $\hat{n}_j$ on the sphere. \revisit{A similar equation for real space E \& B operators was derived in \cite{Zaldarriaga2001a}, however those results were derived for the flat sky case and did not explicitly derive the radial kernel.} \comment{A discussion on this should be in the conclusions.}

The scalar fields E \& B can now be directly derived from the measured Stokes Q \& U parameters by evaluating the following convolution,
%
\beq \label{eq:qu2eb_convolution_explicit}
\bmat E_i \\ B_i  \emat= -\Delta \Omega \sum_{j=1}^{N_{\rm pix}}f(\beta_{ij},\ell_{\rm min},\ell_{\rm max})\bmat \cos(2 \alpha_{ij}) & \sin(2\alpha_{ij})\\  -\sin(2 \alpha_{ij})  & \cos(2 \alpha_{ij}) \emat  \bmat Q_j \\ U_j  \emat \,,
\eeq
%
where $\Delta \Omega$ denotes the pixel area and all the symbols have their usual meaning. 
%\revisit{This has an elegant interpretation: to derive the E and/or B field at any given position we need to find the cosine quadrupole transform and the sine quadrupole transform of the Stokes Q \& U parameters on circles around this position, weigh the transform by the value of the function $f(\beta,\ell_{\rm min},\ell_{\rm max})$, $\beta$ being the radius of the circle and sum up the results with appropriate signs, to construct the respective scalar fields.} 
The above equation can be expressed more concisely as follows,
%
\begin{subequations} \label{eq:qu2eb_simple}
\beqry 
[E + iB](\hat{n}_0) &=& - \Delta \Omega \sum_{j=1}^{N_{\rm pix}} \left(\sum_{\ell=\ell_{\rm min}}^{\ell_{\rm max}} \sqrt{\frac{2 \ell+1}{4 \pi}} P_{\ell}^{2}(\beta_{0j}) \right) {\Bigg( e^{-i2 \alpha_{0j}}   {}_{+2}X (\hat{n}_j) \Bigg)} \,, \label{eq:qu2eb_physical}\\
&=& - \left( \left[ \sum_{\ell=\ell_{\rm min}}^{\ell_{\rm max}} \sqrt{\frac{2 \ell+1}{4 \pi}}Y^*_{\ell 2} \right]  \circ {}_{+2}X \right)\,, \label{eq:qu2eb_convolution} 
\eeqry
\end{subequations}
%
where $\circ$ denotes a convolution and the spherical harmonic functions denote the rotated functions such that the pole of the function coincides with the direction $\hat{n}_0$ and the reference zero for the azimuthal angle is the local longitude $\phi_0$. We know that the spin of product of functions with different spins is given by: ${}_{s_1}f {}_{s_2}g = {}_{s_1 +s_2}h$. Since $[Q + iU]$ is a field with spin +2 and the function $\exp(-i2\alpha)$ is a field with spin -2, the resultant field formed by the product of these two function has spin-0. This makes intuitive, the construction of the spin-0 E and B modes of polarization. Note that the coordinate dependence of the Stokes parameters cannot be integrated out at the location $\hat{n}_0$ where the scalar fields are to be evaluated since the azimuthal angle becomes ill-defined. This issue is averted by the fact that the radial function is formed by taking weighted linear combinations of $P_{\ell}^2$ Legendre polynomials which have vanishing contribution $(P_{\ell}^2(\beta) \propto \sin^2{\beta})$ as $\beta \rightarrow 0 ,\pi$.  \revisit{Also note that the E-modes are constructed by product of functions $(U\sin{2 \alpha}, Q\cos{2\alpha})$ which have the same parity and hence have even parity while the B-modes are constructed by multiplying functions $(Q\sin{2 \alpha}, U\cos{2\alpha})$ of opposite parity and hence have an odd parity.}

Note that this kernel does not depend on the Euler angle $\gamma$ \comment{Why is that and how do you understand this ? }. This local convolution kernel has a part which depends only on the Euler angle $\alpha$ which has no multipole dependence. The remaining part of the kernel is specified by the function $f(\beta,\ell_{\rm min},\ell_{\rm max})$ which depends only on the Euler angle $\beta$ and completely incorporates the multipole $\ell$ dependence of the kernel. We will refer to $f(\beta,\ell_{\rm min},\ell_{\rm max})$ as the radial part of the kernel since it only depends on the angular distance between different locations. Its the radial part of the kernel that completely determines the locality of this operator. 

\rfedit{The azimuthal operations do not depend on the choice of basis functions, and can be argued as a requirement for constructing spin-0 fields by operating on spin-2 fields. The radial kernel however is determined by the choice of the basis functions. Motivated by this observation, one can now think of working with alternate basis functions which have different radial fall off.} \comment{This is generally true for all the local kernels. So shift this line to the section where we explore different radial functions.}
%--------------------------------------------------------
%--------------------------------------------------------
\subsection{Evaluating Stokes parameters Q \& U fields from E \& B fields}\label{sec:eb2qu}
The real space operator which translates E \& B fields to Stokes parameters Q \& U is derived using a similar procedure. The inverse operator is given by the following expression,
%
\beqry
\bar{P} &=& \bar{T}^{-1} *{{}_2\mathcal{Y}} *\tilde T *{{}_0\mathcal{Y}^{\dagger}}\bar{S} = \frac{1}{2} \bar{T}^{\dagger} *{{}_2\mathcal{Y}} *\tilde T *{{}_0\mathcal{Y}^{\dagger}}\bar{S}   \\
&=&  \bar O^{-1} *\bar{S}
\eeqry
%
We do not provide the explicit calculations here, since the derivation is nearly identical to that discussed in the previous section. The inverse operator is given by the following expression,
%
\beq
{\bar O}^{-1}=-\bmat \mathcal{M}_{r} & -\mathcal{M}_{i} \\  \mathcal{M}_{i}  & \mathcal{M}_{r} \emat_{2 N_{\rm pix} \times 2 N_{pix}} =-f(\beta_{ij},\ell_{\rm min},\ell_{\rm max})\bmat \cos(2 \alpha_{ij}) & -\sin(2\alpha_{ij})\\  \sin(2 \alpha_{ij})  & \cos(2 \alpha_{ij}) \emat \,.
\eeq
%
%where all the symbols have the same meaning as discussed in \sec{sec:qu2eb}.
Note that the kernel is different by a mere change in sign on the off-diagonals of the block matrix as compared to \eq{eq:op_qu2eb}.
We can evaluate the Stokes Q \& U parameters from the scalar E \& B  fields by evaluating the following expression,
%
\beq
\bmat Q_i \\ U_i  \emat=-\Delta \Omega\sum_{j=1}^{N_{\rm pix}}f(\beta_{ij},\ell_{\rm min},\ell_{\rm max})\bmat \cos(2 \alpha_{ij}) & -\sin(2\alpha_{ij})\\  \sin(2 \alpha_{ij})  & \cos(2 \alpha_{ij}) \emat  \bmat E_j \\ B_j  \emat \,,
\eeq
%
where all the symbols have their usual meaning. The above equation can again be expressed more concisely as follows,
%
\begin{subequations} \label{eq:eb2qu_simple}
\beqry 
[Q + iU](\hat{n}_0) &=& - \Delta \Omega \sum_{j=1}^{N_{\rm pix}} \left(\sum_{\ell=\ell_{\rm min}}^{\ell_{\rm max}} \sqrt{\frac{2 \ell+1}{4 \pi}} P_{\ell}^{2}(\beta_{0j}) \right)  \mathbf{\Bigg( e^{i2 \alpha_{0j}}   [E+iB](\hat{n}_j) \Bigg)} \,, \label{eq:eb2qu_physical}\\
{}_{+2}X(\hat{n}_0) &=& - \left( \left[ \sum_{\ell=\ell_{\rm min}}^{\ell_{\rm max}} \sqrt{\frac{2 \ell+1}{4 \pi}}Y_{\ell 2} \right]  \circ [E+iB] \right)\,, 
\eeqry
\end{subequations}
%
which is to be interpreted in the same manner as \eq{eq:qu2eb_simple}. Note that the only change in the convolution kernel as compared to that in \eq{eq:op_qu2eb} is that the signs of the off-diagonal block matrices are reversed. Alternatively, the $Y_{\ell 2}$ functions are not conjugated. This can again be simply understood as constructing a spin-2 field by taking a product of a spin-0 field: $[E +iB]$ with a spin $+2$ field: $\exp(+i2\alpha)$. The radial dependence of the operator ${\bar O}^{-1}$  is identical to the that of ${\bar O}$ as one may have expected. In this case all the coordinate dependence in encoded in the function $\exp(+i2\alpha)$, which is ill-defined at the poles ($\beta \rightarrow 0,\pi$) and hence the radial functions have to vanish at these locations, which is the case for the same reasons discussed previously.

\comment{How does the radial kernel reduce to unity on evaluating the the operator on to its inverse ? This will be important to understand how to define alternate radial functions.}
%--------------------------------------------------------
%--------------------------------------------------------
\subsection{Decomposing Q \& U Stokes parameters into those corresponding to E \& B modes respectively}
The Stokes Q \& U parameters can be decomposed into the scalar modes E \& B and vice verse, as seen in the previous sections. The E \& B modes are orthogonal to each other, in the sense that their respective operators are orthogonal to each other as seen in \eq{eq:op_eb_ortho}. It is possible to decompose the Stokes vector \vp{} into one \vp{\rm E} that purely contributes to E modes and another \vp{\rm B} that purely contribute to the B modes of polarization. We can only measure the total Stokes vector which is a sum of the Stokes vectors corresponding to the respective scalar modes.  In this section we derive the real space operators which directly perform this decomposition, without ever having to evaluate the E \& B modes. The derivation is similar to that discussed in \sec{sec:qu2eb}, though the algebra is a little more involved in this case. Here we use the harmonic space projection operators $\tilde O_{E/B}$, defined in \eq{eq:har_eb_op}, to derive the respective real space operators. The Stokes parameters corresponding to each scalar mode are given by the following expressions,
%
\beqry
\bar{P}_E &=&  [\bar T^{-1} * {{}_2\mathcal{Y}} *\tilde T * \tilde O_E* \tilde T^{-1}* {{}_2\mathcal{Y}^{\dagger}} *\bar T] *\bar{P}  \,, \\
&=& [\frac{1}{4} \bar T^{\dagger } * {{}_2\mathcal{Y}} *\tilde T * \tilde O_E* \tilde T^{\dagger} * {{}_2\mathcal{Y}^{\dagger}} *\bar T ]*\bar{P}  \,, \nonumber \\
&=&  \bar O_{E}*\bar{P} \,,\nonumber \\
\bar{P}_B &=&  [\bar T^{-1}* {{}_2\mathcal{Y}}* \tilde T* \tilde O_B* \tilde T^{-1}* {{}_2\mathcal{Y}^{\dagger}}* \bar T]*\bar{P}  \,, \\
&=& [\frac{1}{4} \bar T^{\dagger } * {{}_2\mathcal{Y}} *\tilde T * \tilde O_B* \tilde T^{\dagger} *{{}_2\mathcal{Y}^{\dagger}} *\bar T] *\bar{P}   \,, \nonumber\\
&=&  \bar O_{B}*\bar{P} \,. \nonumber
\eeqry
%
We contract over all the matrix operators to arrive at the the real space operators. On working through the algebra it can be shown that the real space operators have the following form,
%
\beq
\bar O_{E/B} = 0.5 \bmat \mathcal{I}_{r} & \mathcal{I}_{i} \\  -\mathcal{I}_{i}  & \mathcal{I}_{r} \emat \pm \bmat \mathcal{D}_{r} & \mathcal{D}_{i} \\  \mathcal{D}_{i}  & - \mathcal{D}_{r} \emat \,,\\
\eeq
where $\mathcal{I}_{r} ~\&~ \mathcal{D}_{r}$ and $\mathcal{I}_{i} ~\&~ \mathcal{D}_{i}$ are the real and complex parts of the following complex functions,
\beqry
\mathcal{I} &=& \mathcal{I}_{r} + i \mathcal{I}_{i} = \sum_{\ell m} {_{-2}Y}_{\ell m}(\hat n_i) {_{-2}Y}^*_{\ell m}(\hat n_j) \,, \nonumber \\
\mathcal{D}  &=& \mathcal{D}_{r} + i\mathcal{D}_{i} = \sum_{\ell m} {_2Y}_{\ell m}(\hat n_i) {_{-2}Y}^*_{\ell m}(\hat n_j) \,.\nonumber
\eeqry
%
These functions can be further simplified using the properties of spin spherical harmonics listed in Appendix~\ref{sec:ylm_mathprop}. Specifically it can be shown that these functions reduce to the following mathematical forms,
%
\beqrys \label{eq:fn_i}
\mathcal{I}(\hat{n}_i, \hat{n}_j) &=& \sum_{\ell} \sqrt{\frac{2\ell+1}{ 4 \pi}}{_{-2}Y}_{\ell2}(\beta_{ij}, \alpha_{ij}) ~ \rm{e}^{i2 \gamma_{ij}} \label{eq:healpix-compatible-i} = \mathcal{I}_r + i \mathcal{I}_i \,, \\
\mathcal{I}_r + i \mathcal{I}_i &=& \Big [ \cos(2 \alpha_{ij} +  2\gamma_{ij}) + i \sin(2 \alpha_{ij} +  2 \gamma_{ij}) \Big]   _{-2}f(\beta_{ij},\ell_{\rm min},\ell_{\rm max}) \,,
\eeqrys
%
%
\beqrys \label{eq:fn_d}
\mathcal{D}(\hat{n}_i, \hat{n}_j) &=& \sum_{\ell} \sqrt{\frac{2\ell+1}{ 4 \pi}}{_2Y}_{\ell 2}(\beta_{ij}, \alpha_{ij}) ~ \rm{e}^{- i2 \gamma_{ij}} \label{eq:healpix-compatible-m} =\mathcal{D}_r + i \mathcal{D}_i \,, \\
\mathcal{D}_r + i \mathcal{D}_i &=&  \Big [ \cos(2 \alpha_{ij} - 2\gamma_{ij}) + i \sin(2 \alpha_{ij} -  2 \gamma_{ij}) \Big]   _{+2}f(\beta_{ij},\ell_{\rm min},\ell_{\rm max}) \,,
\eeqrys
%
where the functions,
%
\beq
{}_{\pm2}f(\beta,\ell_{\rm min},\ell_{\rm max}) = \sum_{\ell=\ell_{\rm min}}^{\ell_{\rm max}} \sqrt{\frac{2\ell+1}{ 4 \pi}} _{ \pm 2}{f}_{\ell}(\beta) \label{eq:f2_rad_ker}\,,
\eeq
%
can be expressed in terms of $P_{\ell}^2$ Legendre polynomials and are given by the following explicit mathematical forms,
 %
 \beqry
 _{\pm 2}f_{\ell}(\beta) &=& 2 \frac{(\ell-2)!}{(\ell+2)!}  \sqrt{\frac{2\ell +1 }{4 \pi}} \Bigg[ - P_{\ell}^{2} (\cos  \beta) \left( \frac{\ell-4}{\sin^2 \beta} + \frac{1}{2}\ell(\ell-1) \pm \frac{2 (\ell-1) \cos \beta}{\sin^2 \beta} \right) \nonumber \\ 
&+& P_{\ell-1}^2 (\cos \beta) \left( (\ell+2) \frac{\cos \beta}{\sin^2 \beta} \pm \frac{2 (\ell+2)}{ \sin^2 \beta } \right) \Bigg] \,. \label{eq:rad_ker_quequbqu}
 \eeqry
 %
Finally, the Stokes parameters corresponding to the respective scalar fields are given by evaluating the following expressions, 
 %
\beqry \label{eq:op_qu2equbqu}
\bmat Q_i \\ U_i  \emat_{E/B} &=& \sum_{j=1}^{N_{\rm pix}} \Bigg\lbrace {}_{-2}f(\beta_{ij},\ell_{\rm min},\ell_{\rm max}) \bmat \cos(2 \alpha_{ij} + 2\gamma_{ij}) & \sin(2\alpha_{ij} +2 \gamma_{ij}) \\  -\sin(2\alpha_{ij} +2 \gamma_{ij})  & \cos(2 \alpha_{ij} + 2 \gamma_{ij}) \emat  \bmat Q_j \\ U_j  \emat  \\ &\pm& {}_{+2}f(\beta_{ij},\ell_{\rm min},\ell_{\rm max}) \bmat \cos(2 \alpha_{ij} - 2\gamma_{ij}) &  \sin(2\alpha_{ij} - 2 \gamma_{ij}) \\  \sin(2\alpha_{ij} - 2 \gamma_{ij})  & - \cos(2 \alpha_{ij} - 2 \gamma_{ij}) \emat  \bmat Q_j \\ U_j  \emat \Bigg\rbrace 0.5 \Delta\Omega  \,, \nonumber 
\eeqry
%
where all the symbols have their usual meaning. The above expression can be cast in the following simplified form,
%
\begin{subequations}
\beqry
{}_{+2}X_{E/B} &=&0.5 \Delta \Omega\sum_{j=1}^{N_{\rm pix}}  {}_{-2}f(\beta_{ij}) e^{-i2 (\alpha_{ij} + \gamma_{ij})} {}_{+2}X_j \pm {}_{+2}f(\beta_{ij}) e^{i2 (\alpha_{ij} - \gamma_{ij})} {}_{+2}X_j^*\,, \label{eq:qu2equbqu_explicit_convolution} \\
&=& 0.5 \Bigg\lbrace I^* \circ {}_{+2}X\pm D \circ {}_{+2}X^* \Bigg\rbrace \,, \label{eq:qu2equbqu_convolution}
\eeqry
\end{subequations}
%
where in \eq{eq:qu2equbqu_explicit_convolution} we have suppressed the explicit multipole dependence of functions $_{\pm 2}f$ for brevity and in \eq{eq:qu2equbqu_convolution} $\circ$ denotes a convolution. 

It is easy to show that the operator $D$ is a complex but symmetric matrix and that the operator $I$ is Hermitian. The function $I$ is a band limited version of the delta function ($\lim_{\ell \rightarrow \infty} I = \delta(\hat{n}_i - \hat{n}_j)$), when interpreted as a matrix operator forms a band limited version of the identity matrix. Though its non-local ($I \neq 0 $ when $\hat{n}_i \neq \hat{n}_j$) owing to the band limit, for all practical purposes the operator $I$ acts like an identity operator as is confirmed by the following identities it satisfies: (i) $I*I=I$ ; (ii) $D*I=D$. Also $D^*$ is the inverse of $D$ in this band limited sense: $D^**D=I$. Using these properties of the operators $I$ and $D$, one can verify that the real space operators satisfy the following identities,
%
\begin{subequations}
\beqry
\bar O_E * \bar O_E &=& \bar O_E ~~;~~ \bar O_B * \bar O_B = \bar O_B \,, \\
\bar O_E*\bar O_B &=& 0 \,,\\
\bar O_E + \bar O_B &=& I \,,
\eeqry
\end{subequations}
%
which are the real space analogues of \eq{eq:har_op_prop}. While testing the above stated identities one encounters terms like $D*I^*,I^**I \textrm{ and }  I*I^*$ which cannot be simply interpreted, but they always occur in pairs with opposite signs, hence exactly canceling each other. \revisit{Note that unlike in the harmonic case, the sum of the operators is not exactly an identity matrix. \comment{Isn't it possible to do some real space operations which doesn't care for the band limit but still returns the required decomposition into the respective scalar modes ?} This indicates the loss of information occurred from the original data due to making this transformations with some assumed band limit. If one forces the sum of the operators to be exactly an identity operator, then one compromises on the orthogonality property of the $\bar{O}_E$ \& $\bar{O}_B$ operators.}