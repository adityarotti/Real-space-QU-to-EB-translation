\subsection{Matrix notation} \label{sec:mat_pol_intro}
In this section we cast the relation introduced in Sec.~\ref{sec:pol-primer} in matrix notation\footnote{While we work with the matrix and vector sizes given in terms of Healpix pixelization parameter $\rm N_{\rm pix}$, all the relations are equally valid in the continuum limit attained by allowing $\rm N_{\rm pix}\rightarrow \infty$}. This representation will make transparent the derivation of the real space operators we discuss in the following sections. We adopt a convention in which real space quantities are denoted by bar-ed variable while those in harmonic space are denoted by tilde-ed variables.\\
We begin by introducing the matrices encoding the spin spherical harmonic basis vectors,
%
\beq
_sB= \bmat _{+s}Y & 0 \\ 0 & _{-s}Y \emat _{2 \rm N_{\rm pix} \times 2 \rm N_{\rm alms}} \,,
\eeq
%
where $s$ denotes the spin of the basis functions. For this work we will only be working with cases $s \in [0,2]$. In this notation, each column can be mapped to a specific harmonic basis function marked by the pair of indices:$(\ell,m)$ and each row maps to a specific position on the sphere. Note that this matrix is in general not a square matrix. Generally the number of columns is determined by the scheme used to discretely represent the sphere and the number of rows is set by the number of basis functions of interest (often determined the band limit).

We now define the different data vectors and their representation in real and harmonic space as follows,
%
\beqrys
\bar{S} &=& \bmat E \\ B  \emat_{2 \rm N_{\rm pix} \times 1} ~~~~;~~ \bar{X} = \bmat _{+2}X \\ _{-2}X \emat_{2 \rm N_{\rm pix} \times 1} ~~;~~\bar{P} =\fqu_{\tiny {2 \rm N_{\rm pix} \times 1}} \,, \\
\tilde{S} &=& \bmat a^{E} \\ a^{B} \emat _{2 \rm N_{\rm alms} \times 1}  ~~; ~~ \tilde{X} = \bmat _{+2} \tilde{X} \\ _{-2} \tilde{X} \emat_{2 \rm N_{\rm alms} \times 1} \,.
\eeqrys
%
The different symbols have the same meaning as that discussed in \sec{sec:pol-primer}, except that the subscript $_{\ell m}$ for the spherical harmonic coefficients of expansion is suppressed to avoid clutter in notation.

Next we define the operators which govern the transformations between different representations of the polarization field as follows,
%
\beqrys
\bar T &=& \qutox_{2 \rm N_{\rm pix} \times 2 \rm N_{\rm pix}} ~~;~~ \bar T^{-1} = \frac{1}{2} \bar T^{\dagger} \,, \\
\tilde T &=& -\qutox_{2 \rm N_{\rm alms} \times 2 \rm N_{\rm alms}} ~~;~~ \tilde T^{-1} = \frac{1}{2} \tilde T^{\dagger} \,. 
\eeqrys
%
Using the data vectors and the all the operators defined in this section we now write down, in compact notation, the forward and inverse relations between different representations of the polarization field as follows,
%
\beqrys \label{eq:pol_data_relns}
\bar{X} &=& \bar T * \bar{P} ~~;~~\bar{P} = \frac{1}{2} \bar T^{\dagger} * \bar{X} \,, \\
\bar X &=&  {_2B} * \tilde X  ~~;~~ \tilde X ={_2B}^{\dagger} * \bar X  \,, \\
\tilde{X} &=& \tilde T * \tilde{S} ~~;~~ \tilde{S} = \frac{1}{2}\tilde T^{\dagger} * \tilde{X} \,.\\ 
\bar S &=&  {_0B} * \tilde S ~~;~~  \tilde S =  {_0B}^{\dagger} * \bar S \,.
%\tilde X &=&  {_2B}^{\dagger} * \bar X ~~;~~ \tilde{X} = \tilde T * \tilde{S} \,, \\
%\bar{S} &=& {_0B}*\tilde S ~~;~~ \tilde{S} = \frac{1}{2}\tilde T^{\dagger} * \tilde{X} \,.\\
%\bar{X} &=& \bar T * \bar{P} ~~;~~ \tilde{X} = \tilde T * \tilde{S} \,, \\
%\bar{P} &=& \frac{1}{2} \bar T^{\dagger} * \bar{X} ~~;~~ \tilde{S} = \frac{1}{2}\tilde T^{\dagger} * \tilde{X} \,. \\
\eeqrys
%
Next we introduce the harmonic space operators, which project the harmonic space data vector to E or B subspace,
%
\beqrys
\tilde O_E &=& \bmat \mathbb{1} & \mathbb{0} \\ \mathbb{0} & \mathbb{0} \emat _{2 \rm N_{\rm alms} \times 2 \rm N_{\rm alms} }   ~~;~~ \tilde S_E = \tilde O_E* \tilde S \,,\\
\tilde O_B &=& \bmat \mathbb{0} & \mathbb{0} \\ \mathbb{0} & \mathbb{1} \emat _{2 \rm N_{\rm alms} \times 2 \rm N_{\rm alms} } ~~; ~~ \tilde S_B = \tilde O_B *\tilde S \,.
\eeqrys
%
Note that these harmonic space matrices are idempotent, orthogonal to each other and their sum is an identity matrix as can be explicitly seen via the following relations, 
%
\beqrys\label{eq:eb_har_proj}
\tilde O_E * \tilde O_E&=& \tilde O_E ~~;~~  \tilde O_B * \tilde O_B= \tilde O_B \,,\\
 \tilde O_E * \tilde O_B&=& \mathbb{0} \,, \\
 \tilde O_E + \tilde O_B&=& \mathbb{1} \,.
\eeqrys
%
Here it is important to note that these relations are exact in harmonic space.  \revisit{In the following sections our aim is to derive real space analogues of these harmonic space operators. }
%%%%%%%%%%%%%%%%%%%%%%%%%%%%%%%%%%%
\section{Real space operators} \label{sec:real_space_operators}
In this section we derive the real space operators which translate the Stokes parameters Q \& U to E \& B fields and vice versa. We also derive real space operators for directly(without first evaluating the E \& B field themselves) decomposing the Stokes  parameters Q \& U in to Stokes parameters that correspond to the E \& B fields respectively. We extensively make use of the matrix notation introduces in \sec{sec:mat_pol_intro} for these derivations.

All the results are most conveniently expressed as functions of the Euler angles  $\alpha, ~\beta~\&~ \gamma$ on the sphere. Generically, the Euler angles define the rotations that transforms the local cartesian coordinate system defined at the sphere position $\hat{n}_i \equiv (\theta_i,\phi_i)$ such that it aligns with the local cartesian coordinate system at the location $\hat{n}_j \equiv (\theta_j,\phi_j)$ \cite{varshalovich}. The Euler angles can be evaluated as the following functions of the angular coordinates of the points $\hat{n}_i \equiv (\theta_i , \phi_i)$ and $\hat{n}_j \equiv (\theta_j, \phi_j)$,
%
\beqrys \label{eq:fn_euler}
\cos(\beta) &=& \sin(\theta_i)\sin(\theta_j) \cos(\phi_i -\phi_j) + \cos(\theta_i)\cos(\theta_j) \,,\\
\tan(\alpha) &=& \frac{\sin(\phi_i - \phi_j) \sin(\theta_i) \sin(\theta_j)}{\cos(\theta_i) \cos(\beta) - \cos(\theta_j)} \,, \\
\tan(\gamma) &=& \frac{\sin(\phi_i - \phi_j) \sin(\theta_i) \sin(\theta_j)}{\cos(\theta_j) \cos(\beta) - \cos(\theta_i)}\,,
\eeqrys
%
where $\beta$ denotes the angular distance between the two points $\hat{n}_i$ \& $\hat{n}_j$ on the sphere, while the angles $\alpha$ \& $\gamma$ define rotations which co-align the coordinate axes of the two local coordinate systems. While evaluating the above functions we follow the convention that $\beta$ lies in the domain $[0, \pi]$ and that the angles $\alpha ~\&~ \gamma$ lie in the domain $[-\pi,\pi]$. It is important to assign the proper signs to $\alpha ~\&~ \gamma$ by duly accounting for the signs of the term in the numerator and the denominator. 
%--------------------------------------------------------
%--------------------------------------------------------
\subsection{Evaluating E \& B fields from measured Stokes parameters Q \& U}\label{sec:qu2eb}
In \sec{sec:pol-primer} we discussed how the scalar fields E \& B are derived from the Stokes parameters Q \& U. To reiterate, this process involved taking the spin harmonic transform of the complex spin-2 field $({}_{\pm2} \bar X)$, taking linear combinations of the resultant coefficients of expansion $({}_{\pm 2} \tilde X_{\ell m})$ and evaluating the forward spin-0 transform to derive the scalar E \& B fields. Here we derive the real space convolution kernels on the sphere which can be used to directly evaluate the scalar E \& B fields on the sphere.  
We use the relations given in \revisit{\eq{eq:pol_data_relns}}, to write down an equation relating the real space vector of scalars $\bar{S}^{\dagger}=[E, B]$ to the polarization vector $\bar{P}^{\dagger}=[Q, U]$ as given below,
%
\beqrys
\bar{S} &=& {_0B} *\tilde T^{-1}* {_2B^{\dagger}} *\bar T *\bar{P} = \frac{1}{2} {_0B} *\tilde T^{\dagger} {_2B^{\dagger}} *\bar T *\bar{P}   \,, \\
&=&  \bar O *\bar{P} \,.
\eeqrys
%
The explicit form of the real space operator $\bar O$ can be derived by contracting over all the matrix operators. This procedure of contracting over the operators is explicitly worked out in the following set of equations,
%
\beqrys
\bar{O} &=& \frac{1}{2} {_0B} *\tilde T^{\dagger} *{_2B^{\dagger}} *\bar T \,, \\
&=& -0.5 \yzmat{i} \qutoxd \ymatc{j} \qutox   \,, \\
&=& -0.5 \begin{bmatrix} \sum ({}_{0}Y_i ~{}_{2}Y^{T*}_j  +  {}_{0}Y_i~ {}_{-2}Y^{T*}_j) & {\rm i}  \sum ({}_{0}Y_i~ {}_{2}Y^{T*}_j - {}_{0}Y_i ~{}_{-2}Y^{T*}_j)  \\  - {\rm i} \sum  ({}_{0}Y_i {}_{2}Y^{T*}_j - {}_{0}Y_i {}_{-2}Y^{T*}_j) & \sum ({}_{0}Y_i {}_{2}Y^{T*}_j + {}_{0}Y_i {}_{-2}Y^{T*}_j)  \end{bmatrix} \,, \label{eq:qu2eb_ker_1}
\eeqrys
%
where the symbol ${}_{0}Y_i$ is used to denote the matrix ${}_{0}Y_{\hat{n}_i \times \ell m} \equiv {}_{0}Y_{\ell m}(\hat{n}_i)$, the symbol ${}_{\pm 2}Y^{T*}_j$ is used to denote the matrix ${}_{\pm 2}Y^*_{\ell m \times \hat{n}_j} \equiv {}_{\pm 2}Y^*_{\ell m}(\hat{n}_j)$ and the summation is over the multipole indices $\ell,m$. Using the conjugation properties of the spin spherical harmonic functions it can be shown that the following relation holds true,
%
\beq
 \left [\sum_{\ell m} {}_{0}Y_{\ell m}(\hat{n}_i){}_{+2}Y^*_{\ell m}(\hat{n}_j)\right]^* = \sum_{\ell m} {}_{0}Y_{\ell m}(\hat{n}_i){}_{-2}Y^*_{\ell m}(\hat{n}_j) \,.
 \eeq
 %
 where the terms on either side of the equation are those that appear in \eq{eq:qu2eb_ker_1}.
Therefore the different parts of the real space operators  are completely specified in terms of the complex function,
%
\beqrys
\mathcal{M}( \hat{n}_i, \hat{n}_j)  &=& \mathcal{M}_{r} + i \mathcal{M}_{i}  \,,\nonumber \\ 
&=&\sum_{\ell m} {{_0}Y}_{\ell m}(\hat n_i) {{_2}Y}^*_{\ell m}(\hat n_j) = \sum_{\ell} \sqrt{\frac{2\ell+1}{ 4 \pi}}{{_0Y}^*_{\ell 2}}(\beta_{ij},\alpha_{ij})\,,\\
&=&  \Big [ \cos(2 \alpha_{ij}) - i \sin(2 \alpha_{ij} ) \Big]   \sum_{\ell=\ell_{\rm min}}^{\ell_{\rm max}} {\frac{2\ell+1}{ 4 \pi}} \sqrt{\frac{(\ell-2)!}{(\ell+2)!}}P_{\ell 2} (\cos\beta_{ij}) \,, \label{eq:rad_ker_queb} \\
&=&  \Big [ \cos(2 \alpha_{ij}) - i \sin(2 \alpha_{ij} ) \Big] f(\beta_{ij},\ell_{\rm min},\ell_{\rm max}) \,, 
\eeqrys
%
where we have used the property of summation over spin spherical harmonics (see \eq{eq:sum_spin_shf}) listed in Appendix \ref{sec:ylm_mathprop}. Here we first note that this function does not depend on the Euler angle $\gamma$. This function has a part which depends only on the Euler angle $\alpha$ and this part of the function has no multipole dependence.  except the factor of 2 which arises because the polarization field is a spin-2 field. The other part of the function $f(\beta,\ell_{\rm min},\ell_{\rm max})$ depends only on the Euler angle $\beta$ and completely incorporates the multipole dependence of the function. $f(\beta,\ell_{\rm min},\ell_{\rm max})$ will be often be referred to as the radial kernel. The radial kernel is what determines the locality of the operator which translates the Stokes parameters Q \& U to the scalars E \& B.  

Finally the real space operator can be cast in this simple form,
%
\beq\label{eq:op_qu2eb}
\bar O =\bmat  -\mathcal{M}_{r} & -\mathcal{M}_{i} \\  \mathcal{M}_{i}  & -\mathcal{M}_{r} \emat_{2 N_{\rm pix} \times 2 N_{pix}} = -f(\beta_{ij},\ell_{\rm min},\ell_{\rm max})\bmat \cos(2 \alpha_{ij}) & \sin(2\alpha_{ij})\\  -\sin(2 \alpha_{ij})  & \cos(2 \alpha_{ij}) \emat \,,
\eeq
%
where i,j indices map to the location $\hat{n}_i$ and $\hat{n}_j$ on the sphere. \revisit{A similar equation for real space E \& B operators was derived in \cite{Zaldarriaga2001a}, however those results are derived for the flat sky case and do not explicitly derive the radial kernel.} \comment{Maybe a discussion on this should be in the conclusions.}

The scalar fields E \& B can now be directly derived from the measured Stokes Q \& U parameters by evaluating the following expression,
%
\beq \label{eq:qu2eb_convolution}
\bmat E_i \\ B_i  \emat= -f(\beta_{ij},\ell_{\rm min},\ell_{\rm max})\bmat \cos(2 \alpha_{ij}) & \sin(2\alpha_{ij})\\  -\sin(2 \alpha_{ij})  & \cos(2 \alpha_{ij}) \emat  \bmat Q_j \\ U_j  \emat \Delta \Omega\,,
\eeq
%
where we have used the Einstein summation convention: repeated indices are summed over. The factor $\Delta \Omega$ accounts for the finite pixel size and is important for proper normalization. \revisit{This has an elegant interpretation: to derive the E and/or B field at any given position we need to find the cosine quadrupole transform and the sine quadrupole transform of the Stokes Q \& U parameters on circles around this position, weigh the transform by the value of the function $f(\beta,\ell_{\rm min},\ell_{\rm max})$, $\beta$ being the radius of the circle and sum up the results with appropriate signs, to construct the respective scalar fields.}
\rfedit{While the azimuthal operations do not depend on the choice of basis functions, the radial kernel is completely determined by the choice of the basis functions. One can now think of constructing alternate basis functions which have different radial fall off.}
%--------------------------------------------------------
%--------------------------------------------------------
\subsection{Evaluating Stokes parameters Q \& U fields from E \& B fields}\label{sec:eb2qu}
The real space operator which translates E \& B fields to Stokes parameters Q \& U is derived using a similar procedure. The inverse operator is given by the following expression,
%
\beqry
\bar{P} &=& \bar{T}^{-1} *{_2B} *\tilde T *{_0B^{\dagger}}\bar{S} = \frac{1}{2} \bar{T}^{\dagger} *{_2B} *\tilde T *{_0B^{\dagger}}\bar{S}   \\
&=&  \bar O^{-1} *\bar{S}
\eeqry
%
We do not provide the explicit calculations here, since the real space inverse operator can be derived by contracting over all the matrix operators using a procedure nearly identical to that discussed in the previous section. The inverse operator is given by the following expression,
%
\beq
{\bar O}^{-1}=\bmat - \mathcal{M}_{r} & \mathcal{M}_{i} \\  -\mathcal{M}_{i}  & - \mathcal{M}_{r} \emat_{2 N_{\rm pix} \times 2 N_{pix}} =-f(\beta_{ij},\ell_{\rm min},\ell_{\rm max})\bmat \cos(2 \alpha_{ij}) & -\sin(2\alpha_{ij})\\  \sin(2 \alpha_{ij})  & \cos(2 \alpha_{ij}) \emat \,,
\eeq
%
where all the symbols have the same meaning as discussed in \sec{sec:qu2eb}.
Note that the kernel is different by a mere change in sign on the off-diagonals of the block matrix as compared to \eq{eq:op_qu2eb}.
We can evaluate the Stokes Q \& U parameters from the scalar E \& B  fields by evaluating the following expression,
%
\beq
\bmat Q_i \\ U_i  \emat=-f(\beta_{ij})\bmat \cos(2 \alpha_{ij}) & -\sin(2\alpha_{ij})\\  \sin(2 \alpha_{ij})  & \cos(2 \alpha_{ij}) \emat  \bmat E_j \\ B_j  \emat \Delta\Omega \,,
\eeq
%
where again the Einstein summation convention is implied and all the symbols have their usual meaning.

\comment{How does the radial kernel reduce to unity on evaluating the the operator on to its inverse ? }
%--------------------------------------------------------
%--------------------------------------------------------
\subsection{Decomposing Q \& U Stokes parameters into those corresponding to E \& B modes respectively}
The Stokes Q \& U parameters can be decomposed into the scalar modes E \& B and vice verse, as seen in the previous sections. The E \& B modes are orthogonal to each other. It is possible to decompose the Stokes Q \& U parameters into those that purely contribute to E modes and those that purely contribute to the B mode of polarization. We can only measure the total Stokes parameters which is a sum of the Stokes Q \& U corresponding to the respective scalar modes.  In this section we derive the real space operators which directly decompose the total measured Stokes Q \& U parameters to Stokes parameters corresponding to the scalar fields E \& B respectively, \textit{without ever having to evaluate the E \& B modes explicitly}. Again the procedure is analogous to that discussed in \sec{sec:qu2eb}, though the algebra is a little more involved. Here we use the harmonic space E/B projection operators $\tilde O_{E/B}$, defined in \eq{eq:eb_har_proj}, to derive the respective real space operators. It can be shown that the Stokes parameters corresponding to each scalar mode are given by the following expressions,
%
\beqry
\bar{P}_E &=&  [\bar T^{-1} * {_2B} *\tilde T * \tilde O_E* \tilde T^{-1}* {_2B^{\dagger}} *\bar T] *\bar{P}  \,, \\
&=& [\frac{1}{4} \bar T^{\dagger } * {_2B} *\tilde T * \tilde O_E* \tilde T^{\dagger} * {_2B^{\dagger}} *\bar T ]*\bar{P}  \,, \nonumber \\
&=&  \bar O_{E}*\bar{P} \,,\nonumber \\
\bar{P}_B &=&  [\bar T^{-1}* {_2B}* \tilde T* \tilde O_B* \tilde T^{-1}* {_2B^{\dagger}}* \bar T]*\bar{P}  \,, \\
&=& [\frac{1}{4} \bar T^{\dagger } * {_2B} *\tilde T * \tilde O_B* \tilde T^{\dagger} *{_2B^{\dagger}} *\bar T] *\bar{P}   \,, \nonumber\\
&=&  \bar O_{B}*\bar{P} \,. \nonumber
\eeqry
%
We contract over all the matrix operators to arrive at the the real space operators. On simplification it can be shows that the real space operator takes up the following form,
%
\beq
\bar O_{E/B} = 0.5 \bmat \mathcal{I}_{r} \pm \mathcal{D}_{r} & -\mathcal{I}_{i} \pm \mathcal{D}_{i} \\  -\mathcal{I}_{i} \pm \mathcal{D}_{i}  & \mathcal{I}_{r} \mp \mathcal{D}_{r} \emat_{2 N_{\rm pix} \times 2 N_{pix}} \,,\\
\eeq
where $\mathcal{I}_{r} ~\&~ \mathcal{D}_{r}$ and $\mathcal{I}_{i} ~\&~ \mathcal{D}_{i}$ are the real and complex parts of the following complex functions,
\beqry
\mathcal{I} &=& \mathcal{I}_{r} + i \mathcal{I}_{i} = \sum_{\ell m} {_2Y}_{\ell m}(\hat n_i) {_2Y}^*_{\ell m}(\hat n_j) \,, \nonumber \\
\mathcal{D}  &=& \mathcal{D}_{r} + i\mathcal{D}_{i} = \sum_{\ell m} {_2Y}_{\ell m}(\hat n_i) {_{-2}Y}^*_{\ell m}(\hat n_j) \,.\nonumber
\eeqry
%
These functions can be further simplified using the properties of spin spherical harmonics listed in Appendix~\ref{sec:ylm_mathprop}. Specifically it can be shown that these functions reduce to the following mathematical forms,
%
\beqrys \label{eq:fn_i}
\mathcal{I}(\hat{n}_i, \hat{n}_j) &=& \sum_{\ell} \sqrt{\frac{2\ell+1}{ 4 \pi}}{_2Y}_{\ell -2}(\beta_{ij}, \alpha_{ij}) ~ \rm{e}^{- i2 \gamma_{ij}} \label{eq:healpix-compatible-i} = \mathcal{I}_r + i \mathcal{I}_i \,, \\
\mathcal{I}_r + i \mathcal{I}_i &=& \Big [ \cos(2 \alpha_{ij} +  2\gamma_{ij}) - i \sin(2 \alpha_{ij} +  2 \gamma_{ij}) \Big]   _{-2}f(\beta_{ij},\ell_{\rm min},\ell_{\rm max}) \,,
\eeqrys
%
%
\beqrys \label{eq:fn_d}
\mathcal{D}(\hat{n}_i, \hat{n}_j) &=& \sum_{\ell} \sqrt{\frac{2\ell+1}{ 4 \pi}}{_2Y}_{\ell +2}(\beta_{ij}, \alpha_{ij}) ~ \rm{e}^{- i2 \gamma_{ij}} \label{eq:healpix-compatible-m} =\mathcal{D}_r + i \mathcal{D}_i \,, \\
\mathcal{D}_r + i \mathcal{D}_i &=&  \Big [ \cos(2 \alpha_{ij} - 2\gamma_{ij}) + i \sin(2 \alpha_{ij} -  2 \gamma_{ij}) \Big]   _{+2}f(\beta_{ij},\ell_{\rm min},\ell_{\rm max}) \,,
\eeqrys
%
where the functions,
%
\beq
{}_{\pm2}f(\beta,\ell_{\rm min},\ell_{\rm max}) = \sum_{\ell=\ell_{\rm min}}^{\ell_{\rm max}} \sqrt{\frac{2\ell+1}{ 4 \pi}} _{ \pm 2}{f}_{\ell}(\beta) \label{eq:f2_rad_ker}\,,
\eeq
%
can be expressed in terms of $P_{\ell}^2$ Legendre polynomials and are given by the following explicit mathematical forms,
 %
 \beqry
 _{\pm 2}f_{\ell}(\beta) &=& 2 \frac{(\ell-2)!}{(\ell+2)!}  \sqrt{\frac{2\ell +1 }{4 \pi}} \Bigg[ - P_{\ell}^{2} (\cos  \beta) \left( \frac{\ell-4}{\sin^2 \beta} + \frac{1}{2}\ell(\ell-1) \pm \frac{2 (\ell-1) \cos \beta}{\sin^2 \beta} \right) \nonumber \\ 
&+& P_{\ell-1}^2 (\cos \beta) \left( (\ell+2) \frac{\cos \beta}{\sin^2 \beta} \pm \frac{2 (\ell+2)}{ \sin^2 \beta } \right) \Bigg] \,. \label{eq:rad_ker_quequbqu}
 \eeqry
 %
Finally, the Stokes parameters corresponding to the respective scalar fields can be derived by evaluating the following expression, 
 %
\beqry
\bmat Q_i \\ U_i  \emat_{E/B} &=&0.5 \Bigg\lbrace {}_{-2}f(\beta_{ij},\ell_{\rm min},\ell_{\rm max}) \bmat \cos(2 \alpha_{ij} + 2\gamma_{ij}) & \sin(2\alpha_{ij} +2 \gamma_{ij}) \\  \sin(2\alpha_{ij} +2 \gamma_{ij})  & \cos(2 \alpha_{ij} + 2 \gamma_{ij}) \emat  \bmat Q_j \\ U_j  \emat   \\ &\pm& {}_{+2}f(\beta_{ij},\ell_{\rm min},\ell_{\rm max}) \bmat \cos(2 \alpha_{ij} - 2\gamma_{ij}) & - \sin(2\alpha_{ij} - 2 \gamma_{ij}) \\  -\sin(2\alpha_{ij} - 2 \gamma_{ij})  & - \cos(2 \alpha_{ij} - 2 \gamma_{ij}) \emat  \bmat Q_j \\ U_j  \emat \Bigg\rbrace \Delta\Omega \,,\nonumber
\eeqry
%
where all the symbols have their usual meaning.