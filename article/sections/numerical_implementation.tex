\section{Numerical implementation} \label{sec:numerical_implementation}

As discussed in \sec{sec:radial_locality}, the radial kernels decay ($\propto ~ \beta^{-1.5}$) as the distance from the central pixel increases, but they begin to increase on approaching the diametrically opposite position ($\beta \rightarrow \pi$)  as seen in \fig{fig:rad_ker_decay}. For a map containing sufficiently high multipole $\ell_{\rm max}$ information,  the local maxima the radial kernel attains in the vicinity of the diametrically opposite end is significantly smaller than global maxima. This suggests that for sufficiently high resolution CMB maps, it may suffice to restrict the convolution over the Stoke parameters Q \& U to a local region around the central pixel. The size of this local region is determined by the non-locality parameter $\beta_o$. \revisit{It is important to note that this claim is valid only while working with maps that are fairly homogeneous. In the case of foregrounds for example, though the radial kernel falls off, a strong foreground very far from the central pixel may still make significant contribution to the local definition of E and B modes.}

In this section we study the effects of localizing the convolution kernels on the inferred E and B mode maps and their power spectra. We have developed a Python script to compute these local convolution over Stokes Q \& U parameters. To carry out this numerical exercise, we pre-compute the radial part of the kernels, namely the function: $f(\beta), {}_{+2}f(\beta) ~\&~ {}_{-2}f(\beta)$. We use the python routine {\textit scipy.special.lpmn} for the numerical evaluation of the associated Legendre polynomial functions required to compute the respective radial kernels following equations \eq{eq:rad_ker_queb} and \eq{eq:f2_rad_ker}. Specifically while computing the functions ${}_{+2}f(\beta) ~\&~ {}_{-2}f(\beta)$ in the vicinity of $\beta=0~\&~\pi$, we use the limiting forms of the respective functions given in \app{sec:asymptotic_f}. As the next step, for each pixel on the Healpix map we get the pixel numbers of all the neighboring pixels lying within radius of $r_{\rm cutoff}$ from the central pixel using the Healpix routine {\textit query\_disc}. We then use the {\textit pix2ang} function of Healpix to get the angular coordinates of the central $(\theta_o,\phi_o)$ and its surrounding pixels $(\theta_i,\phi_i)$, which are used to calculate the corresponding Euler angles using \eq{eq:fn_euler}. Given these inputs we evaluate the convolutions as simple Reimann sums. We repeat this procedure for each pixel on the Healpix map to yielding the resultant maps (E ,B, Q, U).

\revisit{We use the CMB spectra for a fiducial cosmology and restrict our analysis on lensing induced B-mode spectra.}  For the results presented in the following sections we carry out our analysis on CMB maps at Healpix resolution of ${\rm NSIDE}=64$. In order to understand the effects of restricting the convolution to a local neighbourhood, we evaluate these convolutions on discs with progressively smaller radii surrounding the central pixel.  Specifically the non-locality parameter for a NSIDE=64 map is $\beta_o=22.5^{\circ}$. \revisit{We impose radial cutoffs of $r_{\rm cutoff}=[2\beta_0,\beta_o,0.5\beta_o,0.25\beta_o]$ with an apodization of 3 degrees having a cosine squared profiles on the edges of the discs. }We also evaluate the corresponding maps using standard Healpix routines and use these as reference maps for this exercise. Note that the Healpix evaluations are equivalent to carrying out the convolutions over the full sphere (i.e. $\beta_o=\pi$).  We compare these maps to the reference maps and their respective spectra, to quantitatively understand the effect of the imposed radial cutoff on the convolutions.

%--------------------------------------------------------
%--------------------------------------------------------
\subsection{Constructing E \& B maps using local convolutions}

%
\begin{figure}[!h] 
\centering
\subfigure[E-mode Healpix]{\includegraphics[width=0.31\columnwidth]{simulated/emap-healpix.pdf}}
\subfigure[E-mode local convolution]{\includegraphics[width=0.31\columnwidth]{simulated/emap-2beta.pdf}}
\subfigure[Difference]{\includegraphics[width=0.31\columnwidth]{simulated/emap-diff.pdf}}
\subfigure[B-mode Healpix]{\includegraphics[width=0.31\columnwidth]{simulated/bmap-healpix.pdf}}
\subfigure[B-mode local convolution]{\includegraphics[width=0.31\columnwidth]{simulated/bmap-2beta.pdf}}
\subfigure[Difference]{\includegraphics[width=0.31\columnwidth]{simulated/bmap-diff.pdf}}
\caption{{\textit Left:} Reference E \& B mode maps derived using Healpix. {\textit Middle:} E \& B mode maps derived using $r_{\rm cutoff}=2\beta_o$. {\textit Right:} Difference between the maps shown in the left and middle columns. Note that the differences maps are primarily due to differences in the recovery of large scale modes.}
\label{fig:eb-maps-compare}
\end{figure}
%
%
\begin{figure}[!h] 
\centering
\subfigure[]{\includegraphics[width=0.49\columnwidth]{simulated/ee-spectrum-radial-cutoff.pdf}}
\subfigure[]{\includegraphics[width=0.49\columnwidth]{simulated/bb-spectrum-radial-cutoff.pdf}}
\subfigure[]{\includegraphics[width=0.49\columnwidth]{simulated/relative-percentage-err-ee-spectrum-radial-cutoff.pdf}}
\subfigure[]{\includegraphics[width=0.49\columnwidth]{simulated/relative-percentage-err-bb-spectrum-radial-cutoff.pdf}}
\caption{}
\label{fig:eq-spectra_rad_cutoff}
\end{figure}
%
%--------------------------------------------------------
%--------------------------------------------------------
\subsection{Separating Stokes Q \& U maps corresponding to E \& B modes of polarization}
%
\begin{figure}[!h] 
\centering
\subfigure[E-Q map Healpix]{\includegraphics[width=0.31\columnwidth]{simulated/e-qmap-healpix.pdf}}
\subfigure[E-Q map local convolution]{\includegraphics[width=0.31\columnwidth]{simulated/e-qmap-2beta.pdf}}
\subfigure[E-Q map difference]{\includegraphics[width=0.31\columnwidth]{simulated/e-qmap-diff.pdf}}
\subfigure[E-U map Healpix]{\includegraphics[width=0.31\columnwidth]{simulated/e-umap-healpix.pdf}}
\subfigure[E-U map local convolution]{\includegraphics[width=0.31\columnwidth]{simulated/e-umap-2beta.pdf}}
\subfigure[E-U map difference]{\includegraphics[width=0.31\columnwidth]{simulated/e-umap-diff.pdf}}

\subfigure[B-Q map Healpix]{\includegraphics[width=0.31\columnwidth]{simulated/b-qmap-healpix.pdf}}
\subfigure[B-Q map local convolution]{\includegraphics[width=0.31\columnwidth]{simulated/b-qmap-2beta.pdf}}
\subfigure[B-Q map difference]{\includegraphics[width=0.31\columnwidth]{simulated/b-qmap-diff.pdf}}
\subfigure[B-U map Healpix]{\includegraphics[width=0.31\columnwidth]{simulated/b-umap-healpix.pdf}}
\subfigure[B-U map local convolution]{\includegraphics[width=0.31\columnwidth]{simulated/b-umap-2beta.pdf}}
\subfigure[B-U map difference]{\includegraphics[width=0.31\columnwidth]{simulated/b-umap-diff.pdf}}
\caption{}
\label{fig:equ-bqu-maps-compare}
\end{figure}
%

%
\begin{figure}[!h] 
\centering
\subfigure[]{\includegraphics[width=0.49\columnwidth]{simulated/equ-spectra-radial-cutoff.pdf}}
\subfigure[]{\includegraphics[width=0.49\columnwidth]{simulated/bqu-spectra-radial-cutoff.pdf}}
\subfigure[]{\includegraphics[width=0.49\columnwidth]{simulated/relative-percentage-err-equ-e-spectrum-radial-cutoff.pdf}}
\subfigure[]{\includegraphics[width=0.49\columnwidth]{simulated/relative-percentage-err-bqu-b-spectrum-radial-cutoff.pdf}}
\caption{}
\label{fig:equ-bqu-spectra_rad_cutoff}
\end{figure}
%
%--------------------------------------------------------
%--------------------------------------------------------
\subsection{Multipole filtering using spatial convolutions}
%--------------------------------------------------------
%--------------------------------------------------------
