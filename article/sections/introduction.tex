\section{Introduction}

In this work we follow the convection in which bar-ed variables correspond to those in real space, while the tilde-ed variables correspond to those in harmonic space \cite{Zaldarriaga2001a}. 

This paper is organized in the following manner: In \sec{sec:pol-intro} we present a primer on the description of CMB polarization on the sphere and introduce the matrix notation which provides a more concise description of the same. In \sec{sec:real_space_operators} we introduce the necessary tools  and discuss the derivations of the real space operators. In \sec{sec:visualize_operator} we evaluate the real space operators and present visualizations of these functions. Here we discuss the locality of the real space E \& B operators. In \sec{sec:numerical_implementation} we implement these operators to evaluate E \& B  maps from the Stokes parameters Q \& U and compare these maps and their spectra from those derived using Healpix. We conclude with a discussion and the scope of this new method of analyzing CMB polarization in \sec{sec:discussion}.

\section{E/B description of CMB polarization} \label{sec:pol-intro}

%--------------------------------------------------------
%--------------------------------------------------------
\subsection{Polarization primer}\label{sec:pol-primer}
The CMB polarization is measured in terms of Stokes Q and U parameters. These measurements can be combined to form the complex spin 2 polarization field as follows,
%
\beqry \label{eq:spin-pol}
_{\pm 2}\bar{X}(\hat{n}) &=& Q(\hat{n}) \pm i U (\hat{n}) \nonumber \\ &=& \sum_{\ell m}  {_{\pm 2}} \tilde{X}_{\ell m}  {_{\pm 2}}Y_{\ell m} (\hat{n}) \,.
\eeqry
%
Since these measured quantities depend on the local coordinate system making it is cumbersome to work with them. To overcome this, one describes the CMB polarization field in terms of a scalar field denoted by $E(\hat{n}) $ and a pseudo scalar field $B(\hat{n}) $ \cite{Kamionkowski1997}. These scalar fields are related to the spin-2 polarization field $_{\pm 2}X(\hat{n})$ through the following relation,
%
\beq \label{eq:ebdef}
\mathcal{E}(\hat{n}) = -\frac{1}{2} \big[ \bar{\eth}^2 _{+ 2}X(\hat{n})  +  \eth^2 _{- 2}X(\hat{n}) \big] ~\,;~\mathcal{B}(\hat{n}) = -\frac{1}{2i} \big[ \bar{\eth}^2 _{+ 2}X(\hat{n})  -  \eth^2 _{- 2}X(\hat{n}) \big] \,,
\eeq
%
where $\eth$ and $\bar{\eth}$ denote the spin raising and lowering operators respectively. These $E$ and $B$ fields are spin-0 fields similar to the temperature anisotropies and hence their value are independent of the coordinate system definitions. The spin raising and lowering operators have the following properties \cite{goldberg67},
%
\beqrys \label{eq:spinopylm} 
\eth _s Y_{lm}(\hat{n}) &=& \sqrt{(\ell-s)(\ell+s+1)} _{s+1} Y_{lm}(\hat{n}) \,, \\
\bar{\eth} _s Y_{lm}(\hat{n}) &=& -\sqrt{(\ell+s)(\ell-s+1)} _{s-1} Y_{lm}(\hat{n}) \,, 
\eeqrys
%
where $_s Y_{lm}(\hat{n}) $ denote the spin-s spherical harmonics.

Using \eq{eq:ebdef} and the properties of the spin raising and lowering operators given in \eq{eq:spinopylm} it can be shown that the scalar fields $\mathcal{E}/\mathcal{B}$ are defined via the following set of equations,
%
\beq \label{eq:pseudo}
\mathcal{E}(\hat{n}) = \sum_{\ell m} a^{E}_{\ell m} \sqrt{\frac{(\ell+2)!}{(\ell-2)!}} Y_{\ell m} (\hat{n}) ~\,;~ \mathcal{B}(\hat{n})  =\sum_{\ell m} a^{B}_{\ell m} \sqrt{\frac{(\ell+2)!}{(\ell-2)!}} Y_{\ell m} (\hat{n}) \,,
\eeq
%
where the harmonic coefficients of  $\mathcal{E}/\mathcal{B}$ fields are related to the spin harmonic coefficients of the polarization field through the following equations,
%
\beq\label{eq:x2eb}
a^{E}_{\ell m} = -\frac{1}{2} \Big[ {}_{+2}X_{\ell m} + _{-2}X_{\ell m} \Big] ~\,;~a^{B}_{\ell m} = -\frac{1}{2i} \Big[ {}_{+2}X_{\ell m} - _{-2}X_{\ell m} \Big] 
\eeq
%
In the remainder of this article, we will work with the scalar $E$ and pseudo scalar $B$ fields as defined by the following expressions, 
%
\beq \label{eq:pseudo}
E(\hat{n}) = \sum_{\ell m} a^{E}_{\ell m} Y_{\ell m} (\hat{n}) ~\,;~ B(\hat{n})  =\sum_{\ell m} a^{B}_{\ell m} Y_{\ell m} (\hat{n}) \,.
\eeq
%
Note that the two set of fields $\mathcal{E}/\mathcal{B}$ and $E/B$ differ, since their spherical harmonic coefficients of expansion differ by the factor of $\sqrt{\frac{(\ell+2)!}{(\ell-2)!}}$.
\filbreak	