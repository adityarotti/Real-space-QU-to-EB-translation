\section{Introduction}

In this work we follow the convection in which bar-ed variables correspond to those in real space, while the tilde-ed variables correspond to those in harmonic space \cite{Zaldarriaga2001a}. 

This paper is organized in the following manner: In \sec{sec:pol-intro} we present a primer on the description of CMB polarization on the sphere and introduce the matrix notation which provides a more concise description of the same. In \sec{sec:real_space_operators} we introduce the necessary tools  and discuss the derivations of the real space operators. In \sec{sec:visualize_operator} we evaluate the real space operators and present visualizations of these functions. Here we also discuss the locality of the real space E \& B operators. In \sec{sec:numerical_implementation} we implement these operators to evaluate E \& B  maps from the Stokes parameters Q \& U and compare these maps and their spectra from those derived using Healpix. We conclude with a discussion and the scope of this new method of analyzing CMB polarization in \sec{sec:discussion}.\\ \\ \\ 


\textbf{Some comments, questions and ideas that occurred to me while writing the draft:}
\begin{enumerate}
\item Using these local convolution can one now understand the E to B leakage occurring due to masking ? 

\item Healpix has rings at constant latitude. So the real space convolution is only perfect for pixels near the pole ? What happens when you consider pixels around any arbitrary pixel on a Healpix map, they are probably not arranged in rings. Maybe for this reason working with upgraded map works and finer pixels allow for a better approximation of rings around arbitrary pixels ??
\end{enumerate}
