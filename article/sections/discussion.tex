\section{Discussion}\label{sec:discussion}
\begin{itemize}
\item We have derived the real space kernels for translating Stokes parameters Q \& U to scalars E \& B and vice versa. We have also derived real space kernels which allow for direct separation of Stokes Q \& U parameters without having to first evaluate the scalar field E \& B.

\item These kernels quantify the non-locality of the E and B fields. We have introduced the non-locatity parameter $\beta_0$ which provides a quantitative measure of the non-locality of these fields.  

\item Studying these real space kernels reveals that its the radial part of the kernel which knows about the band limit of the experiment. Motivation for defining radially compact kernels. We have demonstrated that using the radially compact kernels does not bias the spectral information on intermediate angular scales.

\item Small field experiments like BICEP, implicitly implement such radial cut offs due to limited survey area. 


\item Using in conjunction with FEBECOP \cite{febecop} like schemes to directly infer E and B mode maps from raw maps.

\item \textit{Mask leakages} can be understood as arising from improper sine quadrupole and cosine quadrupole transforms on rings with holes in them due to masking. For the global mask (no point sources), by using a radially compact kernel with some $\beta_0$, the pixels which are at a angular distance $\beta_0$ from the edges of the mask have unbiased estimates of the scalar fields E and B.

\end{itemize}
%
%The width of the $\beta$ kernel can be used as an indicator of the width of mask apodization. One may also use the width of the $\beta$ kernel to throw away the regions of the sky close to the edges of the mask. Mask apodization techniques should fail at low Nside. The width of the $\beta$ kernel makes transparent why the E to B leakage issues are only relevant close to the edges of the mask.